% ---------------------------------------------------------------------------------------
\section*{Appendix I: Transition rates} 
% ---------------------------------------------------------------------------------------

We parameterized transition rate matrices for mortality ($\boldsymbol\Omega^{\mu}_t$), 
development ($\boldsymbol\Omega^{\gamma}$), 
and dispersal ($\boldsymbol\Omega^{\delta}_t$) as:
 %
\begin{equation} \label{eq:Theta}
\begin{aligned}
\boldsymbol\Omega^{\mu}_t & = 
\left[
\begin{array}{cc|cc}
    -\omega^{\phi_{j,s}}_t & 0 & 0 & 0 \\
    0 & -\omega^{\phi_{a,s}}_t & 0 & 0 \\
    \hline
    0 & 0 & -\omega^{\phi_{j,n}}_t & 0 \\
    0 & 0 & 0 & -\omega^{\phi_{a,n}}_t \\
    \end{array}
\right] \\
\boldsymbol\Omega^{\gamma} & = 
\left[
\begin{array}{cc|cc}
    -\omega^{\gamma_{j}} & 0 & 0 & 0 \\
    \omega^{\gamma_{j}}  & 0 & 0 & 0 \\
    \hline
    0 & 0 & -\omega^{\gamma_{j}} & 0 \\
    0 & 0 & \omega^{\gamma_{j}}  & 0 \\
    \end{array}
\right] \\
\boldsymbol\Omega^{\delta}_t & = 
\left[
\begin{array}{cc|cc}
    0 & 0 & 0 & 0 \\
    0 & -\omega^{\delta_{a,s}}_t & 0 & \omega^{\delta_{a,n}}_t \\
    \hline
    0 & 0 & 0 & 0 \\
    0 & \omega^{\delta_{a,s}}_t & 0 & -\omega^{\delta_{a,n}}_t \\
    \end{array}
\right]
\end{aligned}
\end{equation}
%
Note that mortality implicitly entails transition to a ``death state'' that is omitted 
for succinctness, as dead individuals do not contribute to future transitions.
For each transition matrix $\boldsymbol\Omega^{\alpha}_t$, 
we then calculated the probability of transitioning as
%
\begin{equation} \label{eq:Psi}
\boldsymbol\Psi^{\alpha}_t = e^{\Delta T\boldsymbol\Omega^{\alpha}_t}
\end{equation}
%
which is the solution to the differential equation associated with the Markov process
specified by $\boldsymbol\Omega^{\alpha}_t$ 
when projected over interval $\Delta T$ 
and initial condition equal to the 4 $\times$ 4 identity matrix.
The elements of $\boldsymbol\Psi^{\alpha}_t$ were then used to parameterize
the matrices given by equations \ref{eq:W} and \ref{eq:B}.
In principle, we could have included all of the demographic transitions in a single
transition matrix. 
However, modeling the different transition processes separately facilitated interpretation
of the resulting transition probabilities, as they would only pertain to a single type 
of demographic transition rather than multiple occurring simultaneously.
This was also computationally advantageous during model fitting, 
for much the same reasons.



% ---------------------------------------------------------------------------------------
\section*{Appendix II: Sensitivity analysis} 
% ---------------------------------------------------------------------------------------

We used the method of \cite{caswell2007sensitivity} to calculate the elasticities 
(proportional sensitivities) of the annual transient population growth rate $\lambda_y$
with respect to perturbations in the seasonal demographic rates.
It was convenient to perform the calculations using the logarithm of $\lambda_y$,
commonly denoted $r_y$.
This parameter is related to total population size $N_y$ by the expression
%
\begin{equation} \label{eq:r}
r_y = \text{log}\left(N_{y+1}\right) - \text{log}\left({N_y}\right).
\end{equation}
%
Note that
%
\begin{equation} \label{eq:lsens}
\frac{\text{d}\lambda_y}{\text{d}\theta} = \lambda \frac{\text{d}r_y}{\text{d}\theta}
\end{equation}
%
where $\frac{\text{d}\lambda_y}{\text{d}\theta}$ can generically be interpreted
as the sensitivity of $\lambda$ with respect to a single parameter $\theta$.
The elasticity of $\lambda$ is then defined as
%
\begin{equation} \label{eq:lelas}
\frac{\theta}{\lambda_y} \frac{\text{d}\lambda_y}{\text{d}\theta} = 
        \theta\frac{\text{d}r_y}{\text{d}\theta}.
\end{equation}
%
The multiplication of $\frac{\text{d}r_y}{\text{d}\theta}$ by $\theta$  
implies proportional perturbations in $\theta$.
Therefore, the sensitivity of $r_y$ with respect to proportional perturbations in $\theta$
equals the elasticity of $\lambda_y$.
This deduction is essentially a restatement of logarithmic relationship of $\lambda_y$
and $r_y$, along with the properties of logarithmic derivatives.

The transient sensitivity of $r_y$ with respect to perturbations in demographic parameters
is defined as
%
\begin{equation} \label{eq:dr}
\frac{\text{d}r_y}{\text{d}\boldsymbol\theta_y^\top} = 
    \frac{\mathbf{c}^\top}{N_{y+1}} \frac{\text{d}\mathbf{x}_{y+1}}
            {\text{d}\boldsymbol\theta_{y+1}^\top}-
        \frac{\mathbf{c}^\top}{N_{y}} \frac{\text{d}\mathbf{x}_y}
            {\text{d}\boldsymbol\theta_y^\top}
\end{equation}
%
where $\boldsymbol\theta_y$ is a vector of demographic parameters, 
$\mathbf{x}_y$ is a 4 $\times$ 1 vector of abundances in each state,
$\mathbf{c}$ is a 4 $\times$ 1 vector of ones,
and ``$\text{d}$'' is the derivative operator.
We were interested in the sensitivity of $r_y$ with respect to proportional 
perturbations in the seasonal demographic rates,
which are connected to $\mathbf{x}_y$ through the annual population projection matrix
$\mathbf{A}_y$ as defined in equation \ref{eq:A}.
If $\boldsymbol\theta_y$ contains the seasonal demographic rates 
(i.e., the combined elements of $\mathbf{P}_{t[y]}$ and $\mathbf{P}_{t[y]+1}$)
and $\boldsymbol\epsilon_y$ is a vector of proportional perturbations in
$\boldsymbol\theta_y$,
then 
%
\begin{equation} \label{eq:dx}
\frac{\text{d}\mathbf{x}_{y+1}}{\text{d}\boldsymbol\theta_{y+1}^\top} = 
    \mathbf{A}_y \frac{\text{d}\mathbf{x}_{y}}{\text{d}\boldsymbol\theta_y^\top}+
        \left(\mathbf{x}_{y}^\top \otimes \mathbf{I}_c \right)
            \frac{\text{dvec}\mathbf{A}_y}{\text{d}\boldsymbol\epsilon_y^\top}
                \text{diag}\boldsymbol\epsilon_y
\end{equation}
%
where $\mathbf{I}_c$ is the $c \times c$ identity matrix with $c$ as the length
of the parameter vector $\theta_y$,
``$\text{vec}$'' is an operator that creates a vector by stacking columns of the operand matrix,
and ``$\text{diag}$'' is an operator that creates a square matrix with the operand vector on
the diagonal and zeros elsewhere. 
Defining an initial population size distribution $\mathbf{x}_0$ 
that is independent of the demographic parameters implies that 
$\frac{\text{d}\mathbf{x}_0}{\text{d}\boldsymbol\theta_0^\top} = \mathbf{0}$.
Using this initial condition,
the sensitivities can then be calculated by iterating equations \ref{eq:dr} and \ref{eq:dx}
for each year, with perturbations $\boldsymbol\epsilon_y$ proportional (or equal)
to the parameter vector $\boldsymbol\theta_y$.
Asymptotic results can be obtained by iterating \ref{eq:dx} many times for a given year,
which eliminates the dependence on the initial values such that each year can be treated
independently. 
The derivatives of the elements of $\mathbf{A}_y$ with respect to the seasonal
demographic rates necessary for evaluating \ref{eq:dx} can be computed analytically
and are shown in the Supplmental Materials.


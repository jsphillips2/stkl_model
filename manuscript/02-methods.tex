

\section*{Methods}

%========================================================================================

\subsection*{Study system}

%========================================================================================

\subsection*{Population estimate}

From 1991 to 2016, surveys of the stickleback population were conducted twice annually, 
the first in either June or July and second in August or September. 
Samples were collected from 5 stations in the south basin and 3 stations in the noth basin.
These stations provided wide coverage of the two basins, 
with the exception of sites near the shoreline 
(sampling of shorline sites began in 2008 but those data are not included here)
and of the eastern portion of the south basin which historically has had neglibile 
densities of sticklebacks.
For each sampling event at each station, 
5 traps were set for two 12-hour [I think] sessions, 
one during the day and one during the night.
After trapping, individuals were sorted into two size classes,
with a threshold of 50mm in the June/July sampling and 45mm in the August/September sampling.
These size categories roughly map onto sexual maturity,
and while it is possible for individuals less than 45mm to be sexually mature,
for this study we treat the small size class as non-reproductive "juveniles"
and the large size class as potentially reprooductive "adults".

Within each basin and size class, 
station catches were of comparable magnitude and 
strongly syncrhonized through time (appendix).
Therefore, our analysis focused on the dynamics of basin-level abundances
of the two size classes through time.
We used a modified N-mixture model to estimate relative basin-level population densities
for the two size classes,
which we than scaled by the ratio of sampling areas to obtain estimates of relative
abundance between the two basins.

The probability of trapping some number of individuals $y_i$ 
was modeled as
%
\begin{equation}
  y_i \sim \text{Binomial}\left(\upphi_{g_{\upphi}[i]}, ~\upnu_{g_{\upnu}[i]}\right)
\end{equation}
%
where $\upphi_{g_{\upphi}[i]}$ is the detection probability 
and $\upnu_{g_{\upnu}[i]}$ is the population density for a given 
station-size-date combination for the $i$th observation.
The density is in units of individuals per station, 
with each station characterizing a sampling area that is 
taken to be the same size for all stations.
The functions ${g_{\upphi}[i]}$ and ${g_{\upnu}[i]}$ map observations to the 
appropriate grouping. 

The discrete population density for each station-size-date combination was modeled as
%
\begin{equation}
  \upnu_{g_{\upnu}[i]} \sim \text{Poisson}\left(\uplambda_{g_{\uplambda}[i]}\right)
\end{equation}
where $\uplambda_{g_{\uplambda}[i]}$ is the mean population density across stations 
within a basin-size-date combination. 
Variation in $\uplambda_{g_{\uplambda}[i]}$ across basin, size, and date was characterized as 
%
\begin{equation}
  \uplambda_{g_{\uplambda}[i]} \sim 
    \text{Gamma}\left(\upmu_{\uplambda}^2/\upsigma_{\uplambda}^2,
      ~\upmu_{\uplambda}/\upsigma_{\uplambda}^2\right)
\end{equation}
%
with overall mean $\upmu_{\uplambda}$ and standard deviation $\upsigma_{\uplambda}$. 
The detection probability for the $i$th observation was modeled using a 
"logistic regression"-style approach:
%
\begin{equation}
  \upphi_{g_{\upphi}[i]} = 
    \text{logit}^{-1}\left(\mathbf{x}_{g_{\upphi}[i]}^\text{T}~{\boldsymbol\upbeta}\right)
\end{equation}
%
\noindent where $\mathbf{x}_{g_{\upphi}[i]}^\text{T}$ is a transposed vector 
of predictor values for the $i$th observation
(including '1' in the first column for the overall intercept)
and $\boldsymbol\upbeta$ is a column vector of coefficients. 
We included main effects for trapping time (day vs. night) and station 
to account for  systematic differences in detection probability, 
as well as main effects and two-way interactions for size class, basin, and date 
to provide a source of "observation error" beyond the binomial sampling process. 

The model was fit using JAGS via 'runjags' in R. 
Weakly-informative gamma priors were used for
$\upmu_{\uplambda}$ and $\upsigma_{\uplambda}$, and Gaussian priors 
for the effects in ${\boldsymbol\upbeta}$. 

%========================================================================================

\subsection*{Demographic model}

We used a multiregional stage-structured model with time-varying demographic rates
to characterize the dynamics of the stickleback population. 
The model included mortality, transitions between life stages, movement between basins,
and recruitment of new individuals into the population. 
These demographic rates were organized into matrices,
which could then be used to project the abundance of individuals in each 
stage and basin through time.
We inferred the demographic rates by  
fitting the model to the times series of abundance estimates,
which is sometimes referred to as "inverse modeling". 
For parameter estimation, 
the model was formulated to match the semiannual nature of the abundance estimates.
However, our demographic analysis of the model focused on the annual dynamics, 
both to circumvent interpretational challenges 
posed by the unequal sampling intervals within years
and to better match the life history of threespine stickleback in M\'{y}vatn.

Time-dependent transitions between stages and basins 
were modeled with a $4\times{4}$ transition rate matrix $\mathbf{Q}_t$ defined as
%
\begin{equation} \label{eq:Q}
\mathbf{Q}_t = 
f\left(
\left[
\begin{array}{c|ccc}
    \overset{s}{\mathbf{U}_t} & \overset{n\rightarrow s}{\mathbf{D}_t} \\
    \hline
    \overset{s\rightarrow n}{\mathbf{D}_t} & \overset{n}{\mathbf{U}_t} 
    \end{array}
\right]
\right)
\end{equation}
%
where $t$ is time, 
$\overset{i}{\mathbf{U}_t}$ is a $2\times{2}$ matrix of stage-transition rates
for basin $i$ ($s$ for south and $n$ for north), and
$\overset{i\rightarrow h}{\mathbf{D}_t}$ is a $2\times{2}$ matrix 
of basin-transition rates from basin $i$ to $h$.
The function $f$ maps the sub-matrices to a full transition rate matrix that 
quantifies the overall loss rate from each stage-basin combination by 
subtracting the sum of the off-diagonal elements from the diagonal element 
for each column (the fully expanded form of $\mathbf{Q}_t$ is shown in the appendix).

The stage-transition matrix for basin $i$ was defined as
%
\begin{equation} \label{eq:U}
\overset{i}{\mathbf{U}_t} = 
\left[
\begin{array}{cccc}
    -\overset{j_i}{\upmu_t} & 0 \\
    \overset{j_i\rightarrow a_i}{\upgamma_t} & -\overset{a_i}{\upmu_t}
    \end{array}
\right]
\end{equation}
%
where $\overset{k_i}{\upmu_t}$ is the mortality rate for stage $k$ in basin $i$
($j$ for juveniles and $a$ for adults)
and $\overset{j_i\rightarrow a_i}{\upgamma_t}$ 
is the development rate from juveniles to adults.
Note that mortality implicitly entails transition to a "death state" that is omitted 
for concision, as dead individuals do not contribute further to the dynamics. 
The basin-transition matrix was defined as
%
\begin{equation} \label{eq:D}
\overset{i\rightarrow h}{\mathbf{D}_t} = 
\left[
\begin{array}{cccc}
    \overset{j_{i}\rightarrow j_{h}}{\updelta_t} & 0 \\
    0 & \overset{a_{i}\rightarrow a_{h}}{\updelta_t}
    \end{array}
\right]
\end{equation}
where $\overset{k_{i}\rightarrow k_{h}}{\updelta_t}$ is the movement rate for stage $k$ 
from basin $i$ to $h$. 
While $\mathbf{Q}_t$ does not explicitly include transitions from juveniles to adults 
across basins, these transitions occur implicitly through the combined effects of
development and movement over a given time interval.

In order to project the transition dynamics in discrete time, 
we related the transition \emph{rate} matrix $\mathbf{Q}_t$ 
to the transition \emph{probability} matrix $\mathbf{P}_t$ 
with the following equation
%
\begin{equation} \label{eq:P}
\mathbf{P}_t = e^{ \upDelta T \mathbf{Q}_t}
\end{equation}
%
which is the solution to the differential equation associated with the Markov process
specified by $\mathbf{Q}_t$ when projected over interval $\upDelta T$.

Appearance of new individuals in the population was modeled with
a $4\times{4}$ recruitment matrix 
%
\begin{equation} \label{eq:Q}
\mathbf{R}_t = 
\left[
\begin{array}{c|ccc}
    \overset{s}{\mathbf{R}_t}  & \mathbf{0} \\
    \hline
    \mathbf{0} & \overset{n}{\mathbf{R}_t}
    \end{array}
\right]
\end{equation}
%
comprising basin-specific sub-matrices
%
\begin{equation} \label{eq:R}
\overset{i}{\mathbf{R}_t} = 
\upDelta T
\left[
\begin{array}{cccc}
    0 & \overset{i}{\uprho_t} \\
    0 & 0
    \end{array}
\right]
\end{equation}
%
where $\overset{i}{\uprho_t}$ is the per capita recruitment rate for basin $i$,
with the assumption that individuals in the small size class ("juveniles") cannot reproduce. 
Recruitment includes both offspring production and survival until 
they are observed as "juveniles" at the next time step. 
This formulation assumes that individuals born in a given basin remain there until the
next time step, at which point they can disperse according 
to $\overset{i\rightarrow h}{\mathbf{D}_t}$.
Note that while we parameterized recruitment as a "rate" to account for
unequal projection intervals, recruitment is not evenly distributed throughout the 
year and is likely concentrated in the summer 
(although not necessarily within the "summer" interval between June and August sampling).

Given the time-dependent transition probability and recruitment matrices, 
the population dynamics were projected as
%
\begin{equation} \label{eq:RPX}
    \mathbf{x}_t = (\mathbf{P}_{t-1} + \mathbf{R}_{t-1})~\mathbf{x}_{t-1}
\end{equation}
%
where $\mathbf{x}_{t-1}$ is a vector of stage- and basin-specific abundances at time $t-1$:
%
\begin{equation} \label{eq:X}
\mathbf{x}_{t-1} = 
\left[
\begin{array}{cccc}
    \overset{j_s}{x_{t-1}} \\
    \overset{a_s}{x_{t-1}} \\
    \overset{j_n}{x_{t-1}} \\
    \overset{a_n}{x_{t-1}}
    \end{array}
\right]
\text{.}
\end{equation}
%

The elements of $\mathbf{P}_t + \mathbf{R}_t}$ characterize the overall contribution of 
each state to each other state,
with each contribution integrating multiple underlying demographic rates.
Becuase it is these overall state contribution that manifest in the data,
the underlying demographic rates can be interpreted as latent variables 
that underpin the state contributions that are the primary focus of the analysis.

We modeled temporal variation in the latent demographic rates 
(i.e. the non-zero elements of 
$\overset{i}{\mathbf{U}_t}$, $\overset{i\rightarrow h}{\mathbf{D}_t}$, 
and $\overset{i}{\mathbf{R}_t}$)
as 
%
\begin{equation} \label{eq:alpha}
\upalpha_t = \upalpha'_t\times e^{\upbeta_{\upalpha}~b_t}
\end{equation}
%
where $\upalpha_t$ is a given demographic parameter,
$\upalpha'_t$ is the value from a corresponding random walk, 
$\upbeta_{\upalpha}$ is a fixed season effect unique to that demographic parameter,
and $b_t$ is a binary index for season (summer = 0 and winter = 1). 
The fixed season effects allowed us to account for systematic differences between the 
sampling intervals that remained consistent throughout the entire time series.

The random walk for each parameter was modeled as
%
\begin{equation} \label{eq:walk}
\upalpha'_t \sim \text{Normal}
\left(
      \upalpha'_{t-1},~\upsigma_{g[\upalpha]}
\right) \text{T}(0, \infty)
\end{equation}
%
with standard deviation $\upsigma_{g[\upalpha]$ and 
truncated from the left ensure that values remained positive. 
The function $g$ maps $\upalpha$ to a given type of demographic process 
(stage-transition, basin-transition, or recruitment),
such that a single random walk standard deviation was used 
for each demographic process type.
While formulated as random walks, 
the realized sequences of inferred demographic rates 
were not truly random walks because they were constrained by fitting the model to data.
Therefore, the "random walks" are best understood as a convenient method 
for allowing the demographic rates to vary through time with implicit "smoothing"
arising from the autocorrelated nature of the walks.

We fit the model in a Bayesian framework.
The prior distributions for random walk standard deviations and initial values were 
modeled using gamma distributions with shape parameter 1.5 and rate parameters set 
to provide a reasonable scale for the corresponding parameter (see appendix). 
This prior has zero density at 0 but is also concave down near zero, 
and therefore allows posterior values arbitrarily close to zero without 
artificially concentrating density at zero. 
Furthermore, its has a long tail and is weakly informative relative to its overall scale. 

% The initial values of the random walks had a tendency to concentrate 
% near the mean of the prior, 
% which is unsurprising given that purpose of the random walks was to allow the 
% demographic rates to move towards reasonable values as informed by the data, 
% which in turn provided limited information for initial value itself.
% To compensate for this, we penalized the initial values for deviating from the
% realized means of the random walks for each step in the MCMC sampling using 
% a normal distribution scaled to have a maximum density of 1 
% (i.e. have no contribution to the posterior) 
% when the initial value equaled the random walk mean,
% and then declined as the initial value deviated from the mean
% at a rate controlled by the corresponding 
% random walk standard deviation $\upsigma_{g[\upalpha]$.
% This way the estimates of initial values were informed by the full time series.

The likelihood of the observed data given the projected abundances was calculated as
%
\begin{equation} \label{eq:likelihood}
\mathcal{L} = 
\displaystyle\prod_{r}
\displaystyle\prod_{t}
\text{Normal}
\left(
\overset{r}{y_t}~|~\overset{r}{x_t},~\upsigma_y
\right)
\end{equation}
%
with standard deviation $\upsigma_y$.
For population densities that are necessarily non-negative, 
it is common to model the likelihood using a distribution that is similarly constrained,
such as a log-normal distribution. 
However, in our case using a log-normal distribution resulted in estimates of 
$\overset{r}{x_t}$ that "ignored" large fluctuations in the observed data 
that likely reflected real changes in the population but were compressed on a log scale. 
Therefore, we opted for a normal (Gaussian) likelihood. 
Because the model was parameterized in a way that ensured
$\overset{r}{x_t}$ was non-negative,
the posterior distribution of $\overset{r}{x_t}$ was also guaranteed to be non-negative
even though normally distributed variables can take negative values. 
However, to model the initial abundances using a normal distribution,
it was necessary to truncate from the left at zero:
%
\begin{equation} \label{eq:x0}
\overset{r}{x_{t=1}} \sim
\text{Normal}
\left(
\overset{r}{y_{t=1}},~\upsigma_y
\right) \text{T}(0, \infty)
\text{.}
\end{equation}
%

We fit the model using Stan 2.19 run from R 3.5.1
with the ‘rstan’ package.
Convergence was assessed by examining the effective sample size 
per iteration of the Markov chain, 
the number of divergent transitions, 
and the potential scale reduction factor (\^{R}),
which quantifies the relative variance within and between chains. 
For all parameters, each of these diagnostics fell within 
acceptable ranges as defined by Stan 2.19.

%========================================================================================

\subsection*{Annual dynamics}

While we parameterized the model in terms of continuous rates
to accommodate the seasonal nature of the data,
we focused our demographic analysis on the annual projection matrix, 
defined as 
%
\begin{equation} \label{eq:A}
\mathbf{A}_y = \left(\mathbf{P}_{\tau[y]+1} + \mathbf{R}_{\tau[y]+1}\right) 
                \left(\mathbf{P}_{\tau[y]} + \mathbf{R}_{\tau[y]}\right)
\end{equation}
%
for year $y$ and sequential time steps within that year $\tau[y]$ and $\tau[y]+1$,
with the year defined to start with the June/July census.
$\mathbf{A}_y$ projects the dynamics from
June/July of one year to June/July of the next year.
Analyzing the annual projection matrix allowed us to circumvent interpretational
issues that arose from unequal sampling intervals within years
and aligned with the annual nature of spawning for M\'{y}vatn stickleback.
To characterize the contribution of different demographic processes 
to the population dynamics,
we calculated the annual contribution of each stage $\times$ basin combination 
to each other stage $\times$ basin combination in the next year as
%
\begin{equation} \label{eq:A}
\mathbf{C}_y = \mathbf{A}_y~{\circ}~\mathbf{X}_{t[y]}
\end{equation}
%
where $\mathbf{C}_{y}$ is a $4\times 4$ matrix of annual contributions,
$\mathbf{X}_{t[y]}$ is a $4\times 4$ matrix 
with idententical rows ${\mathbf{x}_{t[y]}}^\text{T}$,
and ${\circ}$ denotes the Hadamard (element-wise) product. 
This is essentially the same operation as a standard demographic projection
(e.g., equation \ref{eq:RPX}) but without summing the contributions to a given state
from different sources states. 


We characterized the overall dynamics of the population in terms of the annual 
population growth rate $\uplambda_y$, calculated as
%
\begin{equation} \label{eq:lam-n}
\uplambda_y = \frac{N_y}{N_{y-1}}
\end{equation}
%
where $N_y$ is the summed abundance across basins and life stages in year $y$.
Temporal variation in $\uplambda_y$ reflects both variation in the demographic rates
and transient fluctuations due to non-equilibrium state distribution. 
Therefore, it is also informative to calculate the asymptotic population growth rate that 
would obtian if the demographic rates in a given year were fixed indefinitely, 
which is equal to the leading eigenvalue of $\mathbf{A}_y$. 
The asymptotic growth rate $\uplambda^{'}_y$ implies population dynamics
%
\begin{equation} \label{eq:n-prime}
N^{'}_y = \uplambda^{'}_y N^{'}_{y-1}
\end{equation}
%
where $N^{'}_y$ is the analogue to $N_y$ that would obtain in the absence of 
trasience due to non-equilibrium state distributions.
We characterized the importance of transience for the abundance in a given year
as the proportional deviation of the realized population size from asymptotic analogue:
%
\begin{equation} \label{eq:dev}
\kappa = \frac{N_y - N^{'}_{y}}{N^{'}_y}
\end{equation}
%

We related temporal variation in the realized population growth rate 
to changes in the demographic rates 
through both "prospective" and "retrospective" analyses \citep{caswell2001matrix}.
The prosepctive analysis assessed how the population 
would respond to hypothetical perturbations at each time step,
which we quantified as
the partial derivaties (i.e. sensitivities) of $\uplambda_y$ 
with respect to perturbations in elements of $\mathbf{A}_y$ 
according to \cite{caswell2007sensitivity}. 
While proportional sensitivies (i.e. elasticities) 
are commonly used in demographic analyses,
we used sensitivies to retain information about variation in $\uplambda_y$
that is normalized out in calculation of elasticities. 
When projected over a single time step, 
the sensitivity of $\uplambda_y$ with respect to a given element of $\mathbf{A}_y$
equals the relative abundance 
of the contributing population state (Appendix).
In other words, the population growth rate is most sensitive to perturbations
in the contibutions of abundant population states.
However, when perturbations to elements of $\mathbf{A}_y$ are propogated through time,
the sensitivities also include fluctuations
in relative abundances that themselves arise from elements of $\mathbf{A}_y$,
as occurs implicitly for conventional sensitvity analysis of the asymptotic growth rate.

The retrospective analysis decomposed annual variation 
in the realized population growth rate 
into contributions from different demographic processes 
according to \cite{koons2016life} and \cite{koons2017understanding}.
For a given time step, 
the contribution of a change in a demographic rate to the change in $\uplambda_y$
has first-order approximation
%
\begin{equation} \label{eq:l-cont}
\text{contribution}^{\upDelta \uplambda_y}_{\uptheta_k} \approx
(\uptheta_{k,y+1} - \uptheta_{k,y})
\left.\frac{\uppartial\uplambda}{\uppartial\uptheta_k}
\right\vert_{\overline{\boldsymbol\uptheta}}
\end{equation}
%
where $\uptheta_{k,y}$ is element $k$ of a vector $\boldsymbol\uptheta_y$ 
containing elements of the projection matirx $\mathbf{A}_y$ and 
the state distribution vector $\hat{\mathbf{x}}_{\tau[y]}$, 
with total abundance of the state distribution 
standardized to sum to one for each time step.
The partial derivatives in equation \ref{eq:l-cont}
are evaluated for a vector $\overline{\boldsymbol\uptheta}$
containing average values for each element across years.
Similarly, the overall contribution of variance in $\uptheta_k$ 
to variance in $\uplambda_y$ is approximated as 
%
\begin{equation} \label{eq:l-var}
\text{contribution}^{\text{var}(\boldsymbol\uplambda)}_{\uptheta_k} \approx
\sum_{h}\text{cov}(\boldsymbol\uptheta_k, \boldsymbol\uptheta_h)
\left.\frac{\uppartial \uplambda}{\uppartial \uptheta_k}
\frac{\uppartial\uplambda}{\uppartial\uptheta_h}
\right\vert_{\overline{\boldsymbol\uptheta}}
\end{equation}
%
where $\boldsymbol\uplambda$, and $\boldsymbol\uptheta_k$ are vectors
of population growth and demographic rates across years.
While we included contributions of the state distribution 
to changes and variance in $\uplambda$ in our calculations,
these contributions were small relative to the direct contributions 
of the demogoraphic rates.
Therefore, to simplify the presentation we focused on contributions of the demogoraphic rates; 
full results are included in the appendix.

%========================================================================================

\subsection*{For appendix}

The population growth rate is related to elements of the annual project matrix
according to the following relationship:
%
\begin{equation} \label{eq:lam-A}
\uplambda_y = \frac{\sum_{j}\sum_{i}a_{ij}x_j}{\sum_{k}x_k}
\end{equation}
%
where $i$ and $j$ denote columns of the annual projection matrix associated with elements
$a_{ij}$,
and $x_j$ is the element of the population distribution vector associted with column $j$.
While $a_{ij}$ and $x_j$ vary annually, the year index is dropped for clarity.

If projected over a single interval and neglecting any dependence of the state
distribution on previous demographic rates, 
the sensitivity of $\uplambda_y$ with respect to element $a_{ij}$ is 
%
\begin{equation} \label{eq:lam-A}
\frac{\uppartial\uplambda_y}{\uppartial a_{ij}} = \frac{x_j}{\sum_{k}x_k}\end{equation}
%

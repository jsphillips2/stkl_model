
\section*{Abstract} \label{abstract}

Population fluctuations are of enduring interest in ecology,
particularly when they are spatially synchronized across 
multiple subpopulations.
Disentangling the diverse mechanisms underpinning spatially-structured fluctuations
is complicated by transient phenomena arising from non-equilibrium population structure.
In this study, we used a multiregional stage-structured model
with to characterize the  fluctuations of two spatially-coupled subpopulations
(north and south) of threespine stickleback in M\'{y}vatn, Iceland.
By fitting this model to a 29-year times series of structured abundance,
we were able to quantify temporal variation in demographic rates
underpinning the large population fluctuations.
According to the model, the fluctuations were cyclic
at the whole-lake scale with a period of approximately 6 years.
These fluctuations were similar in the two subpopulations 
and were underpinned by variation in demographic rates 
that were similarly correlated through space.
However, the population also displayed source-sink dynamics,
with the subsides from north required to sustain the subpopulation the south.
Furthermore, the cyclic nature of the fluctuations was partially obscured by transience
due to non-equilibrium population structure,
underscoring the value of explicit models of structured population dynamics.
Our results illustrate how both synchronized demographic rates and spatial coupling
through dispersal influence the character of spatially-structured population dynamics.


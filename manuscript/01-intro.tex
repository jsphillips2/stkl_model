\section*{Introduction} \label{introduction}

Population fluctuations are of enduring interest in ecology,
with particular attention paid to periodic or quasi-periodic dynamics
\citep{elton1924, myers2018}.
Various mechanisms have been posited for cyclic fluctuations,
including periodic environmental variation and 
interspecific interactions (e.g., predator-prey; host-parasitoid)
that induce periodicity endogenously 
\citep{nicholson1935, andrewartha1954, rosenzweig1963}. 
Furthermore, exogenous and endogenous factors can interact with each other,
with the potential to produce a wide array of dynamics with
varying degrees of periodicity, as well as more exotic features
\citep{bjornstad2001, turchin2003, ives2008}.

Many populations are spatially structured \citep{hanski1998},
setting the stage for spatially-structured fluctuations 
\citep{bjornstad1999spatial, gouveia2016}.
Indeed, some populations are synchronized across wide geographic extents,
which raises the question of how such synchrony arises
\citep{ranta1995synchrony, krebs2002}.
Perhaps the most widely discusssed mechanism is the Moran effect,
whereby spatial synchrony in some exogenous driver induces synchrony
in the populations under its influence
\citep{moran1953}. 
Spatial coupling through dispersal between sub-populations of a given species
\citep{liebhold2004, goldwyn2008}
or movement of another species with which the focal sub-populations interact
\citep{gilg2009, ims2000}
is another important mechanism of synchronous fluctuations.
Synchrony in exogenous drivers and spatial coupling through dispersal 
can interact with each other and with the internal dynamics of a given population,
complicating the way in which these factors map onto population synchrony
\citep{ranta1995synchrony, kendall2000dispersal, abbott2011}.

Both the Moran effect and spatial coupling can be characterized as demographic processes,
with the former inducing synchrony in demographic rates such as recruitment or survival
and the latter arising from immigration between sub-populations.
From this framing,
a third source of population fluctuations (potentially synchronous) becomes clear:
trasient oscillations due to non-equilibrium distribution of individuals across
demographic states
\citep{caswell2001matrix, koons2017understanding}.
In age- and stage-structured populations, 
such transient oscillations can be very large and alter the long-term dynamics 
through various mechanisms,
such as altering the time-averaged growth rate
\citep{tenhumberg2009}. 
In spatially-structured populations, 
discerete patches play an analagous role to discrete life-stages 
and can similarly exhibit non-equilibrium distributions with associated transience
\citep{ozgul2009}.
When demographic rates vary through time,
as is inherent to the Moran effect,
a population may persistently occupy non-equilibrium state distributions,
which in turn means that transient oscillations may be ubiquitous features of the dynamics.
Therefore, accounting for effects transience may be important for 
fully characterizing the nature and causes of spatially-structured fluctuations
\citep{hastings2010}.

In this study, 
we use a population-dynamics model to analyze spatially-structured fluctuations 
in a population of threespine stickleback (\emph{Gasterosteus aculeatus})
in Lake M\'{y}vatn, Iceland.
M\'{y}vatn is particularly well suited to this exploration for two primary reasons.
First, the stickleback population is spatially structured by the unique geomorphology
of the lake \citep{gislason1998, millet2013}, 
which is divided into two basins connected by two narrow channels.
The southern basin (Syðrifloi) is the larger of the two ($28~\text{m}^2$) and is dominated
by exposed sediment and intermittent mats of filamentous green algae
\citep{einarsson2004myvatn}.
In contrast, the northern basin (Ytrifloi) is substantially smaller ($9~\text{m}^2$)
and more spatially heterogeneous, 
in part due to dredging of the lake bottom that substantially altered its bathymetry.
Despite its smaller area, the north basin has historically sustained much higher
densities of threespine stickleback  \citep{gislason1998}, 
likely owing to the ecological differences between the basins. 

Second, M\'{y}vatn's stickleback population fluctuates substantially through time.
While the causes of these fluctuations are unknown, 
they are likely connected to the large temporal variability 
of other populations in the lake.
M\'{y}vatn is naturally eutrophic due to the inflows of nutrient-rich springs,
which sets the stage for high-amplitude fluctuations in secondary producers
\citep{einarsson2004myvatn}.
Chief among these are chironomid midges and cladocerans
\citep{einarsson2002, einarsson2004clad, gardarsson2004population, ives2008},
both of which are potential food sources for threespine stickleback.
Furthermore, the lake hosts temporally variable populations 
of arctic charr (\emph{Salvelinus alpinus}), 
brown trout (\emph{Salmo trutta}), 
and piscivorous birds that have the potential 
to induce fluctuations in the stickleback population from the top down
\citep{einarsson2004moulting, gudbergsson2004}.
Finally, M\'{y}vatn's stickleback can sustain high loads of the cestode parasite
\emph{Schistocephalus solidus} \citep{gislason1998, karvonen2013},
which could lead to host-parasite fluctuations.

To explore the spatiotemporal dynamics of M\'{y}vatn's stickleback population, 
we fit a stage-structured metapopulation model \citep{caswell2001matrix}
to a 29-year times series of population estimates derived from trapping data.
The model includes temporal variation in demographic rates such as recruitment and survival,
which were characterized as random walks constrained by the observed data.
This approach provides great flexibility in modeling changes in the demographic rates,
including those implicitly arising from nonlinear and density-dependent processes
\citep{zeng1998, ives2012}.
Equipped with the parameterized model, we address the following questions:
%
\begin{enumerate}[label=(\alph*)]
\item
To what extent are stickleback population fluctuations periodic, 
and to what extent does transience arising from non-equilibrium state distributions
alter this periodicity;
%
\item
To what extent are the two sub-populations coupled through dispersal,
and how does this affect the dynamics; 
%
\item
To what extent are the demographic rates synchronized between the two sub-populations,
and what are the relative contributions of within-basin demography 
and between-basin coupling to the lake-wide population dynamics.
\end{enumerate}
%
By investigating these topics, 
we seek to illustrate the demographic basis for spatially-structured population fluctuations
in a wild population, 
which may provide general lessons for other systems. 
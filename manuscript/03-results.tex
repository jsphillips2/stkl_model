% ---------------------------------------------------------------------------------------
\section*{Results}
% ---------------------------------------------------------------------------------------



% ---------------------------------------------------------------------------------------
\subsection*{Model fit and seasonal dynamics} 
% ---------------------------------------------------------------------------------------

The M\'{y}vatn stickleback population fluctuated substantially over the past
three decades, 
and the model largely succeeded in characterizing these fluctuations 
(Figure \ref{fig:fit}).
This stands in contrast to the reduced model with temporally-fixed demographic rates,
which largely failed to capture the population fluctuations.
Therefore, the population fluctuations cannot be purely explained in terms of transient
oscillations, but rather arise at least in part from changes in demographic rates per se. 
There was a close correspondence between the dynamics of the adults in both basins,
although abundance in the south basin was particularly low 
in the second half of the time series.
Juveniles were less obviously correlated with each other 
and displayed "spikier" fluctuations than the adults.
In general, juveniles were substantially more abundant in the north basin than in the south,
with the exception of occasional peaks. 
The high abundance of south basin juveniles in June of 1992 appears quite extreme
relative to the rest of the time series for both basins and life stages.
However, this pattern was observed across multiple traps and locations within the south basin,
suggesting that it reflected genuinely elevated abundance.

Per capita recruitment varied substantially across years and was generally of comparable
magnitude between the two basins (Figure \ref{fig:rec}).
Similar to juvenile abundance, per capita recruitment in the south 
was characterized by relatively sudden and short-lived peaks, 
in comparison with the north basin.
Total recruitment corresponded very closely with juvenile abundance (Figure \ref{fig:fit}),
reflecting the model's inference 
that few juveniles return to the juvenile class at the next time step (see below).
As noted above, 
any cross-basin recruitment would be attributed 
to within-basin recruitment by the model. 
However, gravid females and males with breeding coloration are routinely sampled in the 
south basin, indicating that much of the recruitment within the south basin is 
likely attributable to spawning in the south basin.

In the first half of the time series, 
survival probabilities of north basin juveniles and of adults in both basins were strongly
associated (Figure \ref{fig:surv}), 
with two peaks in survival corresponding with similar peaks in observed abundance 
(Figure \ref{fig:fit}).
This covariation in survival probabilities was less pronounced in the second half 
of the time series, although still present for the north basin juveniles and adults.
Overall, this suggests that survival probability was affected by processes relevant across 
the life cycle and present at the scale of the entire lake.
In contrast to the other population states, 
survival probability of south basin juveniles was very low,
although still high enough to constitute a meaningful contribution over the one or two-month
``summer'' projection interval.
While the difference between the survival probability of south basin juveniles and 
the other states is quite pronounced, 
the basis for this inference is apparent 
in the population estimates (Figure \ref{fig:fit}). 
Large, sudden increases in the abundance of south basin juveniles appear to contribute
to elevated abundance of neither juveniles nor adults at the next time step,
which directly implies low survival.
The model inferred a high development rate from the juvenile to the adult stage,
with a probability of 92\% [88\%, 95\%] over a six-month period. 
Note that the model implicitly treats all individuals within a given state 
as demographically identical,
so this development probability can be interpreted as the average 
for individuals classified as juveniles, 
including individuals that are just below the size threshold for being classified as "adults".
This high development probability reflects the apparent correlation between north basin
juveniles and adults with a lag of approximately one time step (Figure \ref{fig:fit}).
As noted above, we fixed the development probability through time and for both basins
to facilitate estimation of the other demographic parameters.

The model formulates dispersal reciprocally,
and the dispersal probability is best understood as a measure of the extent 
to which the sub-populations appeared to be dynamically mixed
expressed on a per capita basis.
By this measure, the north and south basins appeared 
to be substantially coupled (Figure \ref{fig:disp}), 
which is consistent with the close correspondence 
between the adult dynamics across basins.
To clarify the contribution of dispersal to the dynamics, 
we calculated the net movement of individuals between basins,
with positive values indicating movement from south to north (``northward'')
and negative values indicating the converse (``southward'').
In each time step, 
net movement was generally low and slightly southward,
with a few of peaks in net movement corresponding to 
both peaks in abundance
and changes in the difference between north and south dispersal probabilities.
The cumulative movement over the entire time series was clearly southward,
resulting in a substantial net subsidization of the south basin over that period.



% ---------------------------------------------------------------------------------------
\subsection*{Annual dynamics and sensitivity analysis} 
% ---------------------------------------------------------------------------------------

The transient population growth rate ($\lambda$) fluctuated substantially between years
(Figure \ref{fig:lam}),
with a geometric mean of 0.98 [0.97, 0.99] implying a slight average decline
of $2\%~\text{year}^{-1}$ ($\lambda = 1$ would indicate no change).
Variance per se reduced the transient growth rate, 
as the arithmetic mean (1.19 [1.14, 1.24]) was much higher than the geometric mean
and would have implied substantial growth ($19\%~\text{year}^{-1}$)
if realized in the population.
The asymptotic growth rate was similar in this regard,
although the geometric mean (1.03 [1.00, 1.06]) was slightly higher 
than for the transient growth rate.
This is because the asymptotic growth rate lacked the extreme fluctuations 
of the transient growth rate,
leading to a lesser reduction relative to the arithmetic mean (1.20 [1.14, 1.27]).
These results show that variance arising from both changes in demographic
rates and transience per se reduced the the average population growth rate,
although effect of the former predominated.

The wavelet decomposition detected a strong periodicity of around 6 years
in the asymptotic growth rate over the first two decades, 
which is clearly visible in the corresponding time series (Figure \ref{fig:lam}).
The wavelet decomposition of the transient growth rate was generally similar,
although the 6-year periodicity was weaker.
Furthermore, there was a highly localized periodicity of around 3 years 
at the beginning of the time series 
that coincided with the rapid increase and subsequent decline of south basin juveniles.
Together, these results indicate that the population displayed cyclic dynamics 
driven by changes in demographic rates.
However, these cycles were partially obscured by transient fluctuations arising 
from the non-equilibrium state distributions.
The absence of conspicuous periodicity in the final decade
for both the asymptotic and transient growth rates 
suggests a shift in the dynamics, 
although it is difficult to draw strong conclusions 
based on relatively short time series.

The asymptotic growth rate declined with total abundance
(slope = -0.13 [-0.17, -0.11]; Figure \ref{fig:dens}).
Futhermore, 
the intercept of the linear model implies that 
the asymptotic growth rate approaches 1.77 [1.59, 2.02] 
as population size approaches zero,
which is much higher than the ``observed'' geometric mean of 1.03.
Together, the slope and intercept imply an equilibrium abundance (i.e., $\lambda=1$)
of 4.25 [3.97, 4.52] relative abundance units,
about which there were substantial fluctuations in overall abundance 
(x-axis in Figure \ref{fig:dens}).
These results suggest that negative density dependence
had a large effect on the overall population dynamics,
which is consistent with the periodicity of the asymptotic growth rate.
While we did not attempt to formally assess whether the strength of density dependence
changed through time,
no such shift was visually apparant in Figure \ref{fig:dens}.
Despite the strong negative relationship between the growth rate and population size, 
there was substantial residual variation with modest
temporal autocorrelation (0.203 [0.0826, 0.307]) associated with this linear trend.
This indicates that there were processes, either density dependent or independent,
relevant for the dynamics that were not fully captured in the linear model.

The elasticities in Figure \ref{fig:elas} quantify the proportional change in the transient
growth rate that would result from a proportional perturbation to a given demographic 
rate.
An elasticity of 1 indicates that a 1\% change in a demographic rate would result in
a 1\% in the population growth rate.
In general, the elasticities were positive,
which is expected as increases in processes such as survival or recruitment
should contribute positively to the growth rate.
However, the elasticity with respect to dispersal from north to south was negative,
which means that increases in southward dispersal 
tended to reduce the population growth rate.
This reflected unfavorable demographic conditions in the south basin,
chiefly attributable to low juvenile survival,
but also due to somewhat lower adult survival and inconsistent recruitment.
Counter-intuitively, the elasticity with respect to south basin recruitment was
slightly negative in some years. 
This was a transient phenomenon,
as demonstrated by the corresponding elasticities 
for the asymptotic growth rate that were positive
(X; the asymptotic results were otherwise largely the same as the transient ones).
The elasiticities with respect to 
demographic rates originating in the north basin were all of greater
average magnitude than those originating in the south,
indicating that that total population growth rate was most sensitive to processes
occurring in the north basin.
Among those north basin rates, juvenile survival and recruitment were associated with
the largest elasticities,
suggesting that the dynamics were most sensitive to processes early in the life cycle.
While the elasticities varied somewhat across years, 
these differences were modest relative to the differences among the demographic rate types.
Note that the temporal variation in the elasticity of development arose from
variation in the other demographic rates, 
as the development probability itself was fixed through time.












\section*{Results}

%========================================================================================

\subsection*{Population estimates \& model fit}

The median population densities for both size classes were much higher
in the north basin (330 and 228 per station for juveniles and adults, respectively)
than in the south basin (82 and 41 per station).
The sampled area of the south basin 
was approximately twice as large as the north basin.
Therefore, the disparity in the relative abundances between the two basins was 
lower than for population densities,
although median abundances were still higher in the north basin.
There were large fluctuations in the relative abundances through time
for all stages and basins (Figure \ref{fig:fit}). 
This was especially true for adults in the south basin, 
which generally had low abundance with the exception of two large peaks
where their densities exceeded those for any other stage $\times$ basin combination.
There was a lake-wide decline in abundance in the last decade of the time series,
resulting in collapse in 2016. 
The decline was especially pronounced for juveniles in the north basin,
which reached a peak in abundance shortly before the collapse.
The full model succeeded in characterizing the major features of the dynamics, 
and provided a better fit than the alternative models excluding movement 
between the basins and models fixing the demographic rates through time. 
[Note: this was quantified by both the posterior log-likelihood
and the approximation to leave-one-out cross validation; I'll show this in a table].

%========================================================================================

\subsection*{Annual dynamics}

While we parameterized the model in terms of latent demographic rates
to accommodate the seasonal nature of the data,
we focused our analysis on annual population projections to avoid interpretational issues
with unequal sampling intervals within years and to better match the life history of 
stickleback in M\'{y}vatn.
Estimates of the latent demographic rates are inlcuded in the appendix.
Per capita demographic rates all varied substantially relative to their overall scales
(Figure \ref{fig:cont}a),
with recruitment (i.e. "adults to juveniles") having particularly large fluctuations.
Within-basin demographic rates of a given type (solid lines) 
were modestly synchronized between the basins, 
suggesting lake-wide processes contribute to variation in the demographic rates.
There were also associations between different types 
of within-basin demographic contributions,
with most of them experience local peaks in the mid 1990s and early 2000's. 
These peaks reflect similarly synchronized peaks in the abundance estimates for 
most basin $\times$ stage combinations (Figure \ref{fig:fit}). 
On a per capita basis, 
between-basin contributions (dashed lines) largely reflected the within-basin contributions 
of either the recipient or the donor state, 
which is a consequence of relatively limited temporal variation 
in the latent per capita movement rates
(Figure SX).

Total annual contributions (Figure \ref{fig:cont}b) were dominated by recruitment 
and adult survival in the north basin, particularly in the latter half of the time
series when the abundance of south basin adults was severly depressed. 
Both recruitment and adult survival within the north basin 
declined dramatically around 2010, 
and together they drove the population crash by 2016.
Net movement between the basins (i.e. the difference between dashed lines in a given panel)
were often low, but occasionally quite large.
For example, 
contributions of north basin adults to south basin juveniles 
substantially exceed the contributions of south basin adults to north basin juveniles
for much of the time series.
Indeed, 
in some years the north basin contribution of south basin juveniles even exceed 
recruitment within the south basin, 
indicating a substantial subsidization of the south basin. 

%========================================================================================

\subsection*{Demographic analysis}

The realized population growth rate fluctuated substantially through time, 
and it was generally centered on the replacement rate of one (Figure \ref{fig:pop}a).
However, 
several years of sub-replacement rates following a relatively high growth rate in 2012
manifested as a major decline in the population size towards the end of the time series
(Figure \ref{fig:pop}b).
The deviation of the realized growth rates and popualtion sizes from their
asymptotic analogues were generally very small 
when measured against the overall temporal variability (Figure \ref{fig:pop}a,b).
This indicates that transient flucutations due to non-equilibrium state distributions
contributed little to the population dynamics. 
However, in a given year 
the deviations of the realized population size from the hypothetical population size 
under the asymptotic growth rate 
were occasionally quite substantial (Figure \ref{fig:pop}c). 
Therefore, while transience per se contributed little to the variability of the population,
it was nonethless important for determining 
the aboslute population size at a given point in time.

Sensitivies of the population growth rate to potential perturbations 
was generally low and temporally consistent for juvenile recipients 
(Figure \ref{fig:sens}a).
Furthermore, 
the sensitivies for a given donor state
(e.g. north basin juveiles contributing to either the north or south basin)
to juvenile recipients were strongly synchronized through time. 
This reflects the influence of relative abundance of the donor state on the sensitivity.
However, the contributions to adult recipients were much larger than for juvenile recipients,
regardless of donor state.
Furthermore, even among cases with adult recipients, there was substantial variation 
within donor states depending on the recipient basin. 
These deviations from the expected sensitivity projected over a single time step
are a consequence of how the state distribution is influenced by 
the temporal propogation of perturbations in the demographic rates.
Overall, the largest and most variable sensitivites were for 
the contributions to adults from juveniles and adults within the north basin.
In constrast, 
the sensitivity of contributions to adults within the south basin were
generally lower than either contributions between basins or within the north basin.
Together, these results indicate that the population growth rate was most sensitive
to perturbations within the north basin, followed by pertrubations to exchanges 
between the basins.

Recruitment within the north basin was by far the largest contributor 
to annual changes in the population growth rate, 
followed by recruitment within the south basin (Figure \ref{fig:sens}b).
Between-basin recruitment made contributions of similar magnitude 
to within-basin adult survival, 
while all other demographic rates made relatively minor contributions.
The importance of recruitment in this "retrospective" analysis stands in contrast 
to the results of the "propsective" sensitivity analysis, 
which indicated that the population growth rate was most sensitive 
to contributions to adults.
This is a consequence of the fact that within-basin recruitments were much more variable
than the other demographic rates (Figure \ref{fig:cov}a).
Covariances between most of the demographic rates were weakly positive,
which increased their contributions to the total variance.
In contrast, prevelance of negative covariances would have indicated compensatory dynamics
that would have reduced variation in the population growth rate.
For a given set of basin contributions, 
recruitment made the largest contribution to the variance,
followed by contributions of adults survival.
However, there were large differences among the different sets of basin contributions,
with nearly 40\% of the variance accounted for by processes within the north basin,
followed by roughly equal contributions within the south basin and from north to south,
and with contributions from south to north accounting for relatively little variance
(Figure \ref{fig:cov}b).
Approximately 91\% of the variance in $\uplambda_y$ was accounted for by 
direct contributions from demographic rates, 
while the remainder was accounted for by non-equilibrium state distributions.



% %========================================================================================
% 
% \subsection*{Time-varying demographic rates}
% 
% Mortality rates of juveniles in both basins 
% were fairly stable for the first two decades of the time 
% series, at which point they began to increase steadily (Figure \ref{fig:mort_dev}). 
% Adult mortality was also fairly synchronized between basins,
% with modest fluctuations in the first two decades being followed 
% by increases that coincided with the increased mortality of the juveniles.
% The similarities in mortality patterns across life stages and basins
% suggests that the primary drivers of temporal variation occurred at the whole-lake scale.
% In contrast to mortality, the development of juveniles to adults remained fairly stable
% throughout the time series. 
% Season effects on mortality were generally small,
% with the exception of mortality for large individuals in the north basin (Table \ref{tab:season} ),
% which was modestly lower in winter.
% Development rates were modestly higher in winter than in summer.
% 
% For each life stage, 
% per capita movement rates were negatively correlated between basins
% (Figure \ref{fig:move}),
% which reflects the fact that the data only contained information about
% the net movement between basins. 
% For juveniles, the largest difference between movement rates occurred
% at the end of the time series with movement rates from north to south
% greatly exceeding those from south to north on a per capita basis.
% The opposite pattern held for the adult fish, 
% although the number of south basin adults during the final 
% years of the time series was very low and so overall movement from 
% south to north was also very low.
% Overall, per capita movement rates were higher for juveniles than for adults.
% Movements rates were largely consistent between winter and summer (Table \ref{tab:season}).
% 
% Per capita recruitment varied substantially both through time and between basins.
% In the south basin, recruitment was initially high, then declined 
% to a low point midway through the time series
% before recovering to high levels sustained for the final decade.
% The pattern in the north basin was generally similar, 
% although with relatively less variability in the first part of the time series 
% and the steep increase in recruitment occurring several years later 
% than the corresponding increase in the south.
% Recruitment was lower in "winter" than in summer in both basins (Table \ref{tab:season}),
% which reflects the fact that "winter" spawning was likely attributable
% to the same spawning event as the "summer" spawning
% but happened to fall outside the summer projection period.
% Overall, the magnitude of per capita recruitment was similar between the two basins.
% However, given that the censused area of the south basin was twice that of the north,
% the recruitment per unit area was much higher in the north basin. 
% 
% %========================================================================================
% 
% \subsection*{Annual dynamics}
% 
% When projected over the course of a year, 
% the adult class generally contributed 
% more individuals to the population than did the juvenile class for each basin
% (Figure \ref{fig:cont}).
% Given the similar mortality rates for adults and juveniles,
% this was principally due to adult contributions through reproduction.
% Adult contributions to the adult class included recruitment of 
% juveniles that matured into adults within the same projection year (starting in June/July)
% in which they were born.
% However, adult contributions to the juvenile class were often higher than
% their contributions to the adult class, 
% indicating that a substantial proportion of juveniles born in one projection year 
% remained in the juvenile class by the next year.
% 
% For the most part, 
% contributions across basins were similar, 
% indicating relatively limited net flow of individuals between basins.
% The exception was the large contribution of north basin adults to south basin juveniles
% towards the end of the time series,
% which exceeded the contributions of south basin adults to 
% juveniles in both the north and south basins.
% Therefore, not only was there a net flow from north to south in those years,
% more south basin juveniles were contributed from the north basin than were 
% born in the south basin.
% Statistically, this is a consequence of the substantial number 
% of south basin juveniles at the end of the time series despite the near absence of 
% south basin adults. 
% 
% %========================================================================================
% 
% \subsection*{Contributions to fitness}
% 
% Fitness ($\lambda$) varied substantially in both basins, 
% with large fluctuations early in the time series followed by steady declines
% in the final decade (Figure \ref{fig:lam}). 
% South basin fitness was consistently lower than for the north, 
% which is consistent with the lower population size in the south basin.
% In contrast with fitness, 
% expected lifetime reproductive output of adults ($R_0$) 
% remained fairly steady until the final decade 
% when it increased dramatically in both basins (Figure \ref{fig:r0}). 
% This pattern was broadly similar to the dynamics of per capita recruitment 
% (Figure \ref{fig:rec}).
% The discrepancy between $\lambda$ and $R_0$ was due to changes in the 
% stable age distribution,
% whereby the increasing lifetime reproduction of adults was counterbalanced by
% a shift towards a more juvenile-dominated age structure (especially in the south basin).
% While these patterns are inferred from the demographic model, 
% they are also visually apparent in the observed data:
% the abundance of adults declined while abundance of juveniles increased,
% implying both an increase in adult per capita reproduction and a shift in the 
% stage distribution towards juveniles (Figure \ref{fig:fit}). 
% 
% The potential contributions to changes in fitness 
% ("elasticities"; Figure \ref{fig:elas}) 
% were generally higher for adults than for juveniles in both basins.
% However, in the south basin the relative contributions between the two stages
% varied substantially through time, with the potential contribution of juveniles 
% even slightly exceeding the contribution of adults in 2007.
% In the north basin, the potential contributions of different stages 
% were more consistent through time, 
% due to the lower variability of both $R_0$ and the stable stage distribution. 
% 
% %========================================================================================

\section*{Appendix I: Transition rates} \label{sec:A1}

%========================================================================================

We parameterized transition rate matrices for mortality ($\boldsymbol\Theta^{\mu}_{\tau}$), 
development ($\boldsymbol\Theta^{\gamma}$), and dispersal ($\boldsymbol\Theta^{\delta}_{\tau}$) as:
 %
\begin{equation} \label{eq:Theta}
\begin{aligned}
\boldsymbol\Theta^{\mu}_{\tau} & = 
\left[
\begin{array}{cc|cc}
    -\theta^{\phi_{j,s}}_{\tau} & 0 & 0 & 0 \\
    0 & -\theta^{\phi_{a,s}}_{\tau} & 0 & 0 \\
    \hline
    0 & 0 & -\theta^{\phi_{j,n}}_{\tau} & 0 \\
    0 & 0 & 0 & -\theta^{\phi_{a,n}}_{\tau} \\
    \end{array}
\right] \\
\boldsymbol\Theta^{\gamma} & = 
\left[
\begin{array}{cc|cc}
    -\theta^{\gamma_{j}} & 0 & 0 & 0 \\
    \theta^{\gamma_{j}}  & 0 & 0 & 0 \\
    \hline
    0 & 0 & -\theta^{\gamma_{j}} & 0 \\
    0 & 0 & \theta^{\gamma_{j}}  & 0 \\
    \end{array}
\right] \\
\boldsymbol\Theta^{\delta}_{\tau} & = 
\left[
\begin{array}{cc|cc}
    0 & 0 & 0 & 0 \\
    0 & -\theta^{\delta_{a,s}}_{\tau} & 0 & \theta^{\delta_{a,n}}_{\tau} \\
    \hline
    0 & 0 & 0 & 0 \\
    0 & \theta^{\delta_{a,s}}_{\tau} & 0 & -\theta^{\delta_{a,n}}_{\tau} \\
    \end{array}
\right]
\end{aligned}
\end{equation}
%
Note that mortality implicitly entails transition to a ``death state'' that is omitted 
for concision, as dead individuals do not contribute further to the dynamics.
For each transition matrix $\boldsymbol\Theta^{\alpha}_{\tau}$, 
we then calculated the probability of transitioning as
%
\begin{equation} \label{eq:Psi}
\boldsymbol\Psi^{\alpha}_{\tau} = e^{\Delta T\boldsymbol\Theta^{\alpha}_{\tau}}
\end{equation}
%
which is the solution to the differential equation associated with the Markov process
specified by $\boldsymbol\Theta^{\alpha}_{\tau}$ 
when projected over interval $\Delta T$.
The elements of $\boldsymbol\Psi^{\alpha}_{\tau}$ were then used to parameterize
the matrices given by equations \ref{eq:W} and \ref{eq:B}.
In principle, we could have included all of the demographic transitions in a single
transition matrix. 
However, modeling the different transition processes separately facilitated interpretation
of the resulting transition probabilities, as they would only pertain to a single type 
of demographic transition rather than multiple occurring simultaneously.
This was also computationally advantageous during model fitting, 
for much the same reasons.
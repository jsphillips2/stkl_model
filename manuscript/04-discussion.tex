

\section*{Discussion}

1. Population crash driven by declines in both recruitment and adult survival 
in the north basin (both per capita and total contributions).
\linebreak 
\linebreak 
2. Demographic rates are fairly synchronized between basins, 
suggesting a role of lake-wide processes
\linebreak
\linebreak
3. Variation in population growth rate is mostly due to direct effect of changes 
in demographic rates, 
but transience per se is relevant for absolute abundance in a given year
\linebreak
\linebreak
4. Population growth rate is most sensitive to perturbations in juvenile development
and adult survival within the north basin,
while recruitment within the north basin is the dominant contributor to variance.
\linebreak
\linebreak
5. Movement between basins is important, 
both in that it can be large relative to internal contributions 
and it contributes to the variance in population growth rate 
(~35\% when combining north -> south and south -> north).
\linebreak
\linebreak
6. Both "prospective" and "retrospective" analyses are relevant. 
The latter is more directly explanatory, 
but the former provides important context.
Furthermore, 
the prospective sensitivity analysis is likely more informative for the evolutionary 
dimensions, 
since selection gradients (and the like) depend 
on the sensitivity of the population growth rate
to changes in traits that influence the demographic rates.


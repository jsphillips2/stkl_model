

\section*{Discussion}

We used a stage-structured metapopulation model 
with time-varying demographic rates to characterize the 
spatiotemporal dynamics of threespine stickleback 
in the shallow and productive lake M\'{y}vatn.
From the model, we were able to draw the following inferences regarding the 
dynamics of the population:
%
\begin{enumerate}[label=(\alph*)]
\item
The stickleback population had cyclic dynamics with a period of approximately 6 years.
Furthermore, the population growth rate exhibited negative density dependence,
which is often associated with population cycles.
However, transient fluctuations due to non-equilibrium state distributions
partially obscured the periodicity,
and there may have been a weakening of the periodicity in the final decade.
%
\item
The population was characterized by source-sink dynamics, 
with the north basin subsidizing the south basin and driving the overall dynamics. 
Demographically unfavorable conditions in the south basin were 
primarily due to low juvenile survival.
%
\item
In addition to the spatial coupling between basins,
several within-basin demographic rates were strongly correlated 
between the two basins and 
were associated with temporal variation in the asymptotic population growth rate.
This indicates that processes at the whole-lake scale contributed 
to the population dynamics.
\end{enumerate}
%
Together, these inferences illustrate how synchronization of cyclic fluctuations 
can arise through both spatial coupling and synchronization of demographic rates.
Furthermore, this study illustrates the value of spatially explicit demographic models 
with time-varying parameters for disentangling the contribution of different processes,
including transience per se, to spatiotemporal population dynamics.

Before discussing the further implications of our analysis, 
it is worth noting some of its limitations. 
Demographic analyses are often conducted using mark-recapture type methods,
allowing relatively direct inferences of demographic rates
\citep{lebreton1992, fujiwara2002}.
In the absence of such data, 
we instead parameterized our model by fitting it to a time-series of abundance estimates.
Uniquely identifying demographic processes 
contributing to observed population change is a long-standing challenge 
\citep{wood1994, twombly1994},
and required some simplifying assumptions.
Specifically, 
we assumed that juvenile development rate was fixed through time and between basins,
that only adults dispersed within a given time step,
and that recruits born within a given basin remained within that basin until the next time step.
While acknowledging the uncertainties,
we feel that these constraints on the model are reasonable for 
threespine stickleback in M\'{y}vatn,
as in other systems \citep{yurtseva2019}.
The inferred development rates were sufficiently high that most juveniles would transition 
to adulthood in a given time step,
which is consistent with other data suggesting that M\'{y}vatn stickleback
reach adult size within one year (unpublished data).
This high transition rate of juveniles
limits the relevance of both temporal variation in development and juvenile dispersal,
as most individuals would be come adults and be able to disperse within that time step 
regardless.
While we cannot rule out cross-basin recruitment, 
it seems that spawning does in fact occur within both basins given the ubiquity 
of breeding-condition adults, 
as opposed to all spawning occurring within one basin.
Furthermore, any cross-basin recruitment within a given time step 
is implicitly characterized by the model as additional temporal variation in
per capita recruitment.
Finally, both per capita and total recruitment had low covariance between basins relative
to the overall variance, which is not what would be expected in the presence of 
the homogenizing effect of cross-basin exchange. 

M\'{y}vatn's stickleback population fluctuated substantially over the past three decades,
with the annual population growth rates spanning an order of magnitude and including
periods of both substantial growth and decline.
Furthermore, these fluctuations were substantially synchronized between basins,
due to both spatial coupling via dispersal
and correlated survival between basins. 
Therefore, these fluctuations can be understood as a feature of the lake-wide dynamics
and were likely driven by processes occurring at the whole-lake scale.
The population dynamics were cyclic over most of the time series,
with a period of approximately 6 years. 
This periodicity is not obviously tied to the life history of threespine stickleback,
which likely live on the order of 3 or 4 years in M\'{y}vatn.
Previously,
\cite{wootton2005} compared the dynamics of 3 threespine stickleback populations 
on the island of Great Britain (one riverine, one lacustrine, and one backwater)
and found that the backwater population had cyclic dynamics of approximately
6 years, while the other two lacked any consistent periodicity.
Therefore, the 6-year periodicity of M\'{y}vatn's population is not without precedent,
although it is also not a ubiquitous feature of stickleback populations.
Cyclic dynamics are often associated with negative density-dependence, 
either direct when there are appropriate time lags \citep{may1974}
or indirect when mediated through interspecific interactions 
\citep{nicholson1935, rosenzweig1963}.
Indeed, the asymptotic growth rate of the M\'{y}vatn population declined 
with total population density,
with a 25\% increase above the implied equilibrium corresponding to a 13\% 
reduction in the  growth rate.
Despite its consistency for the first two decades, 
the periodicity of the population growth rate weakened in the final decade, 
with perhaps a hint of reestablishment in the last few years.
This could suggest a change in the underlying dynamics of the system
\citep{carpenter2011},
which has been observed for other populations with cyclic dynamics
\citep{ims2008collapsing}.
However, caution is warranted when drawing such inferences 
from relatively short time series.

Transience was an important feature of the population dynamics,
with the periodicity that was clear in the asymptotic growth rate manifesting 
at most very subtly in the transient growth rate.
Furthermore, transience per se reduced the geometric by 5\%,
although this effect was minor relative to the effect of asymptotic fluctuations
(on the order of 20\%). 
The most conspicuous transient event was the dramatic increase in recruitment of south basin juveniles in 1992, followed by a similarly dramatic decrease.
This event was observed across multiple traps and stations within the south basin,
suggesting that it was a real feature of the dynamics, rather than an error in the data.
While this recruitment event contributed little to the future dynamics, and was
arguably unimportant from the perspective of the stickleback population,
it may well have had a substantial influence on other components of the ecosystem.
Indeed, predation by threespine sticklebacks alters the 
abumndance and community composition of 
zooplankton and phytoplankton in M\'{y}vatn \citep{ersoy2017}
and other systems \citep{harmon2009}.
Ecologists are increasing aware of the importance of ``extreme'' events and the 
processes that bring them about \citep{anderson2017, batt2017}.
Juvenile recruitment in fish as previously been identified as highly episodic,
due to nonlinear effects of multiple environmental drivers \citep{dixon1999}.
In the case of M\'{y}vatn's stickleback, 
per capita recruitment in 1992 was high 
but not nearly so extreme relative to other years as was case for total recruitment.
Therefore, the extreme total recruitment was not simply due to a year of particularly
high recruitment per adult, 
but also due to the disproportionate abundance of south basin adults 
relative the equilibrium state distribution.
This underscores the potential role of demographic transience 
in facilitating extreme population events.

The dynamics of the north and south basin populations were synchronized,
particularly for the adults. 
According to the model, 
this was due to both spatial coupling via dispersal 
and high covariance in certain demographic rates such as adult survival.
Beyond the synchronization, 
the two sub-populations displayed source-sink dynamics \citep{pulliam1988},
with the north basin substantially subsidizing the south.
Furthermore,
the lake-wide population growth rate was most sensitive to demographic processes 
within the north basin,
indicating that the north basin played a primary role in dictating the dynamics 
of the entire population.
However, this does not mean that the southern sub-population failed to contribute 
meaningfully to the dynamics.
On the contrary,
in some years the abundance of south basin juveniles and adults was quite high relative
to the north basin counterparts 
and therefore contributed meaningfully to the total abundance in those years.
Furthermore, it is possible that the absence of net subsidies from the north basin
could be compensated for by elevated demographic rates via reduced intraspecific competition.
Nonetheless, the role of the north basin in the overall dynamics was still dominant.
The relatively limited demographic contribution from the south basin was primarily due
to the very low survival of its juveniles. 
This low juvenile survival is directly implied by the population estimates derived 
from the trapping data,
as south basin juveniles are generally rare with the exception of sporadic peaks that
do not appear to contribute the the abundance of any demographic states in future time steps.
The ecological bases for low survival of south basin juveniles,
as well as temporal variation in all of the demographic rates,
are largely unknown.
However, other populations in M\'{y}vatn have dramatic fluctuations that appear to be 
driven by processes endogenous to the lake
\citep{einarsson2002, ives2008, einarsson2016}. 
Therefore, it is plausible that the same is true for the threespine stickleback.

Our analysis illustrates the potential for both dispersal-mediated spatial coupling 
and synchronized demographic rates to generate synchronized fluctuations in a wild population.
Furthermore, we show that transience can alter the character of such dynamics,
partially obscuring the periodicity that is underpinned by the synchronized temporal
variation in the demographic rates.
Central to this effort was explicitly modeling the demographic structure of the population
and temporal variation in the demographic rates 
that manifested sychronized processes occuring within both sup-populations.
While this study focused on a single population,
the dynamic processes and the methods used to analyze them are likely to relevant for 
other spatially-structured population with cyclic fluctuations.







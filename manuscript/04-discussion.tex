

\section*{Discussion}

In this study, we used a structured population model 
with time-varying demographic rates to characterize the 
spatiotemporal dynamics of sticklebacks in the shallow and productive lake M\'{y}vatn.
From the model, we were able to draw the following inferences regarding the 
dynamics of the population:
%
\begin{enumerate}[label=(\alph*)]
\item
The stickleback population had cyclic dynamics with a period of approximately 6 years.
Furthermore, the population growth rate exhibited negative density dependence,
which is often associated with cyclic dynamics.
However, transient fluctuations due to non-equilibrium state distributions
partially obscured the population cycles,
and there may have been a weakening of the periodicity in the final decade.
%
\item
The population was characterized by source-sink dynamics, 
with the north basin subsidizing the south basin and driving the overall dynamics. 
Demographically unfavorable conditions in the south basin were 
primarily due to low juvenile survival.
%
\item
In addition to the spatial coupling between basins,
several within-basin demographic rates were strongly correlated 
between the two basins and 
were associated with temporal variatin in the asymptotic population growth rate.
This indicates that processes at the whole-lake scale contributed 
to the population dynamics.
\end{enumerate}
%
Together, these inferences illustrate how synchronization of cyclic fluctuations 
can arise through both spatial coupling and synchronization of demographic rates.
Furthermore, this study illustrates the value of spatially explicit demographic models 
with time-varying parameters for disentagling the contribution of different processes,
including transience per se, to spatiotemporal population dynamics.

Before discussing the further implications of our analysis, 
it is worth noting some of its limitations. 







\begin{enumerate}[label=(\alph*)]
\item
Assuming only adults disperse 
(find relevant literature on sticklebacks; address implications of cross-basin recuritment)
%
\item
Inferring demgraphy from dynamics; 
address low survival of south basin juveniles (visible from data) 
and dispersal (also visible fromo data; genetic data indicate substantial mixing)
\end{enumerate}

Remaining discussion points:
\begin{enumerate}[label=(\alph*)]
\item
Population cycles: 
  \begin{enumerate}
  \item
  Fluctuations reduced the long-term growth rate,
  as generally expected for population dynamics.
  \item 
  Population growth rate shows clear periodicity that is not obviouosly constrained by
  life cycle (similar to another stickleback population
  \item
  Negative density depenednece is consistent with periodicity and suggests interactions 
  with other populations (either resources or predators).
  Cycles generally required the tight coupling that arises from specialized associations.
  Such tight couply could exist for certain cladoceran taxa such as Eurycercus or 
  tapeworms. 
  \end{enumerate}
  Covariation among demographic rates in both basins 
  suggests processes operating at lake scale.
\item
Transience: 
  \begin{enumerate}
  \item 
  Partially obscures population fluctuations
  \item
  Contributes to "extreme" event, with regard to large spike in juvenile recruitment
  in the south basin.
  \end{enumerate}
\item
Spatial population structure
  \begin{enumerate}
  \item 
  Basins are dynamically coupled through dispersal.
  \item
  Demographic rates are similar within basins, suggesting lake-wide processes.
  \item
  There is a single population with substantial mixing (supported by genetics) that 
  nonetheless displays some spatial heterogeneity.
  \item
  Moran effect vs. dispersal: both play a role.
  \end{enumerate}
\end{enumerate}


\section*{Discussion}

In this study, we used a structured population model 
with time-varying demographic rates to characterize the 
spatiotemporal dynamics of sticklebacks in the shallow and productive lake M\'{y}vatn.
From the model, we were able to draw the following inferences regarding the 
dynamics of the stickleback population:
%
\begin{enumerate}[label=(\alph*)]
\item
The stickleback population had cyclic dynamics with a period of 6-years,
and these cycles were associated with negative density dependence.
However, transient fluctuations due to non-equilibrium state distributions
partially obscured the population cycles,
and there may have been a shift in the dynamics over the final decade.
%
\item
The population was characterized by source-sink dynamics, 
with the north basin subsidizing the south basin and driving the overall dynamics. 
Demographically unfavorable conditions in the south basin were 
primarily due to low juvenile survival.
%
\item
In addition to the spatial coupling between basins,
several within-basin demographic rates were strongly correlated 
between the two basins and 
were associated with temporal variatin in the asymptotic population growth rate.
This indicates that processes at the whole-lake scale contributed 
to the cyclic population dynamics.
\end{enumerate}
%
Together, these inferences illustrate how synchronization of cyclic fluctuations 
can arise through both spatial coupling and synchronization of demographic rates.
Furthermore, this study illustrates the value of spatially explicit demographic models 
with time-varying parameters for disentagling the contribution of different processes,
including transience per se, to spatiotemporal population dynamics.



% ---------------------------------------------------------------------------------------
% ---------------------------------------------------------------------------------------
% Preliminaries
% ---------------------------------------------------------------------------------------
% ---------------------------------------------------------------------------------------

\documentclass[11pt]{article}
\usepackage[letterpaper, margin=1in]{geometry}
\usepackage{newtxtext,newtxmath}
\usepackage[math-style=ISO]{unicode-math}
\usepackage{fullpage}
\usepackage[authoryear,sectionbib]{natbib}
\linespread{1.7}
\usepackage[utf8]{inputenc}
\usepackage{lineno}
\usepackage{titlesec}
\titleformat{\section}[block]{\Large\bfseries\filcenter}{\thesection}{1em}{}
\titleformat{\subsection}[block]{\Large\itshape\filcenter}{\thesubsection}{1em}{}
\titleformat{\subsubsection}[block]{\large\itshape}{\thesubsubsection}{1em}{}
\titleformat{\paragraph}[runin]{\itshape}{\theparagraph}{1em}{}[. ]\renewcommand{\refname}
  {Literature Cited}
\DeclareTextSymbolDefault{\dh}{T1} % for Icelandic ð symbol:
\usepackage{graphicx} % for figures
\usepackage{booktabs} % for tables
\usepackage{amsmath} % for split math environment
\usepackage{enumitem}




% ---------------------------------------------------------------------------------------
% ---------------------------------------------------------------------------------------
% Title page
% ---------------------------------------------------------------------------------------
% ---------------------------------------------------------------------------------------


\title{Transient dynamics of spatially structured fluctuations 
in a threespine stickleback population}

\author{
Joseph S. Phillips$^{1,2, \dagger}$ \\
\'{A}rni Einarsson$^{3}$ \\ 
Kasha Strickland$^{1}$ \\
Anthony R. Ives$^{2}$ \\
Bjarni K. Kristj\'{a}nsson$^{1}$ \\
Katja R\"{a}s\"{a}nen$^{4}$ 
}

\date{}

\begin{document}

\raggedright
\setlength\parindent{0.25in}

\maketitle


\noindent{} 1. Department of Aquaculture and Fish Biology, 
H\'{o}lar University, Skagafj\"{o}r{\dh}ur 551 Iceland

\noindent{} 2. Department of Integrative Biology, 
University of Wisconsin, Madison, Wisconsin 53706 USA

\noindent{} 3. M\'{y}vatn Research Station, IS-660 M\'{y}vatn, Iceland

\noindent{} 4. Department of Aquatic Ecology, EAWAG and 
Institute of Integrative Biology, ETH Zurich, 
\"{U}berlandstrasse 133, CH-8600 D\"{u}bendorf, Switzerland

\noindent{} $\dagger$ E-mail: joseph@holar.is



\bigskip

Running head: {Spatially-structured population cycles}

\linenumbers{}

\clearpage





% ---------------------------------------------------------------------------------------
% ---------------------------------------------------------------------------------------
% Abstract
% ---------------------------------------------------------------------------------------
% ---------------------------------------------------------------------------------------

\section*{Abstract} \label{abstract}

Uncovering the demographic basis of population fluctuations is challenging 
for spatially structured populations, 
because this requires disentangling synchrony in demographic rates 
from coupling via dispersal. 
In this study, we fit a stage-structured metapopulation model to a 29-year times series 
of population density estimates of threespine stickleback in the heterogeneous 
and productive Lake M\'{y}vatn, Iceland. 
The lake comprises two basins (North and South) connected by a channel, 
through which the stickleback disperse. 
The model includes time-varying demographic rates, 
allowing us to disentangle the contributions of recruitment and survival, 
spatial coupling via dispersal, and transient dynamics 
to the population’s large fluctuations in abundance. 
Our analyses indicate that recruitment was only modestly synchronized between the two basins, 
whereas survival probability of adults was highly synchronized, 
contributing to cyclic fluctuations in the lake-wide population 
with a period of approximately 6 years. 
The analyses further show that the two basins are strongly coupled through dispersal, 
with the north basin subsidizing the south basin and playing a dominant role 
in driving the lake-wide dynamics. 
Moreover, transience due to non-equilibrium distributions of individuals 
across demographic states results in large short-term changes in abundance 
that partially obscure the cyclic nature of the dynamics. 
Our results show that cyclic fluctuations can arise through both spatial coupling 
and synchronization of demographic rates, and they highlight the importance 
of accounting for transience in analyses of population dynamics.


\bigskip

\textit{Keywords}: {demography, \textit{Gasterosteus aculeatus}, metapopulation,
                    M\'{y}vatn, population cycles, state-space model}

\clearpage



% ---------------------------------------------------------------------------------------
% ---------------------------------------------------------------------------------------
% Introduction
% ---------------------------------------------------------------------------------------
% ---------------------------------------------------------------------------------------

\section*{Introduction} \label{introduction}

Temporal fluctuations in demographic processes such as survival and reproduction
are of central importance to population biology.
Changes in demographic rates underpin population fluctuations arising from both 
endogenous processes such as predator-prey interactions and endogenous processes
such as climatic variability.
Furthermore, demographic rates can change in response to trait plasticity and evolution,
and feedbacks between ecological and evolutionary dynamics in wild populations 
are predicated on the potential for demographic rates to vary through time.
While temporal variation in demographic rates has been extensively studied,
these studies are generally restricted to populations subject to long-term monitoring
with repeated observations of uniquely identified individuals (e.g., mark-recapture).
Given the expense and logistical challenges of such studies, 
they are unlikely to be representative of the range of dynamics found in wild populations.
Furthermore, long-term demographic studies tend to focus on directional trends, 
and therefore do not directly address population fluctuations 
(but see...).
Consequently, there is a need for further studies characterizing the demographic underpinnings 
of population fluctuations, especially in systems that are generally
underrepresented in long-term demographic studies.

Uncovering the demographic basis of population fluctuations can be particularly challenging
for metapopulations, 
whereby discrete patches or sub-populations are linked through dispersal
\citep{hanski1998}.
Phenomenologically, the metapopulation dynamics are the aggregate of the patch-level dynamics.
For example, synchronous fluctuations in patch-level abundance will reinforce each other in the metapopulation dynamics while anti-synchronous or compensatory 
fluctuations will cancel each other out. 
The degree of synchrony between the patch-level dynamics will be influenced
by the extent to which the within-patch demographic rates 
(e.g., reproduction and survival) are synchronized between patches.
Synchronization could arise from synchronous changes in environmental
drivers such as climate (i.e., Moran effect) 
or through patches being indirectly coupled through joint interactions with another 
dynamic agent such as a mobile predator. 
Moreover, direct coupling between patches via dispersal can alter the patch-level dynamics.
Dispersal can  result in both greater synchrony and greater asynchrony between patches, 
depending on the level of dispersal and the demographic processes occurring within patches,
including the extent to which demographic rates are synchronized across patches. 
Disentangling the effects of direct coupling through dispersal, 
the degree of synchrony between demographic processes within patches, 
and the interplay between the two 
is an important step in characterizing metapopulation fluctuations.

Further complicating efforts to characterize the demographic basis 
of population fluctuations is the potential role of transience.
``Transience'' refers to short-term dynamics of a system 
that differ from the long-term or asymptotic dynamics under a fixed set of conditions
\citep{hastings2010}.
In the case of demographically-structured populations,
a fixed set of demographic rates implies an equilibrium distribution
of individuals across population states (e.g., stage-classes or patches).
The population's state distribution will tend towards its equilibrium,
and this will generally entail transient changes 
in the population's growth rate \citep{caswell2001matrix}.
The qualitative behavior of transience depends
on the exact configuration of demographic rates
but can entail cyclic fluctuations 
as the population approaches its equilibrium distribution.
By definition, these transient fluctuations occur in the absence of changes in the 
underlying demographic rates.
When demographic rates vary through time,
which is true to at least some extent for all wild populations,
the equilibrium distribution will be a ``moving target''
such that population is likely to be in a perpetual state of disequilibrium.
Whether such perpetual disequilibrium results in large transience 
depends on the extent to which the underlying demographic rates change through time
and the rate at which the population approaches its ever-changing equilibrium distribution.
Despite the growing recognition of its potential ubiquity
\citep{caswell2007sensitivity, koons2017understanding},
the contribution of transience to fluctuations in wild populations is generally unknown.

In this study, 
we analyzed the metapopulation fluctuations of threespine stickleback 
(\textit{Gasterosteus aculeatus})
over the course of three decades in Lake M\'{y}vatn, Iceland.
M\'{y}vatn's stickleback population is spatially structured by the unique geomorphology
of the lake \citep{gislason1998, millet2013}, 
which is divided into two basins connected by two narrow channels.
The South Basin (Sy{\dh}rifl\'{o}i) is the larger of the two ($28~\text{km}^2$) 
and is dominated by exposed sediment and intermittent mats of filamentous green algae
\citep{einarsson2004myvatn}.
In contrast, the North Basin (Ytrifl\'{o}i) is substantially smaller ($9~\text{km}^2$)
and more spatially heterogeneous, 
in part due to dredging of the lake bottom
that substantially altered its bathymetry.
The North Basin has historically sustained much higher
densities of threespine stickleback  \citep{gislason1998}, 
likely due to the ecological differences between the basins. 
Despite the narrow connection between the basins, 
population genetic \citep{millet2013} 
and genomic analyses (unpublished data) indicate limited differentiation,
which in turn implies extensive gene flow and admixture.
This is consistent with previous studies indicating that lacustrine populations
of threespine sticklebacks are generally well-mixed through extensive within-lake dispersal.

Moreover, M\'{y}vatn's stickleback population fluctuates substantially through time.
While the causes of these fluctuations are unknown, 
they are likely connected to the large temporal variability 
of other populations in the lake.
M\'{y}vatn is naturally eutrophic due to the inflows of nutrient-rich springs,
which sets the stage for high-amplitude fluctuations in secondary producers
\citep{einarsson2004myvatn}.
Chief among these are chironomid midges and cladocerans
\citep{einarsson2002, einarsson2004clad, gardarsson2004population, ives2008},
both of which are important food sources for threespine stickleback.
Furthermore, the lake hosts temporally-variable populations 
of Arctic charr (\textit{Salvelinus alpinus}), 
brown trout (\textit{Salmo trutta}), 
and piscivorous birds that have the potential 
to induce fluctuations in the stickleback population from the top down
\citep{einarsson2004moulting, gudbergsson2004}.
Finally, M\'{y}vatn's stickleback can sustain high loads of the cestode parasite
\textit{Schistocephalus solidus} \citep{gislason1998, karvonen2013},
which could lead to host-parasite fluctuations.

To explore the spatiotemporal dynamics of M\'{y}vatn's stickleback population, 
we fit a stage-structured metapopulation model \citep{caswell2001matrix}
to a 29-year times series of population density estimates.
The model included temporal variation in demographic rates 
such as recruitment, survival, and dispersal,
which were inferred by fitting the model to the observed data.
This approach provides great flexibility in modeling changes in the demographic rates,
including those implicitly arising from nonlinear and density-dependent processes
\citep{zeng1998, ives2012}.
Equipped with the parameterized model, 
we quantified the synchrony in recruitment and survival between the two basins
in addition to the degree of coupling through dispersal.
Furthermore, we quantified both the direct and transient effects of 
time-varying recruitment, survival, and dispersal
on the metapopulation dynamics.
Our analysis provides an example case for uncovering the demographic basis 
of spatially structured fluctuations in a wild population, 
which may provide general insights for other systems. 




% ---------------------------------------------------------------------------------------
% ---------------------------------------------------------------------------------------
% Methods
% ---------------------------------------------------------------------------------------
% ---------------------------------------------------------------------------------------

% ---------------------------------------------------------------------------------------
\section*{Methods} 
% ---------------------------------------------------------------------------------------



% ---------------------------------------------------------------------------------------
\subsection*{Data and population estimates} 
% ---------------------------------------------------------------------------------------

From 1991 to 2020, 
the stickleback population of M\'{y}vatn was surveyed in June and August of each year. 
The surveys were conducted at 8 off-shore sites, 
5 in the South Basin and 3 in the North Basin
(Figure \ref{fig:data}a).
These sites provided wide coverage of the lake,
with exception of the eastern portion of the South Basin 
which historically has had negligible densities of sticklebacks.
For each site and survey event, 
we set five un-baited minnow traps for two 12-hour sessions, 
one during the day and one during the night. 
Trapped individuals were sorted into two size classes (small and large)
with a threshold of 50mm in June and 45mm in July, based on \citep{gislason1998}. 
Although there is likely variation in size-at-maturation \citep{singkam2019},
these size classes are expected to correspond with sexual maturity \citep{gislason1998},
and for the purposes of demographic modeling (described below) 
they interpreted as two stage-classes (juvenile and adult).
In general, threespine stickleback reach maturity at 1-2 years of age.
Within each basin and stage class, 
site catches were of comparable magnitude and 
generally synchronized through time (Figure \ref{fig:data}b).
Therefore, our analysis focused on the dynamics of basin-level abundances
of the two size classes through time;
this also had the benefit of making the metapopulation model (described below) more tractable
than would be the case if formulated at the site level.
Note that we grouped site 135 with the South Basin, 
as its stickleback catch was similar to other sites within the South Basin in most years.
However, 
we acknowledge that an unusually large number of adults were trapped at site 135
in 2019 and 2020, which may not be fully captured in our basin-level analysis.

To account for systematic variation in trapping rate between stage classes,
sampling sites, and time of day, 
we fit a modified N-mixture model \citep{royle2004}
to the trap-level data to estimate mean population density across sampling sits 
for each basin $\times$ size-class combination at each time point.
(Appendix I). 
Variation in trapping rate was modeled with fixed effects for 
sampling site,
time of trapping (day vs. night),
stage class,
and the time of trapping $\times$ stage class interaction.
The trapping rates for basin $\times$ stage class combinations
(averaging across sites and trapping times on the logit-link scale) 
were of comparable magnitudes:
0.15 for South Basin juveniles,
0.21 for North Basin juveniles,
0.11 for South Basin adults, 
and 0.17 North Basin adults.
After fitting th N-mixture model,
Ww scaled the relative density estimates by the relative sampling areas of the two basins, 
which we defined as the entire North Basin and the region of the South Basin 
covered included with sampling region 
(excluded the area E of the chain of islands and south of site 135).
This resulted in a 2:1 relative scaling for the South vs. North Basin.
Both the relative density and relative abundance (after scaling by area)
were generally higher in the North Basin than in the South,
consistent with a mark-recapture study conducted in 1989 \citep{gislason1998}.





% ---------------------------------------------------------------------------------------
\subsection*{Metapopulation model} 
% ---------------------------------------------------------------------------------------

We used a stage-structured metapopulation model \citep{caswell2001matrix}
with time-varying demographic rates
to characterize the dynamics of the stickleback population. 
The model projected the population dynamics due to 
recruitment, survival, and stage-transitions within each basin,
as well as dispersal between basins (see below for details). 
Recruitment, survival, and dispersal were allowed to vary through time,
enabling the model to characterize a range of dynamics, 
including those implicitly due to endogenous (e.g., density dependence) 
and exogenous (e.g, environmental variation) processes
\citep{zeng1998, ives2012}. 
We estimated the demographic rates by  
fitting the model to the time series of abundance estimates,
as described in detail below. 
In general terms, this approach works by reconstructing the demographic rates required
to project the distribution of abundances across demographic states 
from one time step to the next. 
While parameter identifiability poses a challenge for such inferences over a single time
step, by explicitly modeling temporal variation in the demographic rates we were able 
to take advantage of shared information across all time points simultaneously to 
successfully constrain the parameter estimates.
The resulting demographic rates are best understood as parameters 
characterizing the observed population dynamics, 
rather than as individual-level demographic rates that would be estimated through a 
technique such as mark-recapture or direct observation. 

For a given time interval from $t-1$ to $t$, 
the population dynamics were projected as
%
\begin{equation} \label{eq:XPX}
    \mathbf{x}_t = \mathbf{P}_{t-1}~\mathbf{x}_{t-1}
\end{equation}
%
where $\mathbf{P}_{t}$ is 4 $\times$ 4 a matrix of demographic rates at time $t$, 
and $\mathbf{x}_{t}$ is a 4 $\times$ 1 vector of abundances 
for a given stage (juveniles $j$; adults $a$) 
and basin (south $s$; north $n$):
%
\begin{equation} \label{eq:X}
\mathbf{x}_{t} = 
\left[
\begin{array}{cccc}
    {x_{j,s,t}} \\
    {x_{a,s,t}} \\
    {x_{j,n,t}} \\
    {x_{a,n,t}}
    \end{array}
\right]
\text{.}
\end{equation}
%
The projection matrix $\mathbf{P}_{t}$ can be expressed as
%
\begin{equation} \label{eq:P}
\mathbf{P}_{t} = 
\left[
\begin{array}{c|ccc}
    \mathbf{W}_{s,t}  & \mathbf{B}_{s\rightarrow n,t} \\
    \hline
    \mathbf{B}_{n\rightarrow s,t} & \mathbf{W}_{n,t}
    \end{array}
\right]
\end{equation}
%
where $\mathbf{W}_{i,t}$ is a 2 $\times$ 2 matrix characterizing 
per capita contributions within basin $i$,
and $\mathbf{B}_{i\rightarrow k,t}$ is a 2 $\times$ 2 matrix characterizing 
contributions from basin $i$ to basin $k$.
Within-basin contributions were modeled as 
 %
\begin{equation} \label{eq:W}
\mathbf{W}_{i,t} = 
\left[
\begin{array}{cccc}
    \phi_{j,i,t}~(1-\gamma_{j})~(1-\delta_{j,i,t}) & 
    \rho_{i,t} \\
    \phi_{j,i,t}~\gamma_{j}~(1-\delta_{a,i,t}) & 
    \phi_{a,i,t}~(1-\delta_{a,i,t})
    \end{array}
\right]
\end{equation}
%
where $\phi_{h,i,t}$ is the survival probability of stage-class $h$, 
$\gamma_{j}$ is the proportion of surviving juveniles that transition into adults
(i.e., the ``stage-transtion'' probability),
$\delta_{h,i,t}$ is the proportion of surviving individuals that disperse to the other basin,
and $\rho_{i,t}$ is per capita recruitment.
We modeled between-basin contributions as
%
\begin{equation} \label{eq:B}
\mathbf{B}_{i,t} = 
\left[
\begin{array}{cccc}
    \phi_{j,i,t}~(1 - \gamma_{j})~\delta_{j,i,t} & 
    0 \\
    
    \phi_{j,i,t}~\gamma_{j}~\delta_{a,i,t} & 
    \phi_{a,i,t}~\delta_{a,i,t}
    \end{array}
\right].
\end{equation}
%
We fixed $\gamma_{j}$ to a single value for both basins and through time 
because allowing it to vary presented convergence difficulties in the model fitting (see below).

To be biologically interpretable, 
survival, dispersal, and stage-transition probabilities must constrained to between
0 and 1.
Furthermore, 
probabilities of all possible fates for an individual beginning in a given state 
(i.e., basin $\times$ stage combination) must sum to 1.
To accommodate these constraints,
we modeled survival, dispersal, and stage-transitions 
in terms of latent transition rate matrices,
from which we calculated the transition probabilities 
(i.e., $\phi_{h,i,t}$, $\gamma_{j}$, and $\delta_{h,i,t}$) 
projected over the interval between time steps (see Appendix II).
This approach imposes an inverse relationship between
the basin-specific dispersal probabilities for a give stage class,
which is appropriate because only the net dispersal between basins manifests 
in the population dynamics. 
Our forumlation first calculates survival, then stage-transitions, and finally dispersal
(equation \ref{eq:POmega}),
which is reflected in the structure of equations \ref{eq:W} and \ref{eq:B}. 

We modeled temporal variation in recruitment and latent transition rates as
%
\begin{equation} \label{eq:theta}
    \omega^{\alpha}_t & \sim \text{Gaussian}
        \left(
            \omega^{\alpha}_{t-1},~\sigma_{\alpha}}
        \right) \text{T}(0, \infty)
\end{equation}
%
where $\alpha$ denotes identity of the rate (e.g., South Basin recruitment), 
and $\sigma_{\alpha}$ is the standard deviation.
This approach results in smooth changes in the demographic rates through time
by applying a Gaussian penalty to the model likelihood (see below) for deviations
from the value of a given demographic rate at the previous time step,
with the scale of this penalty being determined by the standard deviation $\sigma_{\alpha}$.
We used a single standard deviation for recruitment in both basins,
and a single standard deviation for survival and dispersal of all state combinations,
as these demographic processes respectively occurred on similar scales.
The Gaussian penalty was truncated from the left at zero to ensure the 
demographic rates remained positive.

We fit the model in a Bayesian framework, 
using the abundance estimates from the trapping data. 
The likelihood of the ``observed'' abundance given the projected abundances 
and standard deviation $\sigma_y$ was calculated as
%
\begin{equation} \label{eq:likelihood}
\mathcal{L} = 
\displaystyle\prod_{h}
\displaystyle\prod_{i}
\displaystyle\prod_{t}
\text{Gaussian}
    \left(
        y_{h,i,t}~|~x_{h,i,t},~\sigma_y
    \right).
\end{equation}
%
For population densities that are necessarily non-negative, 
it is common to model the likelihood using a distribution that is similarly constrained,
such as a log-normal distribution. 
However, the multiplicative nature of population processes is already entailed 
in the population projection, 
and a log-normal likelihood reduces the relative contribution of large population sizes
that likely reflect meaningful dynamics.
Therefore, we opted for a Gaussian likelihood. 
Because the model was parameterized in a way that ensured
$x_{h,i,t}$ was non-negative,
the posterior distribution of $x_{h,i,t}$ was also guaranteed to be non-negative. 
We used gamma priors with shape parameter 1.5 and scale parameter 0.75
for the initial population size for each stage $\times$ basin combination, 
initial values for demographic rates,
and standard deviations for the Gaussian penalties and model likelihood.
A gamma distribution with shape parameter of 1.5 has zero density at zero 
and is concave down as it approaches its mode,
allowing the posterior to be arbitrarily close to zero 
while not being artificially drawn towards it.
This shape parameter, along with scale parameter of 0.75,
implies a mean of 2, 
which defines a reasonable scale for all of the parameters 
following the scaling of the population estimates (see below).

We fit the model using Stan 2.19 \citep{carpenter2017}
run from R 4.0.3 \citep{r2020}, with the \texttt{rstan} package \citep{Stan2018}.
To facilitate selection of prior parameterizations,
we scaled the population estimates by dividing by the mean,
such that the resulting data had a mean of 1
and ranged from approximately 0 to 11.
We fit the model with 4 chains, 
8000 iterations (4000 of warm-up and 4000 of sampling),
and set the ``adapt-delta'' parameter to 0.9.
Convergence was assessed by the number of divergent transitions 
and the potential scale reduction factor (\^{R}),
which quantifies the relative variance within and between chains. 
We used posterior medians as point estimates
and quantiles (16\% and 84\%) as uncertainty intervals, 
with coverage analogous to standard errors.

In addition to the full model, 
we also fit three reduced model to which it could be compared:
%
\begin{enumerate}[label=(\arabic*)]
\item
the ``adult dispersal" model omitted juvenile dispersal,
as the fit of the full model implied negligible net dispersal of juveniles between the basins;
%
\item
the ``no dispersal'' model omitted dispersal entirely,
allowing us to assess the contribution of dispersal to the model fit; 
%
\item
the ``fixed rates'' model fixed recruitment, survival, and adult dispersal through time
(with juvenile dispersal omitted altogether),
allowing us to assess the contribution of temporal variation in demographic rates
to the model fit. 
\end{enumerate}
%
We assessed goodness-of-fit using two metrics:
the posterior median of the log-likelihood given by equation \ref{eq:likelihood}
and the ``Leave-one-out Information Criterion'' or LOOIC,
which is analogous to the Akaike Information Criterion (AIC)
and can be interpreted in a similar manner.
We calculated LOOIC using the \texttt{loo} package \citep{loo}.




%========================================================================================

\subsection*{Annual dynamics and sensitivity analysis} 

%========================================================================================

While we parameterized the model in terms of seasonal projections
to accommodate the seasonal nature of the data,
we focused our analysis on the annual dynamics to better reflect
the annual nature of spawning and to circumvent interpretational
issues arising from the unequal projection intervals within a year.
Accordingly, we defined the annual projection matrix as
%
\begin{equation} \label{eq:A}
\mathbf{A}_y = \mathbf{P}_{t[y]+1} \mathbf{P}_{t[y]}
\end{equation}
%
for year $y$ and sequential time steps within that year $t[y]$ and $t[y]+1$,
with the year defined to start with the June census.
$\mathbf{A}_y$ projects the dynamics from
June of one year to June of the next year.
Because $\mathbf{P}_{t[y]$ was defined through June of 2020,
we only calculated $\mathbf{A}_y$ from 1991 through 2019.

We characterized the overall dynamics of the population in terms of the annual 
population growth rate $\lambda_y$, calculated as
%
\begin{equation} \label{eq:lam-n}
\lambda_y = \frac{N_{y+1}}{N_y} = 
              \frac{\mathbf{c}^\top \mathbf{A}_y \mathbf{x}_y}
                    {\mathbf{c}^\top \mathbf{x}_y}} 
\end{equation}
%
where $N_y$ is the summed abundance across basins and life stages in June/July of year $y$
and $\mathbf{c}$ is a 4 $\times$ 1 vector of ones.
Temporal variation in $\lambda_y$ reflects both variation in the demographic rates
and transient fluctuations due to non-equilibrium state distributions. 
Therefore, it is also informative to calculate 
the asymptotic population growth rate that 
would obtain under the equilibrium state distribution in a given time step, 
which is equal to real part of the leading eigenvalue of $\mathbf{A}_y$ 
\citep{caswell2001matrix}.

Both the transient and asymptotic population growth rates appeared to display periodic
behavior for at least a portion of the three decade time series.
We quantified this putative periodicity by applying continuous wavelet transforms to 
the time series for the transient and asymptotic growth rates, 
on a log-scale (the results were similar for the raw values) 
and with no detrending.
Wavelet transforms are a generalization of Fourier transforms,
allowing the decomposition of the signal into periodic elements to be localized in time. 
As our use of wavelet transforms was chiefly descriptive and applied to signals 
that were themselves the outputs of a statistical model,
we did not attempt to apply formal statistical inference (i.e. hypothesis testing)
to the wavelet decomposition.
We conducted the wavelet analysis using the R package \texttt{WaveletComp},
and for tractability we applied the wavelet transform 
to the posterior median of $\lambda_y$ (rather than to multiple Markov chain samples).

We conducted a sensitivity analysis to evaluate the effect of perturbations 
in the demographic rates on the population growth rate, 
using the approach of \cite{caswell2007sensitivity} 
that is applicable to transient dynamics. 
The sensitivity of the population growth rate with respect to a demographic parameter
quantifies how much the growth rate would change in response to a perturbation in the 
demographic parameter. 
In order to compare across parameters of different values 
(which is particularly relevant in the present context with time-varying rates),
it is common to calculate proportional change in  
response to proportional perturbations,
otherwise known as ``elasticity''.
For each year, we calculated the elasticity of the annual growth rate with respect
to the demographic rates at each of the two time steps within that year.
To simplify the presentation, 
we added together the two elasticities for a given demographic parameter 
(e.g. recruitment of south basin juveniles) for each year.
We report the sensitivity analysis for both transient growth rates,
although the asymptotic results were very similar.
Transient sensitivity analysis propagates perturbations in the demographic rates 
through time, 
such that transience due to non-equilibrium state distributions is attributed to the
demographic parameters resulting in the non-equilibrium state distribution 
for a given time step
\citep[][relevant equations are reproduced in Appendix III]{caswell2007sensitivity}.
The propagation interval for each perturbation was one year, 
such that we obtained a sequence of yearly elasticities 
for each seasonal demographic rate.
The sensitivity analysis was performed for 2000 samples of the Markov chain generated 
during fitting of the full demographic model to propagate uncertainty parameter estimates.





% ---------------------------------------------------------------------------------------
% ---------------------------------------------------------------------------------------
% Results
% ---------------------------------------------------------------------------------------
% ---------------------------------------------------------------------------------------

% ---------------------------------------------------------------------------------------
\section*{Results}
% ---------------------------------------------------------------------------------------

The full and adult-dispersal models fit the data similarly well (Table \ref{tab:compare})
and provided visually indistinguishable estimates of 
relative abundance (Figure \ref{fig:fit}). 
This result was corroborated by the negligible net-dispersal
of juveniles inferred by the full model.
Negligible net-dispersal does not necessarily imply 
that juveniles did not disperse between basins,
but rather that there was no clear signature of differential dispersal 
in the population dynamics.
In contrast, the model omitting both juvenile and adult dispersal fit
the data more poorly than the models including adult dispersal.
The effect of removing adult dispersal on the abundance estimates was subtle
for North Basin juveniles and adults in both basins,
with the no-dispersal model showing slightly less flexibility in matching 
relatively large fluctuations in the data than the models including adult dispersal.
However, 
the effect of eliminating adult dispersal was more conspicuous for South Basin juveniles,
as the model without dispersal substantially overestimated abundance relative to the data
in several years.
This is because the abundance of South Basin juveniles in those time points was too low 
to account for the subsequent abundance of South Basin adults 
in the absence of subsidies from the North Basin.
Fixing the demographic rates through time resulted in a much worse fit to the data,
with the abundance estimates rapidly attaining an equilibrium state-distribution
and thereby failing the characterize variation in the data.
Together, these results indicated that temporal variation in the demographic rates,
including adult dispersal, were important in accounting 
for the observed dynamics of the population.
However, dispersal of juveniles appeared to have a negligible effect on the dynamics.
Therefore, we present results from the adult-dispersal model hereafter.

The demographic rates fluctuated substantially over the study period,
resulting in large changes in abundance. 
The overall magnitudes of per capita recruitment were similar between the two basins,
although South Basin recruitment was more variable (Figure \ref{fig:rec}a,b).
Covariance in recruitment between the two basins was modest relative 
to the respective variances, 
with several major changes (e.g., the sharp decline following a peak in 2015)
manifesting in both basins.
Survival probabilities covaried substantially across stage classes within basins
and also covaried between basins, particularly for adults (Figure \ref{fig:surv}a,b).
However, the synchronization between basins was primarily visible prior to 2005, 
after which survival probabilities of both stage classes 
in the South Basin sharply declined.
This result is consistent with the persistently low catch of South Basin adults 
over the last decade (Figure \ref{fig:data}b).
In Figure \ref{fig:disp}, we only show the net number of individuals
that dispersed between basins (rather than basin-specific dispersal probabilities),
as this is most dynamically relevant and thereby ``visible'' to the model when fit to the data.
Net dispersal was generally low and punctuated by several ``waves'' of movement from 
north to south. 
While these waves of southward dispersal persisted throughout the time series,
they were substantially larger prior to 2005 than afterwards.
The direction of net dispersal was only northward in a few time points, 
and even then only of meaningful magnitude in two consecutive time points 
following a particularly large southward dispersal event.

We used both the transient and asymptotic growth rates 
to characterize the annual population dynamics.
The transient growth rate quantified the dynamics as they actually occurred,
including fluctuations due to the non-equilibrium distribution of individuals across 
population states (i.e., basin $\times$ stage-class combinations).
In contrast, the asymptotic growth rate assumed an equilibrium state-distribution 
at each time point and thereby isolated the direct effects of the demographic rates.
Both the transient and asymptotic growth rates fluctuated substantially 
across the study period, 
indicating periods of rapid population growth and decline (Figure \ref{fig:lam}a).
Over the first two decades, 
the asymptotic growth rate was cyclic with a period of approximately 6 years
(Figure \ref{fig:lam}b).
However, this was supplanted by higher-frequency fluctuations in the last 10 years,
which are clearly visible in Figure \ref{fig:lam}a.
The 6-year periodicity was much weaker for the transient growth rate,
which instead was dominated by two bouts of high-frequency fluctuations
at the beginning and end of the study period.
Together, these results indicate that there was a cyclic aspect to the large fluctuations 
of M\'{y}vatn's stickleback population, 
although the dynamics appear to have changed over the last decade.
Furthermore, transience due to non-equilibrium state distributions
largely obscured the cyclicity in the realized population dynamics.

We assessed the contributions of each demographic rate to the population dynamics
using transient elasticities,
which quantified the proportional change in the population growth rate in 
response to proportional perturbations in demographic rates.
Becuase the elasticities were calculated for the transient growth rate,
they quantified both the direct effect of changes in demographic rates 
and the indirect effects resulting from changes in the equilibrium ditribution
across basin $\times$ stage-class combinations.
The elasticities were neutral or positive in most years for all demographic rates 
except for dispersal probabilities from the North to the South Basin,
which were substantially negative in most years.
The negative elasticities for southward dispersal indicated that conditions were 
demographically less favorable in the South Basin than in the North,
such that southward dispersal generally reduced the population growth rate.
Among the remaining demographic rates,
the largest elasticities were for juvenile survival, recruitment, and adult survival
in the North Basin, while the analogous rates for the South Basin were generally close to zero.
This pattern in especially pronounced in the most recent years,
coindiciding with low survival (Figure \ref{fig:surv}a) 
and adult abundance (Figure \ref{fig:data}b) in the South Basin.
Together, these results show that the North Basin dominated 
the overall population dynamics, 
and this dominance increased towards the end of the study period.





% ---------------------------------------------------------------------------------------
% ---------------------------------------------------------------------------------------
% Discussion
% ---------------------------------------------------------------------------------------
% ---------------------------------------------------------------------------------------



\section*{Discussion}






% ---------------------------------------------------------------------------------------
% ---------------------------------------------------------------------------------------
% References,figures, etc.
% ---------------------------------------------------------------------------------------
% ---------------------------------------------------------------------------------------

\bibliographystyle{ecology.bst}
\clearpage

\bibliography{refs.bib}

% ---------------------------------------------------------------------------------------
\clearpage
\begin{table}
\caption{\label{tab:compare}
.
}
\setlength{\tabcolsep}{12pt}
\begin{tabular}{lcc}
\toprule
Model                  &    Log-likelihood (posterior median) & LOOIC \\
\cmidrule{1-3}
full model             & -141 & 532 \\
adult dispersal        & -142 & 542 \\
no dispersal           & -223 & 654 \\
fixed rates            & -380 & 790 \\
\bottomrule
\end{tabular}
\end{table}
\clearpage
% ---------------------------------------------------------------------------------------

% ---------------------------------------------------------------------------------------
\clearpage
\begin{figure}
\centering
\includegraphics{../analysis/figures/fig_data.pdf}
\caption{\label{fig:data}
(a) Distribution of trapping sites within M\'{y}van's
South (23, 27, 44, 44, and 135) and North Basins (DN, 124, and 124). 
Gray areas indicate water and white areas indicate land.
(b) Trap-level catch (points) and 
population density estimates from the N-mixture model by basin and stage-class (black line).
For the sake of visualization, the density estimates were scaled 
by the catch rate averaged (on the logit-link scale) across trapping periods,
and trapping sites for a given basin $\times$ stage combination.
}
\end{figure}
\clearpage
% ---------------------------------------------------------------------------------------

% ---------------------------------------------------------------------------------------
\clearpage
\begin{figure}
\centering
\includegraphics{../analysis/figures/fig_fit.pdf}
\caption{\label{fig:fit}
Relative abundances (points) and fitted values from different versions 
of the demographic model (lines).
We calculated relative abundances by scaling the density estimates from the 
N-mixture model by the sampling area spanned by the traps within each basin. 
Solid lines are posterior medians,
and shaded regions are 68\% uncertainty intervals, 
analagous to the coverage of standard errors.
Note that the fitted values for the full and adult-dispersal models 
are visually indistinguishable.
}
\end{figure}
\clearpage
% ---------------------------------------------------------------------------------------

% ---------------------------------------------------------------------------------------
\clearpage
\begin{figure}
\centering
\includegraphics{../analysis/figures/fig_rec.pdf}
\caption{\label{fig:rec}
(a) Per capita recruitment as inferred from the adult-dispersal model.
Solid lines are posterior medians,
and shaded regions are 68\% uncertainty intervals, 
analagous to the coverage of standard errors.
(b) Covariance matrix for per capita recruitment by basin,
calculated from posterior medians.
The color shading corresponds to the magnitude of the covariance term,
which is also shown numerically. 
The values were scaled by the mean for the entire matrix to emphasize 
the relative magnitudes among the covariance terms.
}
\end{figure}
\clearpage
% ---------------------------------------------------------------------------------------

% ---------------------------------------------------------------------------------------
\clearpage
\begin{figure}
\centering
\includegraphics{../analysis/figures/fig_surv.pdf}
\caption{\label{fig:surv}
(a) Surival probabilites as inferred from the adult-dispersal model.
Solid lines are posterior medians,
and shaded regions are 68\% uncertainty intervals, 
analagous to the coverage of standard errors.
(b) Covariance matrix for survival probability by basin $\times$ stage combination,
calculated from posterior medians.
The color shading corresponds to the magnitude of the covariance term,
which is also shown numerically. 
The values were scaled by the mean for the entire matrix to emphasize 
the relative magnitudes among the covariance terms.
}
\end{figure}
\clearpage
% ---------------------------------------------------------------------------------------

% ---------------------------------------------------------------------------------------
\clearpage
\begin{figure}
\centering
\includegraphics{../analysis/figures/fig_disp.pdf}
\caption{\label{fig:disp}
Net dispersal of adults between basins as inferred from the adult-dispersal model.
We calculated net dispersal as the flux of individuals (in units of relative density)
from south to north (``northward'') minus the flux from north to south (``southward'').
The model was formulated such that dispersal was calculated after survival and stage-transitions,
which was reflected in the calculation of net dispersal.
Solid lines are posterior medians,
and shaded regions are 68\% uncertainty intervals, 
analagous to the coverage of standard errors.
Posterior summaries were applied to the calculation of net dispersal itself,
rather than net dispersal being calculated from posterior summaries.
}
\end{figure}
\clearpage
% ---------------------------------------------------------------------------------------

% ---------------------------------------------------------------------------------------
\clearpage
\begin{figure}
\centering
\includegraphics{../analysis/figures/fig_lam.pdf}
\caption{\label{fig:lam}
(a) Transient and asymptotic population growth rates,
projected annually from the adult-dispersal model.
Solid lines are posterior medians,
and shaded regions are 68\% uncertainty intervals, 
analagous to the coverage of standard errors.
The dashed horizontal line shows $\lambda=1$, 
which corresponds to no change in the population size.
(b) Periodigrams from wavelet transforms of transient and asymptotic growth rates,
with darker shading representing stronger signal associated with a given periodic element.
The black contour lines are based on signal quantiles, and denote regions of high signal.
While the wavelet decomposition was conducted for periods up to the maximum period length
(29 years), the signal associated with periods >10 years was very weak. 
So, for clarity the figure truncates the periodigram at 11 years.
}
\end{figure}
\clearpage
% ---------------------------------------------------------------------------------------

% ---------------------------------------------------------------------------------------
\clearpage
\begin{figure}
\centering
\includegraphics{../analysis/figures/fig_elas.pdf}
\caption{\label{fig:elas}
Elasticity analysis of the transient population growth rate 
with respect to the time-varying demographic rates.
Points are posterior medians,
and shaded regions are 68\% uncertainty intervals, 
analagous to the coverage of standard errors.
The black vertical lines show time-averages of the posterior medians 
for each demographic parameter to aid visualization.
}
\end{figure}
\clearpage
% ---------------------------------------------------------------------------------------




% ---------------------------------------------------------------------------------------
% ---------------------------------------------------------------------------------------
% Appendices
% ---------------------------------------------------------------------------------------
% ---------------------------------------------------------------------------------------

\renewcommand{\thefigure}{A\arabic{figure}}
\renewcommand{\theequation}{A\arabic{equation}}
\renewcommand{\thetable}{A\arabic{table}}
\setcounter{equation}{0}
\setcounter{figure}{0}
\setcounter{table}{0}

% ---------------------------------------------------------------------------------------
\section*{Appendix I: N-mixture model} 
% ---------------------------------------------------------------------------------------

To account for potential variation in trapping rate between stage classes,
trapping sites, and time of day, 
we used a modified N-mixture model to estimate relative basin-level population densities
for the two stage classes.
The probability of trapping some number of individuals $y_i$ 
was modeled as
%
\begin{equation}
  y_i \sim \text{Binomial}\left(\eta_{g_{\eta}[i]}, ~\nu_{g_{\nu}[i]}\right)
\end{equation}
%
where $\eta_{g_{\eta}[i]}$ is the trapping rate 
and $\nu_{g_{\nu}[i]}$ is the population density for a given 
station-size-date combination for the $i$th observation.
The density is in units of individuals per station, 
with each station characterizing a sampling area that is 
taken to be the same size for all stations.
The functions ${g_{\eta}[i]}$ and ${g_{\nu}[i]}$ map observations to the 
appropriate grouping. 
The discrete population density for each station-size-date combination was modeled as
%
\begin{equation}
  \nu_{g_{\nu}[i]} \sim \text{Poisson}\left(\kappa_{g_{\kappa}[i]}\right)
\end{equation}
%
where $\kappa_{g_{\kappa}[i]}$ is the mean population density across stations 
within a basin-size-date combination. 
Variation in $\kappa_{g_{\kappa}[i]}$ across basin, size, 
and date was characterized as 
%
\begin{equation}
  \kappa_{g_{\kappa}[i]} \sim 
    \text{Exponential}\left(\zeta \right)
\end{equation}
%
with rate parameter $\zeta = 0.001$ (implying a mean of 1000).
The trapping rate for the $i$th observation was modeled using a 
``logistic regression''-style approach:
%
\begin{equation}
  \eta_{g_{\eta}[i]} = 
    \text{logit}^{-1}\left(\mathbf{z}_{g_{\eta}[i]}^\top~{\boldsymbol\beta}\right)
\end{equation}
%
\noindent where $\mathbf{z}_{g_{\eta}[i]}^\text{T}$ is a transposed vector 
of predictor values for the $i$th observation
(including 1 in the first column for the overall intercept)
and $\boldsymbol\beta$ is a vector of coefficients. 
We included main effects for station, trapping time (day vs. night), size class,
and the size class $\times$ trapping time interaction.
We used a Gaussian prior with mean of 0 and standard deviation of 2 
for the coefficients $\boldsymbol\beta$.
The model was fit using JAGS via \texttt{runjags} in R4.0.0 with,
using 4 chains and 15000 iterations (5000 adaptation, 5000 burn-in, and 5000 sampling).
Convergence was assessed by the potential scale reduction factor (\^{R}),
which quantifies the relative variance within and between chains. 




% ---------------------------------------------------------------------------------------
\section*{Appendix II: Transition rates} 
% ---------------------------------------------------------------------------------------

We parameterized latent transition rate matrices for mortality ($\boldsymbol\Omega^{\mu}_t$), 
stage-transition ($\boldsymbol\Omega^{\gamma}$), 
and dispersal ($\boldsymbol\Omega^{\delta}_t$) as:
 %
\begin{equation} \label{eq:Theta}
\begin{aligned}
\boldsymbol\Omega^{\mu}_t & = 
\left[
\begin{array}{cc|cc}
    -\omega^{\phi_{j,s}}_t & 0 & 0 & 0 \\
    0 & -\omega^{\phi_{a,s}}_t & 0 & 0 \\
    \hline
    0 & 0 & -\omega^{\phi_{j,n}}_t & 0 \\
    0 & 0 & 0 & -\omega^{\phi_{a,n}}_t \\
    \end{array}
\right] \\
\boldsymbol\Omega^{\gamma} & = 
\left[
\begin{array}{cc|cc}
    -\omega^{\gamma_{j}} & 0 & 0 & 0 \\
    \omega^{\gamma_{j}}  & 0 & 0 & 0 \\
    \hline
    0 & 0 & -\omega^{\gamma_{j}} & 0 \\
    0 & 0 & \omega^{\gamma_{j}}  & 0 \\
    \end{array}
\right] \\
\boldsymbol\Omega^{\delta}_t & = 
\left[
\begin{array}{cc|cc}
    -\omega^{\delta_{j,s}}_t & 0 & \omega^{\delta_{j,n}}_t & 0 \\
    0 & -\omega^{\delta_{a,s}}_t & 0 & \omega^{\delta_{a,n}}_t \\
    \hline
    \omega^{\delta_{j,s}}_t & 0 & -\omega^{\delta_{j,n}}_t & 0 \\
    0 & \omega^{\delta_{a,s}}_t & 0 & -\omega^{\delta_{a,n}}_t \\
    \end{array}
\right]
\end{aligned}
\end{equation}
%
Note that mortality implicitly entails transition to a ``death state'' that is omitted 
for succinctness, as dead individuals do not contribute to future transitions.
For each transition matrix $\boldsymbol\Omega^{\alpha}_t$, 
we then calculated the probability of transitioning as
%
\begin{equation} \label{eq:Psi}
\boldsymbol\Psi^{\alpha}_t = e^{\boldsymbol\Omega^{\alpha}_t}
\end{equation}
%
which is the solution to the differential equation associated with the Markov process
specified by $\boldsymbol\Omega^{\alpha}_t$ 
with initial condition equal to the 4 $\times$ 4 identity matrix.
The unequal projection interval duration from June-August and August-June was handled 
implicitly by the time-varying rates, 
which proved more computationally stable than explicitly accounting for the projection
interval duration in equation \ref{eq:Psi}.

The transition probability matrix was calculated as 
 %
\begin{equation} \label{eq:POmega}
\mathbf{P}_{t} = \boldsymbol\Psi^{\delta}_t~\boldsymbol\Psi^{\gamma}~\boldsymbol\Psi^{\mu}_t.
\end{equation}
%
The order of mutiplication implies that proportional survival is calculated first,
followed by stage-transtions, and finally dispersal,
resulting in the configuration of transition probalities
given in equations \ref{eq:W} and \ref{eq:B}.
In principle, we could have included all of the demographic transitions in a single
transition matrix,
which would imply that all of the transition processes occurred simultaneously. 
However, modeling the different transition processes sequentially facilitated interpretation
of the resulting transition probabilities 
(i.e., the matrix elements equations \ref{eq:W} and \ref{eq:B}), 
as they would only pertain to a single type of demographic transition
rather than multiple transition processes occurring simultaneously.
This also facilitated convergence of the MCMC algorithm during model fitting, 
for much the same reasons.




% ---------------------------------------------------------------------------------------
\section*{Appendix III: Sensitivity analysis} 
% ---------------------------------------------------------------------------------------

We used the method of \cite{caswell2007sensitivity} to calculate the elasticities 
(proportional sensitivities) of the annual transient population growth rate $\lambda_y$
with respect to perturbations in the seasonal demographic rates.
It was convenient to perform the calculations using the logarithm of $\lambda_y$,
commonly denoted $r_y$.
This parameter is related to total population size $N_y$ by the expression
%
\begin{equation} \label{eq:r}
r_y = \text{log}\left(N_{y+1}\right) - \text{log}\left({N_y}\right).
\end{equation}
%
Note that
%
\begin{equation} \label{eq:lsens}
\frac{\text{d}\lambda_y}{\text{d}\theta} = \lambda \frac{\text{d}r_y}{\text{d}\theta}
\end{equation}
%
where $\frac{\text{d}\lambda_y}{\text{d}\theta}$ can generically be interpreted
as the sensitivity of $\lambda$ with respect to perturbations 
in a single parameter $\theta$.
The elasticity of $\lambda$ is then defined as
%
\begin{equation} \label{eq:lelas}
\frac{\theta}{\lambda_y} \frac{\text{d}\lambda_y}{\text{d}\theta} = 
        \theta\frac{\text{d}r_y}{\text{d}\theta}.
\end{equation}
%
The multiplication of $\frac{\text{d}r_y}{\text{d}\theta}$ by $\theta$  
implies proportional perturbations in $\theta$.
Therefore, the sensitivity of $r_y$ with respect to proportional perturbations in $\theta$
equals the elasticity of $\lambda_y$.
This deduction is essentially a restatement of logarithmic relationship of $\lambda_y$
and $r_y$, along with the properties of logarithmic derivatives.

The transient sensitivity of $r_y$ with respect to perturbations in demographic parameters
is defined as
%
\begin{equation} \label{eq:dr}
\frac{\text{d}r_y}{\text{d}\boldsymbol\theta_y^\top} = 
    \frac{\mathbf{c}^\top}{N_{y+1}} \frac{\text{d}\mathbf{x}_{y+1}}
            {\text{d}\boldsymbol\theta_{y+1}^\top}-
        \frac{\mathbf{c}^\top}{N_{y}} \frac{\text{d}\mathbf{x}_y}
            {\text{d}\boldsymbol\theta_y^\top}
\end{equation}
%
where $\boldsymbol\theta_y$ is a vector of demographic parameters, 
$\mathbf{x}_y$ is a 4 $\times$ 1 vector of abundances in each state,
$\mathbf{c}$ is a 4 $\times$ 1 vector of ones,
and ``$\text{d}$'' is the derivative operator.
We were interested in the sensitivity of $r_y$ with respect to proportional 
perturbations in the seasonal demographic rates,
which are connected to $\mathbf{x}_y$ through the annual population projection matrix
$\mathbf{A}_y$ as defined in equation \ref{eq:A}.
If $\boldsymbol\theta_y$ contains the seasonal demographic rates 
(i.e., the collective elements of $\mathbf{P}_{t[y]}$ and $\mathbf{P}_{t[y]+1}$)
and $\boldsymbol\epsilon_y$ is a vector of proportional perturbations in
$\boldsymbol\theta_y$,
then 
%
\begin{equation} \label{eq:dx}
\frac{\text{d}\mathbf{x}_{y+1}}{\text{d}\boldsymbol\theta_{y+1}^\top} = 
    \mathbf{A}_y \frac{\text{d}\mathbf{x}_{y}}{\text{d}\boldsymbol\theta_y^\top}+
        \left(\mathbf{x}_{y}^\top \otimes \mathbf{I}_c \right)
            \frac{\text{dvec}\mathbf{A}_y}{\text{d}\boldsymbol\epsilon_y^\top}
                \text{diag}\boldsymbol\epsilon_y
\end{equation}
%
where $\mathbf{I}_c$ is the $c \times c$ identity matrix with $c$ as the length
of the parameter vector $\theta_y$,
``$\text{vec}$'' is an operator that creates a vector by stacking columns of the operand matrix,
and ``$\text{diag}$'' is an operator that creates a square matrix with the operand vector on
the diagonal and zeros elsewhere. 
Defining an initial population size distribution $\mathbf{x}_0$ 
that is independent of the demographic parameters implies that 
$\frac{\text{d}\mathbf{x}_0}{\text{d}\boldsymbol\theta_0^\top} = \mathbf{0}$.
Using this initial condition,
the sensitivities can then be calculated by iterating equations \ref{eq:dr} and \ref{eq:dx}
for each year, with perturbations $\boldsymbol\epsilon_y$ proportional (or equal)
to the parameter vector $\boldsymbol\theta_y$.
Asymptotic results can be obtained by iterating \ref{eq:dx} many times for a given year,
which eliminates the dependence on the initial values such that each year can be treated
independently. 
The derivatives of the elements of $\mathbf{A}_y$ with respect to the seasonal
demographic rates necessary for evaluating \ref{eq:dx} can be computed analytically
and are shown in the Supplemental Materials.




\end{document}


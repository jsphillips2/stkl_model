% ---------------------------------------------------------------------------------------
% ---------------------------------------------------------------------------------------
% Preliminaries
% ---------------------------------------------------------------------------------------
% ---------------------------------------------------------------------------------------

\documentclass[11pt]{article}
\usepackage[letterpaper, margin=1in]{geometry}
\usepackage{newtxtext,newtxmath}
\usepackage[math-style=ISO]{unicode-math}
\usepackage{fullpage}
\usepackage[authoryear,sectionbib]{natbib}
\linespread{1.7}
\usepackage[utf8]{inputenc}
\usepackage{lineno}
\usepackage{titlesec}
\titleformat{\section}[block]{\Large\bfseries\filcenter}{\thesection}{1em}{}
\titleformat{\subsection}[block]{\Large\itshape\filcenter}{\thesubsection}{1em}{}
\titleformat{\subsubsection}[block]{\large\itshape}{\thesubsubsection}{1em}{}
\titleformat{\paragraph}[runin]{\itshape}{\theparagraph}{1em}{}[. ]\renewcommand{\refname}
  {Literature Cited}
\DeclareTextSymbolDefault{\dh}{T1} % for Icelandic ð symbol:
\usepackage{graphicx} % for figures
\usepackage{booktabs} % for tables
\usepackage{amsmath} % for split math environment
\usepackage{enumitem}




% ---------------------------------------------------------------------------------------
% ---------------------------------------------------------------------------------------
% Title page
% ---------------------------------------------------------------------------------------
% ---------------------------------------------------------------------------------------


\title{Transient dynamics of spatially-structured fluctuations 
in a threespine stickleback population}

\author{
Joseph S. Phillips$^{1,2, \dagger}$ \\
\'{A}rni Einarsson$^{3}$ \\ 
Kasha Strickland$^{1}$ \\
Anthony R. Ives$^{2}$ \\
Bjarni K. Kristj\'{a}nsson$^{1}$ \\
Katja R\"{a}s\"{a}nen$^{4}$ 
}

\date{}

\begin{document}

\raggedright
\setlength\parindent{0.25in}

\maketitle


\noindent{} 1. Department of Aquaculture and Fish Biology, 
H\'{o}lar University, Skagafj\"{o}r{\dh}ur 551 Iceland

\noindent{} 2. Department of Integrative Biology, 
University of Wisconsin, Madison, Wisconsin 53706 USA

\noindent{} 3. M\'{y}vatn Research Station, IS-660 M\'{y}vatn, Iceland

\noindent{} 4. Department of Aquatic Ecology, EAWAG and 
Institute of Integrative Biology, ETH Zurich, 
\"{U}berlandstrasse 133, CH-8600 D\"{u}bendorf, Switzerland

\noindent{} $\dagger$ E-mail: joseph@holar.is



\bigskip

Running head: {Spatially-structured population cycles}

\linenumbers{}

\clearpage





% ---------------------------------------------------------------------------------------
% ---------------------------------------------------------------------------------------
% Abstract
% ---------------------------------------------------------------------------------------
% ---------------------------------------------------------------------------------------

\section*{Abstract} \label{abstract}

Uncovering the demographic basis of population fluctuations is challenging 
for spatially structured populations, 
because this requires disentangling synchrony in demographic rates 
from coupling via dispersal. 
In this study, we fit a stage-structured metapopulation model to a 29-year times series 
of population density estimates of threespine stickleback in the heterogeneous 
and productive Lake M\'{y}vatn, Iceland. 
The lake comprises two basins (north and south) connected by a channel, 
through which the stickleback disperse. The model includes time-varying demographic rates, 
allowing us to disentangle the contributions of recruitment and survival, 
spatial coupling via dispersal, and transient dynamics 
to the population’s large fluctuations in abundance. 
Our analyses indicate that recruitment is largely unsynchronized between the two basins, 
whereas survival probability of adults is highly synchronized, 
contributing to cyclic fluctuations in the lake-wide population 
with a period of approximately 6 years. 
The analyses further show that the two basins are strongly coupled through dispersal, 
with the north basin subsidizing the south basin and playing a dominant role 
in driving the lake-wide dynamics. 
Moreover, transience due to non-equilibrium distributions of individuals 
across demographic states results in large short-term changes in abundance 
that partially obscure the cyclic nature of the dynamics. 
Our results show that cyclic fluctuations can arise through both spatial coupling 
and synchronization of demographic rates, and they highlight the importance 
of accounting for transience in analyses of population dynamics.


\bigskip

\textit{Keywords}: {demography, \emph{Gasterosteus aculeatus}, M\'{y}vatn, 
                    population cycles, source-sink dynamics, state-space model}

\clearpage



% ---------------------------------------------------------------------------------------
% ---------------------------------------------------------------------------------------
% Introduction
% ---------------------------------------------------------------------------------------
% ---------------------------------------------------------------------------------------

\section*{Introduction} \label{introduction}

Population fluctuations are of enduring interest in ecology,
with particular attention paid to periodic or quasi-periodic dynamics
\citep{elton1924, myers2018}.
Various mechanisms have been posited for cyclic fluctuations,
including periodic environmental variation and 
interspecific interactions (e.g., predator-prey; host-parasitoid)
that induce periodicity endogenously 
\citep{nicholson1935, andrewartha1954, rosenzweig1963}. 
Furthermore, exogenous and endogenous factors can interact with each other,
with the potential to produce a wide array of dynamics with
varying degrees of periodicity, as well as more exotic features
\citep{bjornstad2001, turchin2003, ives2008}.

Many populations are spatially structured \citep{hanski1998},
setting the stage for spatially-structured fluctuations 
\citep{bjornstad1999spatial, gouveia2016}.
Indeed, some populations are synchronized across wide geographic extents,
which raises the question of how such synchrony arises
\citep{ranta1995synchrony, krebs2002}.
Perhaps the most widely discusssed mechanism is the Moran effect,
whereby spatial synchrony in some exogenous driver induces synchrony
in the populations under its influence
\citep{moran1953}. 
Spatial coupling through dispersal between sub-populations of a given species
\citep{liebhold2004, goldwyn2008}
or movement of another species with which the focal sub-populations interact
\citep{gilg2009, ims2000}
is another important mechanism of synchronous fluctuations.
Synchrony in exogenous drivers and spatial coupling through dispersal 
can interact with each other and with the internal dynamics of a given population,
complicating the way in which these factors map onto population synchrony
\citep{ranta1995synchrony, kendall2000dispersal, abbott2011}.

Both the Moran effect and spatial coupling can be characterized as demographic processes,
with the former inducing synchrony in demographic rates such as recruitment or survival
and the latter arising from immigration between sub-populations.
From this framing,
a third source of population fluctuations (potentially synchronous) becomes clear:
trasient oscillations due to non-equilibrium distribution of individuals across
demographic states
\citep{caswell2001matrix, koons2017understanding}.
In age- and stage-structured populations, 
such transient oscillations can be very large and alter the long-term dynamics 
through various mechanisms,
such as ``transient amplification''
\citep{neubert1997, tenhumberg2009}. 
In spatially-structured populations, 
discerete patches play an analagous role to discrete life-stages 
and can similarly exhibit non-equilibrium distributions with associated transience
\citep{ozgul2009}.
When demographic rates vary through time,
as is inherent to the Moran effect,
a population may persistently occupy non-equilibrium state distributions,
which in turn means that transient oscillations may be ubiquitous features of the dynamics.
Therefore, accounting for effects transience may be important for 
fully characterizing the nature and causes of spatially-structured fluctuations
\citep{hastings2010}.

In this study, 
we use a population-dynamics model to analyze spatially-structured fluctuations 
in a population of threespine stickleback (\emph{Gasterosteus aculeatus})
in Lake M\'{y}vatn, Iceland.
M\'{y}vatn is particularly well suited to this exploration for two primary reasons.
First, the stickleback population is spatially structured by the unique geomorphology
of the lake \citep{gislason1998, millet2013}, 
which is divided into two basins connected by two narrow channels.
The southern basin (Syðrifloi) is the larger of the two ($28~\text{m}^2$) and is dominated
by exposed sediment and intermittent mats of filamentous green algae
\citep{einarsson2004myvatn}.
In contrast, the northern basin (Ytrifloi) is substantially smaller ($9~\text{m}^2$)
and more spatially heterogeneous, 
in part due to dredging of the lake bottom that substantially altered its bathymetry.
Despite its smaller area, the north basin has historically sustained much higher
densities of threespine stickleback  \citep{gislason1998}, 
likely owing to the ecological differences between the basins. 

Second, M\'{y}vatn's stickleback population fluctuates substantially through time.
While the causes of these fluctuations are unknown, 
they are likely connected to the large temporal variability 
of other populations in the lake.
M\'{y}vatn is naturally eutrophic due to the inflows of nutrient-rich springs,
which sets the stage for high-amplitude fluctuations in secondary producers
\citep{einarsson2004myvatn}.
Chief among these are chironomid midges and cladocerans
\citep{einarsson2002, einarsson2004clad, gardarsson2004population, ives2008},
both of which are potential food sources for threespine stickleback.
Furthermore, the lake hosts temporally variable populations 
of arctic charr (\emph{Salvelinus alpinus}), 
brown trout (\emph{Salmo trutta}), 
and piscivorous birds that have the potential 
to induce fluctuations in the stickleback population from the top down
\citep{einarsson2004moulting, gudbergsson2004}.
Finally, M\'{y}vatn's stickleback can sustain high loads of the cestode parasite
\emph{Schistocephalus solidus} \citep{gislason1998, karvonen2013},
which could lead to host-parasite fluctuations.

To explore the spatiotemporal dynamics of M\'{y}vatn's stickleback population, 
we fit a stage-structured metapopulation model \citep{caswell2001matrix}
to a 29-year times series of population estimates derived from trapping data.
The model includes temporal variation in demographic rates such as recruitment and survival,
which were characterized as random walks constrained by the observed data.
This approach provides great flexibility in modeling changes in the demographic rates,
including those implicitly arising from nonlinear and density-dependent processes
\citep{zeng1998, ives2012}.
Equipped with the parameterized model, we address the following questions:
%
\begin{enumerate}[label=(\alph*)]
\item
To what extent are stickleback population fluctuations periodic, 
and to what extent does transience arising from non-equilibrium state distributions
alter this periodicity;
%
\item
To what extent are the two sub-populations coupled through dispersal,
and how does this affect the dynamics; 
%
\item
To what extent are the demographic rates synchronized between the two sub-populations,
and what are the relative contributions of within-basin demography 
and between-basin coupling to the lake-wide population dynamics.
\end{enumerate}
%
By investigating these topics, 
we seek to illustrate the demographic basis for spatially-structured population fluctuations
in a wild population, 
which may provide general lessons for other systems. 




% ---------------------------------------------------------------------------------------
% ---------------------------------------------------------------------------------------
% Methods
% ---------------------------------------------------------------------------------------
% ---------------------------------------------------------------------------------------

% ---------------------------------------------------------------------------------------
\section*{Methods} 
% ---------------------------------------------------------------------------------------



% ---------------------------------------------------------------------------------------
\subsection*{Data and population estimates} 
% ---------------------------------------------------------------------------------------

From 1991 to 2020, surveys of the stickleback population were conducted twice annually, 
the first in either June or July and second in August or September. 
Samples were collected from 5 stations in the south basin and 3 stations in the north basin.
These stations provided wide coverage of the two basins, 
with the exception of sites near the shoreline 
(sampling of shoreline sites began in 2008 but those data are not included here)
and of the eastern portion of the south basin which historically has had negligible 
densities of sticklebacks.
For each sampling event at each station, 
5 traps were set for two 12-hour sessions, 
one during the day and one during the night.
Individuals were sorted into two size classes before being counted,
with a threshold of 50mm in the June/July sampling and 45mm in the August/September sampling.
These size categories roughly map onto sexual maturity,
and while although there is likely variation in size-at-maturation \citep{singkam2019}
it should serve as a reasonable approximation for the purpose 
of modeling the population dynamics.
Within each basin and size class, 
station catches were of comparable magnitude and 
strongly synchronized through time (Supplemenal materials).
Therefore, our analysis focused on the dynamics of basin-level abundances
of the two size classes through time.

To account for systematic variation in trapping probability between size classes,
trapping stations, and time of day, 
we fit a modified N-mixture model \citep{royle2004}
to the trap-level data to estimate mean population density across sampling stations 
for each basin $\times$ size-class combination at each time point.
(Appendix I). 
We then scaled these relative density estimates by the relative samplings areas 
of the two basins, 
which we defined as the entire north basin and the south basin west of a chain
of islands beyond which there are no trapping data 
due to historically low stickleback abudance.
This resulted in a 2:1 relative scaling for the south vs. north basin.
Both the relative density and relative abundance (after scaling by area) of threespine
stickleback was generally higher in the north basin than in the south,
consistent with a mark-recapture study conducted in 1989 \citep{gislason1998}



% ---------------------------------------------------------------------------------------
\subsection*{Metapopulation model} 
% ---------------------------------------------------------------------------------------

We used a stage-structured metapopulation model \citep{caswell2001matrix}
with time-varying demographic rates
to characterize the dynamics of the stickleback population. 
The model projected the population dynamics due to 
within-basin recruitment, survival, and development, 
as well as dispersal between basins. 
Recruitment, survival, and dispersal were allowed to vary through time,
enabling the model to characterize a range of dynamics, 
including those implicitly due to endogenous (e.g., density dependence) 
and exogenous (e.g, environmental variation) processes
\citep{zeng1998, ives2012}. 
We estimated the demographic rates by  
fitting the model to the time series of abundance estimates. 
In general terms, this approach works by reconstructing the demographic rates required
to project the distribution of abundances across demographic states 
from one time step to the next. 
While parameter identifiability poses a challenge for such inferences over a single time
step, by explicitly modeling temporal variation in the demographic rates we were able 
to take advantage of shared information across all time points simultaneously to 
successfully constrain the parameter estimates.
The resulting demographic rates are best understood as parameters 
characterizing the observed population dynamics, 
rather than as individual-level demographic rates that would be estimated through a 
technique such as mark-recapture. 

For a given time interval from $t-1$ to $t$, 
the structured population dynamics were projected as
%
\begin{equation} \label{eq:XPX}
    \mathbf{x}_t = \mathbf{P}_{t-1}~\mathbf{x}_{t-1}
\end{equation}
%
where $\mathbf{P}_{t}$ is 4 $\times$ 4 a matrix of demographic rates at time $t$, 
and $\mathbf{x}_{t}$ is a 4 $\times$ 1 vector of abundances 
for a given stage (juveniles $j$; adults $a$) 
and basin (south $s$; north $n$):
%
\begin{equation} \label{eq:X}
\mathbf{x}_{t} = 
\left[
\begin{array}{cccc}
    {x_{j,s,t}} \\
    {x_{a,s,t}} \\
    {x_{j,n,t}} \\
    {x_{a,n,t}}
    \end{array}
\right]
\text{.}
\end{equation}
%
The projection matrix $\mathbf{P}_{t}$ can be expressed as
%
\begin{equation} \label{eq:P}
\mathbf{P}_{t} = 
\left[
\begin{array}{c|ccc}
    \mathbf{W}_{s,t}  & \mathbf{B}_{s\rightarrow n,t} \\
    \hline
    \mathbf{B}_{n\rightarrow s,t} & \mathbf{W}_{n,t}
    \end{array}
\right]
\end{equation}
%
where $\mathbf{W}_{i,t}$ is a 2 $\times$ 2 matrix characterizing 
per capita contributions within basin $i$,
and $\mathbf{B}_{i\rightarrow k,t}$ is a 2 $\times$ 2 matrix characterizing 
contributions from basin $i$ to basin $k$.
Within-basin contributions were modeled as 
 %
\begin{equation} \label{eq:W}
\mathbf{W}_{i,t} = 
\left[
\begin{array}{cccc}
    \phi_{j,i,t}~(1-\gamma_{j}) & 
    \rho_{i,t} \\
    \phi_{j,i,t}~\gamma_{j}~(1-\delta_{a,i,t}) & 
    \phi_{a,i,t}~(1-\delta_{a,i,t})
    \end{array}
\right]
\end{equation}
%
where $\phi_{h,i,t}$ is the survival probability of life-stage $h$, 
$\gamma_{j}$ is the proportion of surviving juveniles that develop into adults,
$\delta_{a,i,t}$ is the proportion of surviving adults that disperse to the other basin,
and $\rho_{i,t}$ is per capita recruitment.
We modeled between-basin contributions as
%
\begin{equation} \label{eq:B}
\mathbf{B}_{i,t} = 
\left[
\begin{array}{cccc}
    0 & 
    0 \\
    
    \phi_{j,i,t}~\gamma_{j}~\delta_{a,i,t} & 
    \phi_{a,i,t}~\delta_{a,i,t}
    \end{array}
\right].
\end{equation}
%
We fixed $\gamma_{j}$ to a single value for both basins and through time because 
it was difficult to statistically separate changes in development from changes 
in survival probability.
This is unsurprising, as both survival and development probabilities
determined the contribution of juveniles to the adult age class,
and the development probability was sufficiently 
high that few juveniles returned to the juvenile class at the next time step. 
Similarly, we assumed that only individuals that were adults at the end of the projection
interval dispersed, because a model allowing juveniles to disperse failed to detect 
a meaningful signature of net juvenile dispersal 
(again due to the high probability of development over the course of a projection interval).
Taken literally, the model implies that individuals born in a given basin
remain within that basin until the next time step.
However, any recruitment across basins should largely manifest 
as additional temporal variability in the within-basin recruitment,
and therefore is implicitly accommodated by the model.
Furthermore, gravid females and males in breeding coloration are 
routinely found in both basins, making it likely that spawning occurs in each.

To accommodate the unequal duration of the the summer (2-3 months) and winter (9-10 month)
intervals, we modeled the transition probabilities 
($\phi_{h,i,t}$, $\gamma_{j}$, and $\delta_{a,i,t}$)
in terms of latent transition rates $\omega^{\alpha}_{t}$ 
(where $\alpha$ denotes the corresponding demographic parameter).
For each class of transition process (mortality, development, and dispersal),
we specified a transition matrix from which we could then calculate
the probability of transition over a given interval $\Delta T$ 
(see Appendix II).
We did not explicitly model unequal projection intervals for reproduction,
as spawning was not evenly spread throughout the year but instead was concentrated
in the summer (but spanned the ``summer'' and ``winter'' sampling periods).

We modeled temporal variation in demographic rates
(including for $\alpha$ = $\rho_{i,t})$ as random walks:
%
\begin{equation} \label{eq:theta}
    \omega^{\alpha}_t & \sim \text{Normal}
        \left(
            \omega^{\alpha}_{t-1},~\sigma_{g[\alpha]}}
        \right) \text{T}(0, \infty)
\end{equation}
%
with standard deviation $\sigma_{g[\alpha]$ and 
truncated from the left to ensure that values remained positive. 
The function $g$ maps $\alpha$ to a given type of demographic process 
(reproduction, mortality, or dispersal),
such that a single random walk standard deviation was used 
for each type of demographic process.
While formulated as random walks, 
the realized sequences of inferred demographic rates 
were not truly random walks because they were constrained by fitting the model to data.
Therefore, the ``random walks'' are best understood as a convenient method 
for allowing the demographic rates to vary through time with implicit ``smoothing''
arising from the autocorrelated nature of the walks.

We fit the model in a Bayesian framework, 
using the abundance estimates from the trapping data. 
The likelihood of the ``observed'' abundance given the projected abundances 
and standard deviation $\sigma_y$ was calculated as
%
\begin{equation} \label{eq:likelihood}
\mathcal{L} = 
\displaystyle\prod_{h}
\displaystyle\prod_{i}
\displaystyle\prod_{t}
\text{Normal}
    \left(
        y_{h,i,t}~|~x_{h,i,t},~\sigma_y
    \right).
\end{equation}
%
For population densities that are necessarily non-negative, 
it is common to model the likelihood using a distribution that is similarly constrained,
such as a log-normal distribution. 
However, the multiplicative nature of population processes is already entailed 
in the population projection, 
and a log-normal likelihood reduces the relative contribution of large population sizes
that likely reflect meaningful dynamics.
Therefore, we opted for a normal (Gaussian) likelihood. 
Because the model was parameterized in a way that ensured
$x_{h,i,t}$ was non-negative,
the posterior distribution of $x_{h,i,t}$ was also guaranteed to be non-negative. 
We used gamma priors with shape parameter 1.5 and scale parameter 0.75
for the initial population size for each stage $\times$ basin combination, 
initial values for random walks,
and standard deviations for the random walks and likelihood.
A gamma distribution with shape parameter of 1.5 has zero density at zero 
and is concave down as it approaches its mode,
allowing the posterior to be arbitrarily close to zero 
while not being artificially drawn towards it.
This shape parameter, along with scale parameter of 0.75,
implies a mean of 2, 
which defines a reasonable scale for all of the parameters 
following the scaling of the population estimates (see below).

We fit the model using Stan 2.19 \citep{Carpenter2017}
run from R 4.0.0 with the \emph{rstan} package \citep{Stan2018}.
To facilitate selection of prior parameterizations,
we scaled the population estimates by dividing by the mean,
such that the resulting in data that had a mean of 1
and ranged from approximately 0 to 11.
We fit the model with 4 chains, 
15000 iterations (7500 of warm-up and 7500 of sampling),
and set the ``adapt-delta'' parameter to 0.9.
Convergence was assessed by the number of divergent transitions 
and the potential scale reduction factor (\^{R}),
which quantifies the relative variance within and between chains. 
We used posterior medians as point estimates
and quantiles (16\% and 84\%) as uncertainty intervals, 
with coverage analogous to standard errors.

To assess the extent to which the data supported statistically meaningful inferences
of temporal variation in demographic rates,
we compared the fit of the model to a reduced version will all demographic rates
fixed through time
(while still accounting for the unequal projection interval for transition probabilities).
We assessed goodness-of-fit using two metrics:
(1) the posterior median of the logarithm
of the likelihood given by equation \ref{eq:likelihood}
and (2) the ``Leave-one-out Information Criterion'' or LOOIC,
which is analogous to the Akaike Information Criterion (AIC)
and can be interpreted in a similar manner.
We calculated LOOIC using the \emph{loo} package \citep{loo}.
The full model had a much higher log-likelihood (posterior median: -230 vs. -378)
and much lower LOOIC (657 vs. 809) than the reduced model,
indicating that allowing temporal variation in demographic rates provided a
substantially improved fit to the data.



%========================================================================================

\subsection*{Annual dynamics and sensitivity analysis} 

%========================================================================================

While we parameterized the model in terms of seasonal projections
to accommodate the seasonal nature of the data,
we focused our analysis on the annual dynamics to better reflect
the annual nature of spawning and to circumvent interpretational
issues arising from the unequal projection intervals within a year.
Accordingly, we defined the annual projection matrix as
%
\begin{equation} \label{eq:A}
\mathbf{A}_y = \mathbf{P}_{t[y]+1} \mathbf{P}_{t[y]}
\end{equation}
%
for year $y$ and sequential time steps within that year $t[y]$ and $t[y]+1$,
with the year defined to start with the June/July census.
$\mathbf{A}_y$ projects the dynamics from
June/July of one year to June/July of the next year.
Because $\mathbf{P}_{t[y]$ was defined through June of 2020,
we only calculated $\mathbf{A}_y$ from 1991 through 2019.

We characterized the overall dynamics of the population in terms of the annual 
population growth rate $\lambda_y$, calculated as
%
\begin{equation} \label{eq:lam-n}
\lambda_y = \frac{N_{y+1}}{N_y} = 
              \frac{\mathbf{c}^\top \mathbf{A}_y \mathbf{x}_y}
                    {\mathbf{c}^\top \mathbf{x}_y}} 
\end{equation}
%
where $N_y$ is the summed abundance across basins and life stages in June/July of year $y$
and $\mathbf{c}$ is a 4 $\times$ 1 vector of ones.
Temporal variation in $\lambda_y$ reflects both variation in the demographic rates
and transient fluctuations due to non-equilibrium state distributions. 
Therefore, it is also informative to calculate 
the asymptotic population growth rate that 
would obtain under the equilibrium state distribution in a given time step, 
which is equal to real part of the leading eigenvalue of $\mathbf{A}_y$ 
\citep{caswell2001matrix}.

Both the transient and asymptotic population growth rates appeared to display periodic
behavior for at least a portion of the three decade time series.
We quantified this putative periodicity by applying continuous wavelet transforms to 
the time series for the transient and asymptotic growth rates, 
on a log-scale (the results were similar for the raw values) 
and with no detrending.
Wavelet transforms are a generalization of Fourier transforms,
allowing the decomposition of the signal into periodic elements to be localized in time. 
As our use of wavelet transforms was chiefly descriptive and applied to signals 
that were themselves the outputs of a statistical model,
we did not attempt to apply formal statistical inference (i.e. hypothesis testing)
to the wavelet decomposition.
We conducted the wavelet analysis using the R package \emph{WaveletComp},
and for tractability we applied the wavelet transform 
to the posterior median of $\lambda_y$ (rather than to multiple Markov chain samples).

Periodic population fluctuations are often attributable to density-dependent processes,
such as consumer-resource and host-parasite interactions.
To provide a general assessment of density dependence,
we regressed the log-transformed population growth rate ($\lambda_y$)
against overall population size ($N_y$) as inferred from the full demographic model.
The model was fit using the ``gls'' function from the ``nlme'' package in R,
inclcuding a lag-1 autocorrelation structure in the residuals,
although this had a limited effect on the overall inference.
We report this analysis for the asymptotic growth rate,
as classical mechanisms of density dependence processes 
arise through direct changes in the demographic rates, 
which in turn manifest in the asymptotic growth rate.
However, the results for the transient growth rate were similar. 
The regression was performed for 1000 Markov chain samples
to propagate uncertainty in $\lambda_y$ and $N_y$.

We conducted a sensitivity analysis to evaluate the effect of perturbations 
in the demographic rates on the population growth rate, 
using the approach of \cite{caswell2007sensitivity} 
that is applicable to transient dynamics. 
The sensitivity of the population growth rate with respect to a demographic parameter
quantifies how much the growth rate would change in response to a perturbation in the 
demographic parameter. 
In order to compare across parameters of different values 
(which is particularly relevant in the present context with time-varying rates),
it is common to calculate sensitivities 
in response to proportional (as opposed to additive) perturbations,
otherwise known as ``elasticities''.
For each year, we calculated the elasticity of the annual growth rate with respect
to the demographic rates at each of the two time steps within that year.
To simplify the presentation, 
we added together the two elasticities for a given demographic parameter 
(e.g. recruitment of south basin juveniles) for each year.
We conducted the sensitivity analysis for both transient and asymptotic growth rates.
Transient sensitivity analysis propagates perturbations in the demographic rates 
through time, 
such that transience due to non-equilibrium state distributions is attributed to the
demographic parameters resulting in the non-equilibrium state distribution 
for a given time step
\citep[][relevant equations are reproduced in Appendix III]{caswell2007sensitivity}.
The propagation interval for each perturbation was one year, 
such that we obtained a sequence of yearly elasticities 
for each seasonal demographic rate.
We obtained approximate asymptotic results by propagating each yearly perturbation
for 100 time steps before applying the next perturbation corresponding to the next year.
The sensitivity analysis was performed for 1000 samples of the Markov chain generated 
during fitting of the full demographic model to propagate uncertainty parameter estimates.





% ---------------------------------------------------------------------------------------
% ---------------------------------------------------------------------------------------
% Results
% ---------------------------------------------------------------------------------------
% ---------------------------------------------------------------------------------------

% ---------------------------------------------------------------------------------------
\section*{Results}
% ---------------------------------------------------------------------------------------



% ---------------------------------------------------------------------------------------
\subsection*{Model fit and seasonal dynamics} 
% ---------------------------------------------------------------------------------------

The M\'{y}vatn stickleback population fluctuated substantially over the past
three decades, 
and the model largely succeeded in characterizing these fluctuations 
(Figure \ref{fig:fit}).
This stands in contrast to the reduced model with temporally-fixed demographic rates,
which largely failed to capture the population fluctuations (\emph{Supplemental materials}).
Therefore, the population fluctuations cannot be purely explained 
in terms of transient oscillations due to non-equilibrium state structure, 
but rather arise at least in part from directly changes in the demographic rates. 
The dynamics of the adults in both basins were strongly synchronized,
as judged by the large covariance relative to the mean variance of the two time series.
The juvenile dynamics were less sychronized with each other,
and they were ``spikier'' than the adult fluctuations.
In general, juveniles were substantially more abundant in the north basin than in the south,
with the exception of occasional peaks. 
The high abundance of south basin juveniles in June of 1992 appears quite extreme
relative to the rest of the time series for both basins and life stages.
However, this pattern was observed across multiple traps and locations within the south basin,
suggesting that it reflected genuinely elevated abundance.

Per capita recruitment varied substantially across years and was generally of comparable
magnitude between the two basins (Figure \ref{fig:rec}),
although it was at most modestly synchronized between basins.
Similar to juvenile abundance, per capita recruitment in the south 
was characterized by relatively sudden and short-lived peaks, 
in comparison with the north basin.
Total recruitment corresponded very closely with juvenile abundance (Figure \ref{fig:fit}),
reflecting the model's inference 
that few juveniles return to the juvenile class at the next time step (see below).
As noted above, 
any cross-basin recruitment would be attributed 
to within-basin recruitment by the model. 
However, gravid females and males with breeding coloration are routinely sampled in the 
south basin, indicating that much of the recruitment within the south basin is 
likely attributable to spawning in the south basin.

Adult survival probabilities were strongly synchronized between basins 
and also appeared similar to the north basin juveniles (Figure \ref{fig:surv}),
with all three had two peaks in survival corresponding with 
similar peaks in observed abundance (Figure \ref{fig:fit}).
This covariation in survival probabilities was less pronounced in the second half 
of the time series, although still present for the north basin juveniles and adults.
Overall, this suggests that survival probability was affected by processes relevant across 
the life cycle and present at the scale of the entire lake.
In contrast to the other population states, 
survival probability of south basin juveniles was very low,
although still high enough to constitute a meaningful contribution over the one or two-month
``summer'' projection interval.
While the difference between the survival probability of south basin juveniles and 
the other states is quite pronounced, 
the basis for this inference is apparent 
in the population estimates (Figure \ref{fig:fit}). 
Large, sudden increases in the abundance of south basin juveniles appear to contribute
to elevated abundance of neither juveniles nor adults at the next time step,
which directly implies low survival.
The model inferred a high development rate from the juvenile to the adult stage,
with a probability of 92\% [88\%, 95\%] over a six-month period. 
Note that the model implicitly treats all individuals within a given state 
as demographically identical,
so this development probability can be interpreted as the average 
for individuals classified as juveniles, 
including individuals that are just below the size threshold for being classified as adults.
This high development probability reflects the apparent correlation between north basin
juveniles and adults with a lag of approximately one time step (Figure \ref{fig:fit}).
As noted above, we fixed the development probability through time and for both basins
to facilitate estimation of the other demographic parameters.

The model formulates dispersal reciprocally and temporal variation in the 
dispersal probabilites is inferred from the net exchenge of individuals between
the two sub-populations.
Consequently, dispersal probabilites for the two basins necessarily covary negatively 
(Figure \ref{fig:disp}).
Therefore, the dispersal probability is best understood as a measure of the extent 
to which the sub-populations appeared to be dynamically mixed
expressed on a per capita basis.
By this measure, the north and south basins appeared 
to be substantially coupled (Figure \ref{fig:disp}), 
which is consistent with the close correspondence 
between the adult dynamics across basins.
To clarify the contribution of dispersal to the dynamics, 
we calculated the net movement of individuals between basins,
with positive values indicating movement from south to north (``northward'')
and negative values indicating the converse (``southward'').
In each time step, 
net movement was generally low and slightly southward,
with a few of peaks in net movement corresponding to 
both peaks in abundance
and changes in the difference between north and south dispersal probabilities.
The cumulative movement over the entire time series was clearly southward,
resulting in a substantial net subsidization of the south basin over that period.



% ---------------------------------------------------------------------------------------
\subsection*{Annual dynamics and sensitivity analysis} 
% ---------------------------------------------------------------------------------------

The transient population growth rate ($\lambda$) fluctuated substantially between years
(Figure \ref{fig:lam}),
with a geometric mean of 0.98 [0.97, 0.99] implying a slight average decline
of $2\%~\text{year}^{-1}$ ($\lambda = 1$ would indicate no change).
Variance per se reduced the transient growth rate, 
as the arithmetic mean (1.19 [1.14, 1.24]) was much higher than the geometric mean
and would have implied substantial growth ($19\%~\text{year}^{-1}$)
if realized in the population.
The asymptotic growth rate was similar in this regard,
although the geometric mean (1.03 [1.00, 1.06]) was slightly higher 
than for the transient growth rate.
This is because the asymptotic growth rate lacked the extreme fluctuations 
of the transient growth rate,
leading to a lesser reduction relative to the arithmetic mean (1.20 [1.14, 1.27]).
These results show that variance arising from both changes in demographic
rates and transience per se reduced the the average population growth rate,
although the effect of the former predominated.

The wavelet decomposition detected a strong periodicity of around 6 years
in the asymptotic growth rate over the first two decades, 
which is clearly visible in the corresponding time series (Figure \ref{fig:lam}).
The wavelet decomposition of the transient growth rate was generally similar,
although the 6-year periodicity was weaker.
Furthermore, there was a highly localized periodicity of around 3 years 
at the beginning of the time series 
that coincided with the rapid increase and subsequent decline of south basin juveniles.
Together, these results indicate that the population displayed cyclic dynamics 
driven by changes in demographic rates.
However, these cycles were partially obscured by transient fluctuations arising 
from the non-equilibrium state distributions.
The absence of conspicuous periodicity in the final decade
for both the asymptotic and transient growth rates 
suggests a shift in the dynamics, 
although it is difficult to draw strong conclusions 
based on relatively short time series.

The asymptotic growth rate declined with total abundance
(slope = -0.13 [-0.17, -0.11]; Figure \ref{fig:dens}).
Futhermore, 
the intercept of the linear model implied that 
the asymptotic growth rate approached 1.77 [1.59, 2.02] 
as population size approached zero,
which was much higher than the ``observed'' geometric mean of 1.03.
The slope and intercept implied an equilibrium abundance (when $\lambda=1$)
of 4.25 [3.97, 4.52] relative abundance units,
about which there were substantial fluctuations in overall abundance 
(variation along the x-axis in Figure \ref{fig:dens}).
These results suggest that negative density dependence was relevant for the populaton
dynamics without tightly constaining the population to a fixed equilibrium,
which is consistent with the periodicity of the asymptotic growth rate.
While we did not attempt to formally assess whether the strength of density dependence
changed through time,
no such shift was visually apparant in Figure \ref{fig:dens}.
Despite the strong negative relationship between the growth rate and population size, 
there was substantial residual variation with modest
temporal autocorrelation (0.20 [0.08, 0.31]) associated with this linear trend.
This indicates that there were processes, either density dependent or independent,
relevant for the dynamics 
that were not fully captured in the linear model of the population growth rate.

The elasticities in Figure \ref{fig:elas} quantified the proportional change 
in the transient growth rate 
that would result from a proportional perturbation to a given demographic rate.
An elasticity of 1 indicates that a 1\% change in a demographic rate would result in
a 1\% in the population growth rate.
In general, the elasticities were positive,
which is expected as increases in processes such as survival or recruitment
should contribute positively to the growth rate.
However, the elasticity with respect to dispersal from north to south was negative,
which means that increases in southward dispersal 
tended to reduce the population growth rate.
This reflected unfavorable demographic conditions in the south basin,
chiefly attributable to low juvenile survival,
but also due to somewhat lower adult survival and inconsistent recruitment.
Counter-intuitively, the elasticity with respect to south basin recruitment was
slightly negative in some years. 
This was a transient phenomenon,
as demonstrated by the corresponding elasticities 
for the asymptotic growth rate that were positive
(X; the asymptotic results were otherwise largely the same as the transient ones).
The elasiticities with respect to 
demographic rates originating in the north basin were all of greater
average magnitude than those originating in the south,
indicating that that total population growth rate was most sensitive to processes
occurring in the north basin.
Among those north basin rates, juvenile survival and recruitment were associated with
the largest elasticities,
suggesting that the dynamics were most sensitive to processes early in the life cycle.
While the elasticities varied somewhat across years, 
these differences were modest relative to the differences among the demographic rate types.
Note that the temporal variation in the elasticity of development arose from
variation in the other demographic rates, 
as the development probability itself was fixed through time.

Together, the southward orientation of net dispersal and the negative elasticity 
of the population growth rate with respect to southward dispersal implied source-sink dynamics. 
To explore this further, 
we projected the population dynamics eliminating dispersal between basins, 
assuming that of the all individuals that would otherwise disperse remained within
their basin of origin.
This resulted in the rapid extinction of the south basin sub-population
and approximately exponential growth 
of the north basin sub-population (Figure \ref{fig:sink}).
Under natural conditions, 
altering the number of individuals in each basin 
would likely alter the other demographic rates.
For example, increased rention of individuals in the north basin in the absence of dispersal
could increase intraspecific competition and thereby lower recruitment or survival.
Therefore, the scenario described above is not intended to be a realistic projection, 
but rather an illustration of the direct effect of dispersal on the observed dynamics.
The key points are 
(a) that the south basin goes extinct and 
(b) that the north basin grows beyond what it would with southward dispersal,
given this hypothetical scenario.
Therefore, the model does in fact display source-sink dynamics,
with persistence in the south basin contingent on subsidies from the north.





% ---------------------------------------------------------------------------------------
% ---------------------------------------------------------------------------------------
% Discussion
% ---------------------------------------------------------------------------------------
% ---------------------------------------------------------------------------------------



\section*{Discussion}

We used a stage-structured metapopulation model 
with time-varying demographic rates to characterize the 
spatiotemporal dynamics of threespine stickleback 
in the shallow and productive lake M\'{y}vatn.
From the model, we were able to draw the following inferences regarding the 
dynamics of the population:
%
\begin{enumerate}[label=(\alph*)]
\item
The stickleback population had cyclic dynamics with a period of approximately 6 years.
Furthermore, the population growth rate exhibited negative density dependence,
which is often associated with population cycles.
However, transient fluctuations due to non-equilibrium state distributions
partially obscured the periodicity,
and there may have been a weakening of the periodicity in the final decade.
%
\item
The population was characterized by source-sink dynamics, 
with the north basin subsidizing the south basin and driving the overall dynamics. 
Demographically unfavorable conditions in the south basin were 
primarily due to low juvenile survival.
%
\item
In addition to the spatial coupling between basins,
several within-basin demographic rates were strongly correlated 
between the two basins and 
were associated with temporal variation in the asymptotic population growth rate.
This indicates that processes at the whole-lake scale contributed 
to the population dynamics.
\end{enumerate}
%
Together, these inferences illustrate how synchronization of cyclic fluctuations 
can arise through both spatial coupling and synchronization of demographic rates.
Furthermore, this study illustrates the value of spatially explicit demographic models 
with time-varying parameters for disentangling the contribution of different processes,
including transience per se, to spatiotemporal population dynamics.

Before discussing the further implications of our analysis, 
it is worth noting some of its limitations. 
Demographic analyses are often conducted using mark-recapture type methods,
allowing relatively direct inferences of demographic rates
\citep{lebreton1992, fujiwara2002}.
In the absence of such data, 
we instead parameterized our model by fitting it to a time-series of abundance estimates.
Uniquely identifying demographic processes 
contributing to observed population change is a long-standing challenge 
\citep{wood1994, twombly1994},
and required some simplifying assumptions.
Specifically, 
we assumed that juvenile development rate was fixed through time and between basins,
that only adults dispersed within a given time step,
and that recruits born within a given basin remained within that basin until the next time step.
While acknowledging the uncertainties,
we feel that these constraints on the model are reasonable for 
threespine stickleback in M\'{y}vatn,
as in other systems \citep{yurtseva2019}.
The inferred development rates were sufficiently high that most juveniles would transition 
to adulthood in a given time step,
which is consistent with other data suggesting that M\'{y}vatn stickleback
reach adult size within one year (unpublished data).
This high transition rate of juveniles
limits the relevance of both temporal variation in development and juvenile dispersal,
as most individuals would be come adults and be able to disperse within that time step 
regardless.
While we cannot rule out cross-basin recruitment, 
it seems that spawning does in fact occur within both basins given the ubiquity 
of breeding-condition adults, 
as opposed to all spawning occurring within one basin.
Furthermore, any cross-basin recruitment within a given time step 
is implicitly characterized by the model as additional temporal variation in
per capita recruitment.
Finally, both per capita and total recruitment had low covariance between basins relative
to the overall variance, which is not what would be expected in the presence of 
the homogenizing effect of cross-basin exchange. 

M\'{y}vatn's stickleback population fluctuated substantially over the past three decades,
with the annual population growth rates spanning an order of magnitude and including
periods of both substantial growth and decline.
Furthermore, these fluctuations were substantially synchronized between basins,
due to both spatial coupling via dispersal
and correlated survival between basins. 
Therefore, these fluctuations can be understood as a feature of the lake-wide dynamics
and were likely driven by processes occurring at the whole-lake scale.
The population dynamics were cyclic over most of the time series,
with a period of approximately 6 years. 
This periodicity is not obviously tied to the life history of threespine stickleback,
which likely live on the order of 3 or 4 years in M\'{y}vatn.
Previously,
\cite{wootton2005} compared the dynamics of 3 threespine stickleback populations 
on the island of Great Britain (one riverine, one lacustrine, and one backwater)
and found that the backwater population had cyclic dynamics of approximately
6 years, while the other two lacked any consistent periodicity.
Therefore, the 6-year periodicity of M\'{y}vatn's population is not without precedent,
although it is also not a ubiquitous feature of stickleback populations.
Cyclic dynamics are often associated with negative density-dependence, 
either direct when there are appropriate time lags \citep{may1974}
or indirect when mediated through interspecific interactions 
\citep{nicholson1935, rosenzweig1963}.
Indeed, the asymptotic growth rate of the M\'{y}vatn population declined 
with total population density,
with a 25\% increase above the implied equilibrium corresponding to a 13\% 
reduction in the  growth rate.
Despite its consistency for the first two decades, 
the periodicity of the population growth rate weakened in the final decade, 
with perhaps a hint of reestablishment in the last few years.
This could suggest a change in the underlying dynamics of the system
\citep{carpenter2011},
which has been observed for other populations with cyclic dynamics
\citep{ims2008collapsing}.
However, caution is warranted when drawing such inferences 
from relatively short time series.

Transience was an important feature of the population dynamics,
with the periodicity that was clear in the asymptotic growth rate manifesting 
at most very subtly in the transient growth rate.
Furthermore, transience per se reduced the geometric by 5\%,
although this effect was minor relative to the effect of asymptotic fluctuations
(on the order of 20\%). 
The most conspicuous transient event was the dramatic increase in recruitment of south basin juveniles in 1992, followed by a similarly dramatic decrease.
This event was observed across multiple traps and stations within the south basin,
suggesting that it was a real feature of the dynamics, rather than an error in the data.
While this recruitment event contributed little to the future dynamics, and was
arguably unimportant from the perspective of the stickleback population,
it may well have had a substantial influence on other components of the ecosystem.
Indeed, predation by threespine sticklebacks alters the 
abumndance and community composition of 
zooplankton and phytoplankton in M\'{y}vatn \citep{ersoy2017}
and other systems \citep{harmon2009}.
Ecologists are increasing aware of the importance of ``extreme'' events and the 
processes that bring them about \citep{anderson2017, batt2017}.
Juvenile recruitment in fish as previously been identified as highly episodic,
due to nonlinear effects of multiple environmental drivers \citep{dixon1999}.
In the case of M\'{y}vatn's stickleback, 
per capita recruitment in 1992 was high 
but not nearly so extreme relative to other years as was case for total recruitment.
Therefore, the extreme total recruitment was not simply due to a year of particularly
high recruitment per adult, 
but also due to the disproportionate abundance of south basin adults 
relative the equilibrium state distribution.
This underscores the potential role of demographic transience 
in facilitating extreme population events.

The dynamics of the north and south basin populations were synchronized,
particularly for the adults. 
According to the model, 
this was due to both spatial coupling via dispersal 
and high covariance in certain demographic rates such as adult survival.
Beyond the synchronization, 
the two sub-populations displayed source-sink dynamics \citep{pulliam1988},
with the north basin substantially subsidizing the south.
Furthermore,
the lake-wide population growth rate was most sensitive to demographic processes 
within the north basin,
indicating that the north basin played a primary role in dictating the dynamics 
of the entire population.
However, this does not mean that the southern sub-population failed to contribute 
meaningfully to the dynamics.
On the contrary,
in some years the abundance of south basin juveniles and adults was quite high relative
to the north basin counterparts 
and therefore contributed meaningfully to the total abundance in those years.
Furthermore, it is possible that the absence of net subsidies from the north basin
could be compensated for by elevated demographic rates via reduced intraspecific competition.
Nonetheless, the role of the north basin in the overall dynamics was still dominant.
The relatively limited demographic contribution from the south basin was primarily due
to the very low survival of its juveniles. 
This low juvenile survival is directly implied by the population estimates derived 
from the trapping data,
as south basin juveniles are generally rare with the exception of sporadic peaks that
do not appear to contribute the the abundance of any demographic states in future time steps.
The ecological bases for low survival of south basin juveniles,
as well as temporal variation in all of the demographic rates,
are largely unknown.
However, other populations in M\'{y}vatn have dramatic fluctuations that appear to be 
driven by processes endogenous to the lake
\citep{einarsson2002, ives2008, einarsson2016}. 
Therefore, it is plausible that the same is true for the threespine stickleback.

Our analysis illustrates the potential for both dispersal-mediated spatial coupling 
and synchronized demographic rates to generate synchronized fluctuations in a wild population.
Furthermore, we show that transience can alter the character of such dynamics,
partially obscuring the periodicity that is underpinned by the synchronized temporal
variation in the demographic rates.
Central to this effort was explicitly modeling the demographic structure of the population
and temporal variation in the demographic rates 
that manifested sychronized processes occuring within both sup-populations.
While this study focused on a single population,
the dynamic processes and the methods used to analyze them are likely to relevant for 
other spatially-structured population with cyclic fluctuations.




% ---------------------------------------------------------------------------------------
% ---------------------------------------------------------------------------------------
% References,figures, etc.
% ---------------------------------------------------------------------------------------
% ---------------------------------------------------------------------------------------

\bibliographystyle{ecology.bst}
\clearpage

\bibliography{refs.bib}

% ---------------------------------------------------------------------------------------
\clearpage
\begin{figure}
\centering
\includegraphics{../figures/figs/fig_fit.pdf}
\caption{\label{fig:fit}
Structured population estimates (points) and 
fit from the structured population model 
with time-varying demgoraphic rates (lines). 
The shaded regions correspond to Bayesian standard errors.
Note that the intervals used to project the model were unequal,
matching the structure of the data.
These unequal projection intervals are responsible for the "jagged" oscillations
that are occassionally visible in the model fit.
The covariances and mean variances are calcualted across basins for a given life-stage
and serve as a measure of synchrony between basins.
}
\end{figure}
\clearpage
% ---------------------------------------------------------------------------------------

% ---------------------------------------------------------------------------------------
\clearpage
\begin{figure}
\centering
\includegraphics{../figures/figs/fig_rec.pdf}
\caption{\label{fig:rec}
Per capita (top panel) and total (bottom panel) recruitment, 
inferred from the structured population model.
The covariances and mean variances are calcualted across basins
and serve as a measure of synchrony between basins.
Uncertainty intervals are omitted for clarity.
}
\end{figure}
\clearpage
% ---------------------------------------------------------------------------------------

% ---------------------------------------------------------------------------------------
\clearpage
\begin{figure}
\centering
\includegraphics{../figures/figs/fig_surv.pdf}
\caption{\label{fig:surv}
Survival probabilities inferred from the structured population model.
Thin lines show the survival probabilities for the projection intervals
used to fit the model, 
which typically oscillate between 2 and 10 months.
The thick lines standardize the survival probabilities to the mean 
projection interval duration of 6 months to highlight the temporal patterns.
Note that survival probabilities converge to zero as the projection interval increases,
so the fact that 6-month survival of south basin juveniles is near zero does not 
imply that there were never contributions of south basin juveniles to subsequent time steps
(as can be seen from the thin lines).
The covariances and mean variances are calcualted 
for the 6-month standardize rates
across basins for a given life-stage
and serve as a measure of synchrony between basins.
Uncertainty intervals are omitted for clarity.
}
\end{figure}
\clearpage
% ---------------------------------------------------------------------------------------

% ---------------------------------------------------------------------------------------
\clearpage
\begin{figure}
\centering
\includegraphics{../figures/figs/fig_disp.pdf}
\caption{\label{fig:disp}
Dispersal probabilities (top panel) and net movement (bottom panel)
inferred from the structured population model.
In the top panel,
thin lines show the survival probabilities for the projection intervals
used to fit the model, 
which typically oscillate between 2 and 10 months.
The thick lines standardize the survival probabilities to the mean 
projection interval duration of 6 months to highlight the temporal patterns.
The covariance and mean variance are calcualted 
for the 6-month standardize rates
across basins
and serve as a measure of synchrony between basins.
In the bottom panel,
net movement is defined such that positive values indicate movement from
south to north.
The solid line indicates the net movement at each time step,
while the dotted line indicates cumulative net movement from the 
beginning of the time series up to a given time step.
Uncertainty intervals are omitted for clarity.
}
\end{figure}
\clearpage
% ---------------------------------------------------------------------------------------

% ---------------------------------------------------------------------------------------
\clearpage
\begin{figure}
\centering
\includegraphics{../figures/figs/fig_lam.pdf}
\caption{\label{fig:lam}
Time series (top row) and periodigrams from wavelet transforms (bottom row) 
of the transient (first column) and asymptotic (second column) population growth rates.
In the top row, shaded regions correspond to Bayesian standard errors.
The dashed horizontal line shows $\lambda=1$, 
which corresponds to no change in the population size.
In the periodigrams, red indicates high signal while blue indicates low signal. 
The black contour lines are based on signal quantiles, and denote regions of high signal.
While the wavelet decomposition was conducted for periods up to the maximum period length
(29 years), the signal associated with periods >10 years was very weak. 
So, for clarity the figure truncates the periodigram at 11 years.
}
\end{figure}
\clearpage
% ---------------------------------------------------------------------------------------

% ---------------------------------------------------------------------------------------
\clearpage
\begin{figure}
\centering
\includegraphics{../figures/figs/fig_dens.pdf}
\caption{\label{fig:dens}
Asymptotic growth rate plotted against total population size (points) as 
inferred from the structured population model.
The solid black line shows the fitted values from a linear regression with a lag-1
residual autocorrelation structure, 
which accounted for uncertainty in both the growth rate and population size.
The error bars (vertical and horizontal)
and the shaded region associated with the regression line 
correspond to Bayesian standard errors.
The dashed horizontal line shows $\lambda=1$, 
which corresponds to no change in the population size.
The dashed vertical line shows the intersection of the regression line with $\lambda=1$,
which corresponds with the implied equilibrium population size.
}
\end{figure}
\clearpage
% ---------------------------------------------------------------------------------------

% ---------------------------------------------------------------------------------------
\clearpage
\begin{figure}
\centering
\includegraphics{../figures/figs/fig_elas.pdf}
\caption{\label{fig:elas}
Elasticity analysis of the transient population growth rate 
with respect to the time-varying demographic rates.
The error bars correspond to Bayesian standard errors.
The black vertical lines show time-averages of the posterior medians 
for each demographic parameter to aid visualization.
}
\end{figure}
\clearpage
% ---------------------------------------------------------------------------------------

% ---------------------------------------------------------------------------------------
\clearpage
\begin{figure}
\centering
\includegraphics{../figures/figs/fig_sink.pdf}
\caption{\label{fig:sink}
Projected population dynamics with no dispersal,
assuming all individuals that would otherwise disperse remain 
in their basin of origin.
Note that under natural conditions the absence of dispersal 
would likely increase density dependence in other demographic rates.
So, this projection not intended to be realistic, 
but rather as an illustration of the source-sink nature of the dynamics.
The shaded regions correspond to Bayesian standard errors.
}
\end{figure}
\clearpage
% ---------------------------------------------------------------------------------------

\renewcommand{\thefigure}{A\arabic{figure}}
\renewcommand{\theequation}{A\arabic{equation}}
\renewcommand{\thetable}{A\arabic{table}}
\setcounter{equation}{0}
\setcounter{figure}{0}
\setcounter{table}{0}

% ---------------------------------------------------------------------------------------
\section*{Appendix I: N-mixture model} 
% ---------------------------------------------------------------------------------------

To account for potential variation in trapping probability between size classes,
locations, and time of day, 
we used a modified N-mixture model to estimate relative basin-level population densities
for the two size classes.
The probability of trapping some number of individuals $y_i$ 
was modeled as
%
\begin{equation}
  y_i \sim \text{Binomial}\left(\eta_{g_{\eta}[i]}, ~\nu_{g_{\nu}[i]}\right)
\end{equation}
%
where $\eta_{g_{\eta}[i]}$ is the detection probability 
and $\nu_{g_{\nu}[i]}$ is the population density for a given 
station-size-date combination for the $i$th observation.
The density is in units of individuals per station, 
with each station characterizing a sampling area that is 
taken to be the same size for all stations.
The functions ${g_{\eta}[i]}$ and ${g_{\nu}[i]}$ map observations to the 
appropriate grouping. 
The discrete population density for each station-size-date combination was modeled as
%
\begin{equation}
  \nu_{g_{\nu}[i]} \sim \text{Poisson}\left(\kappa_{g_{\kappa}[i]}\right)
\end{equation}
%
where $\kappa_{g_{\kappa}[i]}$ is the mean population density across stations 
within a basin-size-date combination. 
Variation in $\kappa_{g_{\kappa}[i]}$ across basin, size, 
and date was characterized as 
%
\begin{equation}
  \kappa_{g_{\kappa}[i]} \sim 
    \text{Exponential}\left(\zeta \right)
\end{equation}
%
with rate parameter $\zeta = 0.001$ (implying a mean of 1000).
The detection probability for the $i$th observation was modeled using a 
``logistic regression''-style approach:
%
\begin{equation}
  \eta_{g_{\eta}[i]} = 
    \text{logit}^{-1}\left(\mathbf{z}_{g_{\eta}[i]}^\top~{\boldsymbol\beta}\right)
\end{equation}
%
\noindent where $\mathbf{z}_{g_{\eta}[i]}^\text{T}$ is a transposed vector 
of predictor values for the $i$th observation
(including 1 in the first column for the overall intercept)
and $\boldsymbol\beta$ is a vector of coefficients. 
We included main effects for station, trapping time (day vs. night), size class,
and the size class $\times$ trapping time interaction.
We used a Gaussian prior with mean of 0 and standard deviation of 2 
for the coefficients $\boldsymbol\beta$.
The model was fit using JAGS via \emph{runjags} in R4.0.0 with,
using 4 chains and 15000 iterations (5000 adaptation, 5000 burn-in, and 5000 sampling).
Convergence was assessed by the potential scale reduction factor (\^{R}),
which quantifies the relative variance within and between chains. 





% ---------------------------------------------------------------------------------------
% ---------------------------------------------------------------------------------------
% Appendix
% ---------------------------------------------------------------------------------------
% ---------------------------------------------------------------------------------------


% ---------------------------------------------------------------------------------------
\section*{Appendix II: Transition rates} 
% ---------------------------------------------------------------------------------------

We parameterized transition rate matrices for mortality ($\boldsymbol\Omega^{\mu}_t$), 
development ($\boldsymbol\Omega^{\gamma}$), 
and dispersal ($\boldsymbol\Omega^{\delta}_t$) as:
 %
\begin{equation} \label{eq:Theta}
\begin{aligned}
\boldsymbol\Omega^{\mu}_t & = 
\left[
\begin{array}{cc|cc}
    -\omega^{\phi_{j,s}}_t & 0 & 0 & 0 \\
    0 & -\omega^{\phi_{a,s}}_t & 0 & 0 \\
    \hline
    0 & 0 & -\omega^{\phi_{j,n}}_t & 0 \\
    0 & 0 & 0 & -\omega^{\phi_{a,n}}_t \\
    \end{array}
\right] \\
\boldsymbol\Omega^{\gamma} & = 
\left[
\begin{array}{cc|cc}
    -\omega^{\gamma_{j}} & 0 & 0 & 0 \\
    \omega^{\gamma_{j}}  & 0 & 0 & 0 \\
    \hline
    0 & 0 & -\omega^{\gamma_{j}} & 0 \\
    0 & 0 & \omega^{\gamma_{j}}  & 0 \\
    \end{array}
\right] \\
\boldsymbol\Omega^{\delta}_t & = 
\left[
\begin{array}{cc|cc}
    0 & 0 & 0 & 0 \\
    0 & -\omega^{\delta_{a,s}}_t & 0 & \omega^{\delta_{a,n}}_t \\
    \hline
    0 & 0 & 0 & 0 \\
    0 & \omega^{\delta_{a,s}}_t & 0 & -\omega^{\delta_{a,n}}_t \\
    \end{array}
\right]
\end{aligned}
\end{equation}
%
Note that mortality implicitly entails transition to a ``death state'' that is omitted 
for succinctness, as dead individuals do not contribute to future transitions.
For each transition matrix $\boldsymbol\Omega^{\alpha}_t$, 
we then calculated the probability of transitioning as
%
\begin{equation} \label{eq:Psi}
\boldsymbol\Psi^{\alpha}_t = e^{\Delta T\boldsymbol\Omega^{\alpha}_t}
\end{equation}
%
which is the solution to the differential equation associated with the Markov process
specified by $\boldsymbol\Omega^{\alpha}_t$ 
when projected over interval $\Delta T$ 
and initial condition equal to the 4 $\times$ 4 identity matrix.
The elements of $\boldsymbol\Psi^{\alpha}_t$ were then used to parameterize
the matrices given by equations \ref{eq:W} and \ref{eq:B}.
In principle, we could have included all of the demographic transitions in a single
transition matrix. 
However, modeling the different transition processes separately facilitated interpretation
of the resulting transition probabilities, as they would only pertain to a single type 
of demographic transition rather than multiple occurring simultaneously.
This was also computationally advantageous during model fitting, 
for much the same reasons.



% ---------------------------------------------------------------------------------------
\section*{Appendix III: Sensitivity analysis} 
% ---------------------------------------------------------------------------------------

We used the method of \cite{caswell2007sensitivity} to calculate the elasticities 
(proportional sensitivities) of the annual transient population growth rate $\lambda_y$
with respect to perturbations in the seasonal demographic rates.
It was convenient to perform the calculations using the logarithm of $\lambda_y$,
commonly denoted $r_y$.
This parameter is related to total population size $N_y$ by the expression
%
\begin{equation} \label{eq:r}
r_y = \text{log}\left(N_{y+1}\right) - \text{log}\left({N_y}\right).
\end{equation}
%
Note that
%
\begin{equation} \label{eq:lsens}
\frac{\text{d}\lambda_y}{\text{d}\theta} = \lambda \frac{\text{d}r_y}{\text{d}\theta}
\end{equation}
%
where $\frac{\text{d}\lambda_y}{\text{d}\theta}$ can generically be interpreted
as the sensitivity of $\lambda$ with respect to perturbations 
in a single parameter $\theta$.
The elasticity of $\lambda$ is then defined as
%
\begin{equation} \label{eq:lelas}
\frac{\theta}{\lambda_y} \frac{\text{d}\lambda_y}{\text{d}\theta} = 
        \theta\frac{\text{d}r_y}{\text{d}\theta}.
\end{equation}
%
The multiplication of $\frac{\text{d}r_y}{\text{d}\theta}$ by $\theta$  
implies proportional perturbations in $\theta$.
Therefore, the sensitivity of $r_y$ with respect to proportional perturbations in $\theta$
equals the elasticity of $\lambda_y$.
This deduction is essentially a restatement of logarithmic relationship of $\lambda_y$
and $r_y$, along with the properties of logarithmic derivatives.

The transient sensitivity of $r_y$ with respect to perturbations in demographic parameters
is defined as
%
\begin{equation} \label{eq:dr}
\frac{\text{d}r_y}{\text{d}\boldsymbol\theta_y^\top} = 
    \frac{\mathbf{c}^\top}{N_{y+1}} \frac{\text{d}\mathbf{x}_{y+1}}
            {\text{d}\boldsymbol\theta_{y+1}^\top}-
        \frac{\mathbf{c}^\top}{N_{y}} \frac{\text{d}\mathbf{x}_y}
            {\text{d}\boldsymbol\theta_y^\top}
\end{equation}
%
where $\boldsymbol\theta_y$ is a vector of demographic parameters, 
$\mathbf{x}_y$ is a 4 $\times$ 1 vector of abundances in each state,
$\mathbf{c}$ is a 4 $\times$ 1 vector of ones,
and ``$\text{d}$'' is the derivative operator.
We were interested in the sensitivity of $r_y$ with respect to proportional 
perturbations in the seasonal demographic rates,
which are connected to $\mathbf{x}_y$ through the annual population projection matrix
$\mathbf{A}_y$ as defined in equation \ref{eq:A}.
If $\boldsymbol\theta_y$ contains the seasonal demographic rates 
(i.e., the collective elements of $\mathbf{P}_{t[y]}$ and $\mathbf{P}_{t[y]+1}$)
and $\boldsymbol\epsilon_y$ is a vector of proportional perturbations in
$\boldsymbol\theta_y$,
then 
%
\begin{equation} \label{eq:dx}
\frac{\text{d}\mathbf{x}_{y+1}}{\text{d}\boldsymbol\theta_{y+1}^\top} = 
    \mathbf{A}_y \frac{\text{d}\mathbf{x}_{y}}{\text{d}\boldsymbol\theta_y^\top}+
        \left(\mathbf{x}_{y}^\top \otimes \mathbf{I}_c \right)
            \frac{\text{dvec}\mathbf{A}_y}{\text{d}\boldsymbol\epsilon_y^\top}
                \text{diag}\boldsymbol\epsilon_y
\end{equation}
%
where $\mathbf{I}_c$ is the $c \times c$ identity matrix with $c$ as the length
of the parameter vector $\theta_y$,
``$\text{vec}$'' is an operator that creates a vector by stacking columns of the operand matrix,
and ``$\text{diag}$'' is an operator that creates a square matrix with the operand vector on
the diagonal and zeros elsewhere. 
Defining an initial population size distribution $\mathbf{x}_0$ 
that is independent of the demographic parameters implies that 
$\frac{\text{d}\mathbf{x}_0}{\text{d}\boldsymbol\theta_0^\top} = \mathbf{0}$.
Using this initial condition,
the sensitivities can then be calculated by iterating equations \ref{eq:dr} and \ref{eq:dx}
for each year, with perturbations $\boldsymbol\epsilon_y$ proportional (or equal)
to the parameter vector $\boldsymbol\theta_y$.
Asymptotic results can be obtained by iterating \ref{eq:dx} many times for a given year,
which eliminates the dependence on the initial values such that each year can be treated
independently. 
The derivatives of the elements of $\mathbf{A}_y$ with respect to the seasonal
demographic rates necessary for evaluating \ref{eq:dx} can be computed analytically
and are shown in the Supplmental Materials.




\end{document}


\documentclass[11pt]{article}
% \usepackage[sc]{mathpazo} % Like Palatino with extensive math support
\usepackage[letterpaper, margin=1in]{geometry}
% \usepackage{mathptmx} % Like Times New Roman
\usepackage{newtxtext,newtxmath}
\usepackage[math-style=ISO]{unicode-math}
\usepackage{fullpage}
\usepackage[authoryear,sectionbib,sort]{natbib}

\linespread{1.7}

\usepackage[utf8]{inputenc}
\usepackage{lineno}
\usepackage{titlesec}
\titleformat{\section}[block]{\Large\bfseries\filcenter}{\thesection}{1em}{}
\titleformat{\subsection}[block]{\Large\itshape\filcenter}{\thesubsection}{1em}{}
\titleformat{\subsubsection}[block]{\large\itshape}{\thesubsubsection}{1em}{}
\titleformat{\paragraph}[runin]{\itshape}{\theparagraph}{1em}{}[. ]\renewcommand{\refname}{Literature Cited}

% For Icelandic ð symbol:
\DeclareTextSymbolDefault{\dh}{T1}
% Increased spacing in math mode:
\medmuskip=8mu % by default it is equal to 4 mu
\thickmuskip=10mu % by default it is equal to 5 mu

% Figures
\usepackage{graphicx}
% Table
\usepackage{booktabs}


%%%%%%%%%%%%%%%%%%%%%
% Line numbering
%%%%%%%%%%%%%%%%%%%%%
%
% Please use line numbering with your initial submission and
% subsequent revisions. After acceptance, please turn line numbering
% off by adding percent signs to the lines %\usepackage{lineno} and
% to %\linenumbers{} and %\modulolinenumbers[3] below.
%
% To avoid line numbering being thrown off around math environments,
% the math environments have to be wrapped using
% \begin{linenomath*} and \end{linenomath*}
%
% (Thanks to Vlastimil Krivan for pointing this out to us!)

\title{Transient dynamics of a spatially-structured population with time-varying demographic rates}



% This version of the LaTeX template was last updated on
% November 8, 2019.

%%%%%%%%%%%%%%%%%%%%%
% Authorship
%%%%%%%%%%%%%%%%%%%%%
% Please remove authorship information while your paper is under review,
% unless you wish to waive your anonymity under double-blind review. You
% will need to add this information back in to your final files after
% acceptance.

% \author{
% Joseph S. Phillips$^{1,2, \dagger}$ \\
% Kasha, Bjarni, Katja, Arni, Tony...}



\usepackage{amsmath} % for split math environment


\date{}

\begin{document}

\raggedright
\setlength\parindent{0.25in}

\maketitle


% \noindent{} 1. Department of Aquaculture and Fish Biology, H\'{o}lar University, Skagafj\"{o}r{\dh}ur 551 Iceland
% 
% \noindent{} 2. Department of Integrative Biology, University of Wisconsin, Madison, Wisconsin 53706 USA
% 
% \noindent{} $\dagger$ E-mail: joseph@holar.is



\bigskip

Running head: {}

% \textit{Manuscript elements}: %Figure~1, figure~2, table~1, online appendices~A and B (including $-- figure~A1 and figure~A2). Figure~2 is to print in color.

\linenumbers{}
% \modulolinenumbers[3]

\clearpage

% ---------------------------------------------------------------------------------------
% ---------------------------------------------------------------------------------------
% Abstract
% ---------------------------------------------------------------------------------------
% ---------------------------------------------------------------------------------------


\section*{Abstract} \label{abstract}

Population fluctuations are of enduring interest in ecology,
particularly when they are spatially synchronized across 
multiple subpopulations.
Disentangling the diverse mechanisms underpinning spatially-structured fluctuations
is complicated by transient phenomena arising from non-equilibrium population structure.
In this study, we used a multiregional stage-structured model
with to characterize the  fluctuations of two spatially-coupled subpopulations
(north and south) of threespine stickleback in M\'{y}vatn, Iceland.
By fitting this model to a 29-year times series of structured abundance,
we were able to quantify temporal variation in demographic rates
underpinning the large population fluctuations.
According to the model, the fluctuations were cyclic
at the whole-lake scale with a period of approximately 6 years.
These fluctuations were similar in the two subpopulations 
and were underpinned by variation in demographic rates 
that were similarly correlated through space.
However, the population also displayed source-sink dynamics,
with the subsides from north required to sustain the subpopulation the south.
Furthermore, the cyclic nature of the fluctuations was partially obscured by transience
due to non-equilibrium population structure,
underscoring the value of explicit models of structured population dynamics.
Our results illustrate how both synchronized demographic rates and spatial coupling
through dispersal influence the character of spatially-structured population dynamics.



\bigskip

\textit{Keywords}: {}





% ---------------------------------------------------------------------------------------
% ---------------------------------------------------------------------------------------
% Introduction
% ---------------------------------------------------------------------------------------
% ---------------------------------------------------------------------------------------

\section*{Introduction} \label{introduction}

Population fluctuations are of enduring interest in ecology,
with particular attention paid to periodic or quasi-periodic dynamics
\citep{elton1924, myers2018}.
Various mechanisms have been posited for cyclic fluctuations,
including periodic environmental variation and 
interspecific interactions (e.g., predator-prey; host-parasitoid)
that induce periodicity endogenously 
\citep{nicholson1935, andrewartha1954, rosenzweig1963}. 
Furthermore, exogenous and endogenous factors can interact with each other,
with the potential to produce a wide array of dynamics with
varying degrees of periodicity, as well as more exotic features
\citep{bjornstad2001, turchin2003, ives2008}.

Many populations are spatially structured \citep{hanski1998},
setting the stage for spatially-structured fluctuations 
\citep{bjornstad1999spatial, gouveia2016}.
Indeed, some populations are synchronized across wide geographic extents,
which raises the question of how such synchrony arises
\citep{ranta1995synchrony, krebs2002}.
Perhaps the most widely discusssed mechanism is the Moran effect,
whereby spatial synchrony in some exogenous driver induces synchrony
in the populations under its influence
\citep{moran1953}. 
Spatial coupling through dispersal between sub-populations of a given species
\citep{liebhold2004, goldwyn2008}
or movement of another species with which the focal sub-populations interact
\citep{gilg2009, ims2000}
is another important mechanism of synchronous fluctuations.
Synchrony in exogenous drivers and spatial coupling through dispersal 
can interact with each other and with the internal dynamics of a given population,
complicating the way in which these factors map onto population synchrony
\citep{ranta1995synchrony, kendall2000dispersal, abbott2011}.

Both the Moran effect and spatial coupling can be characterized as demographic processes,
with the former inducing synchrony in demographic rates such as recruitment or survival
and the latter arising from immigration between sub-populations.
From this framing,
a third source of population fluctuations (potentially synchronous) becomes clear:
trasient oscillations due to non-equilibrium distribution of individuals across
demographic states
\citep{caswell2001matrix, koons2017understanding}.
In age- and stage-structured populations, 
such transient oscillations can be very large and alter the long-term dynamics 
through various mechanisms,
such as ``transient amplification''
\citep{neubert1997, tenhumberg2009}. 
In spatially-structured populations, 
discerete patches play an analagous role to discrete life-stages 
and can similarly exhibit non-equilibrium distributions with associated transience
\citep{ozgul2009}.
When demographic rates vary through time,
as is inherent to the Moran effect,
a population may persistently occupy non-equilibrium state distributions,
which in turn means that transient oscillations may be ubiquitous features of the dynamics.
Therefore, accounting for effects transience may be important for 
fully characterizing the nature and causes of spatially-structured fluctuations
\citep{hastings2010}.

In this study, 
we use a population-dynamics model to analyze spatially-structured fluctuations 
in a population of threespine stickleback (\emph{Gasterosteus aculeatus})
in Lake M\'{y}vatn, Iceland.
M\'{y}vatn is particularly well suited to this exploration for two primary reasons.
First, the stickleback population is spatially structured by the unique geomorphology
of the lake \citep{gislason1998, millet2013}, 
which is divided into two basins connected by two narrow channels.
The southern basin (Syðrifloi) is the larger of the two ($28~\text{m}^2$) and is dominated
by exposed sediment and intermittent mats of filamentous green algae
\citep{einarsson2004myvatn}.
In contrast, the northern basin (Ytrifloi) is substantially smaller ($9~\text{m}^2$)
and more spatially heterogeneous, 
in part due to dredging of the lake bottom that substantially altered its bathymetry.
Despite its smaller area, the north basin has historically sustained much higher
densities of threespine stickleback  \citep{gislason1998}, 
likely owing to the ecological differences between the basins. 

Second, M\'{y}vatn's stickleback population fluctuates substantially through time.
While the causes of these fluctuations are unknown, 
they are likely connected to the large temporal variability 
of other populations in the lake.
M\'{y}vatn is naturally eutrophic due to the inflows of nutrient-rich springs,
which sets the stage for high-amplitude fluctuations in secondary producers
\citep{einarsson2004myvatn}.
Chief among these are chironomid midges and cladocerans
\citep{einarsson2002, einarsson2004clad, gardarsson2004population, ives2008},
both of which are potential food sources for threespine stickleback.
Furthermore, the lake hosts temporally variable populations 
of arctic charr (\emph{Salvelinus alpinus}), 
brown trout (\emph{Salmo trutta}), 
and piscivorous birds that have the potential 
to induce fluctuations in the stickleback population from the top down
\citep{einarsson2004moulting, gudbergsson2004}.
Finally, M\'{y}vatn's stickleback can sustain high loads of the cestode parasite
\emph{Schistocephalus solidus} \citep{gislason1998, karvonen2013},
which could lead to host-parasite fluctuations.

To explore the spatiotemporal dynamics of M\'{y}vatn's stickleback population, 
we fit a stage-structured metapopulation model \citep{caswell2001matrix}
to a 29-year times series of population estimates derived from trapping data.
The model includes temporal variation in demographic rates such as recruitment and survival,
which were characterized as random walks constrained by the observed data.
This approach provides great flexibility in modeling changes in the demographic rates,
including those implicitly arising from nonlinear and density-dependent processes
\citep{zeng1998, ives2012}.
Equipped with the parameterized model, we address the following questions:
%
\begin{enumerate}[label=(\alph*)]
\item
To what extent are stickleback population fluctuations periodic, 
and to what extent does transience arising from non-equilibrium state distributions
alter this periodicity;
%
\item
To what extent are the two sub-populations coupled through dispersal,
and how does this affect the dynamics; 
%
\item
To what extent are the demographic rates synchronized between the two sub-populations,
and what are the relative contributions of within-basin demography 
and between-basin coupling to the lake-wide population dynamics.
\end{enumerate}
%
By investigating these topics, 
we seek to illustrate the demographic basis for spatially-structured population fluctuations
in a wild population, 
which may provide general lessons for other systems. 





% ---------------------------------------------------------------------------------------
% ---------------------------------------------------------------------------------------
% Methods
% ---------------------------------------------------------------------------------------
% ---------------------------------------------------------------------------------------




\section*{Methods}

%========================================================================================

\subsection*{Study system}

%========================================================================================

\subsection*{Population estimate}

From 1991 to 2016, surveys of the stickleback population were conducted twice annually, 
the first in either June or July and second in August or September. 
Samples were collected from 5 stations in the south basin and 3 stations in the noth basin.
These stations provided wide coverage of the two basins, 
with the exception of sites near the shoreline 
(sampling of shorline sites began in 2008 but those data are not included here)
and of the eastern portion of the south basin which historically has had neglibile 
densities of sticklebacks.
For each sampling event at each station, 
5 traps were set for two 12-hour [I think] sessions, 
one during the day and one during the night.
After trapping, individuals were sorted into two size classes,
with a threshold of 50mm in the June/July sampling and 45mm in the August/September sampling.
These size categories roughly map onto sexual maturity,
and while it is possible for individuals less than 45mm to be sexually mature,
for this study we treat the small size class as non-reproductive "juveniles"
and the large size class as potentially reprooductive "adults".

Within each basin and size class, 
station catches were of comparable magnitude and 
strongly syncrhonized through time (appendix).
Therefore, our analysis focused on the dynamics of basin-level abundances
of the two size classes through time.
We used a modified N-mixture model to estimate relative basin-level population densities
for the two size classes,
which we than scaled by the ratio of sampling areas to obtain estimates of relative
abundance between the two basins.

The probability of trapping some number of individuals $y_i$ 
was modeled as
%
\begin{equation}
  y_i \sim \text{Binomial}\left(\upphi_{g_{\upphi}[i]}, ~\upnu_{g_{\upnu}[i]}\right)
\end{equation}
%
where $\upphi_{g_{\upphi}[i]}$ is the detection probability 
and $\upnu_{g_{\upnu}[i]}$ is the population density for a given 
station-size-date combination for the $i$th observation.
The density is in units of individuals per station, 
with each station characterizing a sampling area that is 
taken to be the same size for all stations.
The functions ${g_{\upphi}[i]}$ and ${g_{\upnu}[i]}$ map observations to the 
appropriate grouping. 

The discrete population density for each station-size-date combination was modeled as
%
\begin{equation}
  \upnu_{g_{\upnu}[i]} \sim \text{Poisson}\left(\uplambda_{g_{\uplambda}[i]}\right)
\end{equation}
where $\uplambda_{g_{\uplambda}[i]}$ is the mean population density across stations 
within a basin-size-date combination. 
Variation in $\uplambda_{g_{\uplambda}[i]}$ across basin, size, and date was characterized as 
%
\begin{equation}
  \uplambda_{g_{\uplambda}[i]} \sim 
    \text{Gamma}\left(\upmu_{\uplambda}^2/\upsigma_{\uplambda}^2,
      ~\upmu_{\uplambda}/\upsigma_{\uplambda}^2\right)
\end{equation}
%
with overall mean $\upmu_{\uplambda}$ and standard deviation $\upsigma_{\uplambda}$. 
The detection probability for the $i$th observation was modeled using a 
"logistic regression"-style approach:
%
\begin{equation}
  \upphi_{g_{\upphi}[i]} = 
    \text{logit}^{-1}\left(\mathbf{x}_{g_{\upphi}[i]}^\text{T}~{\boldsymbol\upbeta}\right)
\end{equation}
%
\noindent where $\mathbf{x}_{g_{\upphi}[i]}^\text{T}$ is a transposed vector 
of predictor values for the $i$th observation
(including '1' in the first column for the overall intercept)
and $\boldsymbol\upbeta$ is a column vector of coefficients. 
We included main effects for trapping time (day vs. night) and station 
to account for  systematic differences in detection probability, 
as well as main effects and two-way interactions for size class, basin, and date 
to provide a source of "observation error" beyond the binomial sampling process. 

The model was fit using JAGS via 'runjags' in R. 
Weakly-informative gamma priors were used for
$\upmu_{\uplambda}$ and $\upsigma_{\uplambda}$, and Gaussian priors 
for the effects in ${\boldsymbol\upbeta}$. 

%========================================================================================

\subsection*{Demographic model}

We used a multiregional stage-structured model with time-varying demographic rates
to characterize the dynamics of the stickleback population. 
The model included mortality, transitions between life stages, movement between basins,
and recruitment of new individuals into the population. 
These demographic rates were organized into matrices,
which could then be used to project the abundance of individuals in each 
stage and basin through time.
We inferred the demographic rates by  
fitting the model to the times series of abundance estimates,
which is sometimes referred to as "inverse modeling". 
For parameter estimation, 
the model was formulated to match the semiannual nature of the abundance estimates.
However, our demographic analysis of the model focused on the annual dynamics, 
both to circumvent interpretational challenges 
posed by the unequal sampling intervals within years
and to better match the life history of threespine stickleback in M\'{y}vatn.

Time-dependent transitions between stages and basins 
were modeled with a $4\times{4}$ transition rate matrix $\mathbf{Q}_t$ defined as
%
\begin{equation} \label{eq:Q}
\mathbf{Q}_t = 
f\left(
\left[
\begin{array}{c|ccc}
    \overset{s}{\mathbf{U}_t} & \overset{n\rightarrow s}{\mathbf{D}_t} \\
    \hline
    \overset{s\rightarrow n}{\mathbf{D}_t} & \overset{n}{\mathbf{U}_t} 
    \end{array}
\right]
\right)
\end{equation}
%
where $t$ is time, 
$\overset{i}{\mathbf{U}_t}$ is a $2\times{2}$ matrix of stage-transition rates
for basin $i$ ($s$ for south and $n$ for north), and
$\overset{i\rightarrow h}{\mathbf{D}_t}$ is a $2\times{2}$ matrix 
of basin-transition rates from basin $i$ to $h$.
The function $f$ maps the sub-matrices to a full transition rate matrix that 
quantifies the overall loss rate from each stage-basin combination by 
subtracting the sum of the off-diagonal elements from the diagonal element 
for each column (the fully expanded form of $\mathbf{Q}_t$ is shown in the appendix).

The stage-transition matrix for basin $i$ was defined as
%
\begin{equation} \label{eq:U}
\overset{i}{\mathbf{U}_t} = 
\left[
\begin{array}{cccc}
    -\overset{j_i}{\upmu_t} & 0 \\
    \overset{j_i\rightarrow a_i}{\upgamma_t} & -\overset{a_i}{\upmu_t}
    \end{array}
\right]
\end{equation}
%
where $\overset{k_i}{\upmu_t}$ is the mortality rate for stage $k$ in basin $i$
($j$ for juveniles and $a$ for adults)
and $\overset{j_i\rightarrow a_i}{\upgamma_t}$ 
is the development rate from juveniles to adults.
Note that mortality implicitly entails transition to a "death state" that is omitted 
for concision, as dead individuals do not contribute further to the dynamics. 
The basin-transition matrix was defined as
%
\begin{equation} \label{eq:D}
\overset{i\rightarrow h}{\mathbf{D}_t} = 
\left[
\begin{array}{cccc}
    \overset{j_{i}\rightarrow j_{h}}{\updelta_t} & 0 \\
    0 & \overset{a_{i}\rightarrow a_{h}}{\updelta_t}
    \end{array}
\right]
\end{equation}
where $\overset{k_{i}\rightarrow k_{h}}{\updelta_t}$ is the movement rate for stage $k$ 
from basin $i$ to $h$. 
While $\mathbf{Q}_t$ does not explicitly include transitions from juveniles to adults 
across basins, these transitions occur implicitly through the combined effects of
development and movement over a given time interval.

In order to project the transition dynamics in discrete time, 
we related the transition \emph{rate} matrix $\mathbf{Q}_t$ 
to the transition \emph{probability} matrix $\mathbf{P}_t$ 
with the following equation
%
\begin{equation} \label{eq:P}
\mathbf{P}_t = e^{ \upDelta T \mathbf{Q}_t}
\end{equation}
%
which is the solution to the differential equation associated with the Markov process
specified by $\mathbf{Q}_t$ when projected over interval $\upDelta T$.

Appearance of new individuals in the population was modeled with
a $4\times{4}$ recruitment matrix 
%
\begin{equation} \label{eq:Q}
\mathbf{R}_t = 
\left[
\begin{array}{c|ccc}
    \overset{s}{\mathbf{R}_t}  & \mathbf{0} \\
    \hline
    \mathbf{0} & \overset{n}{\mathbf{R}_t}
    \end{array}
\right]
\end{equation}
%
comprising basin-specific sub-matrices
%
\begin{equation} \label{eq:R}
\overset{i}{\mathbf{R}_t} = 
\upDelta T
\left[
\begin{array}{cccc}
    0 & \overset{i}{\uprho_t} \\
    0 & 0
    \end{array}
\right]
\end{equation}
%
where $\overset{i}{\uprho_t}$ is the per capita recruitment rate for basin $i$,
with the assumption that individuals in the small size class ("juveniles") cannot reproduce. 
Recruitment includes both offspring production and survival until 
they are observed as "juveniles" at the next time step. 
This formulation assumes that individuals born in a given basin remain there until the
next time step, at which point they can disperse according 
to $\overset{i\rightarrow h}{\mathbf{D}_t}$.
Note that while we parameterized recruitment as a "rate" to account for
unequal projection intervals, recruitment is not evenly distributed throughout the 
year and is likely concentrated in the summer 
(although not necessarily within the "summer" interval between June and August sampling).

Given the time-dependent transition probability and recruitment matrices, 
the population dynamics were projected as
%
\begin{equation} \label{eq:RPX}
    \mathbf{x}_t = (\mathbf{P}_{t-1} + \mathbf{R}_{t-1})~\mathbf{x}_{t-1}
\end{equation}
%
where $\mathbf{x}_{t-1}$ is a vector of stage- and basin-specific abundances at time $t-1$:
%
\begin{equation} \label{eq:X}
\mathbf{x}_{t-1} = 
\left[
\begin{array}{cccc}
    \overset{j_s}{x_{t-1}} \\
    \overset{a_s}{x_{t-1}} \\
    \overset{j_n}{x_{t-1}} \\
    \overset{a_n}{x_{t-1}}
    \end{array}
\right]
\text{.}
\end{equation}
%

The elements of $\mathbf{P}_t + \mathbf{R}_t}$ characterize the overall contribution of 
each state to each other state,
with each contribution integrating multiple underlying demographic rates.
Becuase it is these overall state contribution that manifest in the data,
the underlying demographic rates can be interpreted as latent variables 
that underpin the state contributions that are the primary focus of the analysis.

We modeled temporal variation in the latent demographic rates 
(i.e. the non-zero elements of 
$\overset{i}{\mathbf{U}_t}$, $\overset{i\rightarrow h}{\mathbf{D}_t}$, 
and $\overset{i}{\mathbf{R}_t}$)
as 
%
\begin{equation} \label{eq:alpha}
\upalpha_t = \upalpha'_t\times e^{\upbeta_{\upalpha}~b_t}
\end{equation}
%
where $\upalpha_t$ is a given demographic parameter,
$\upalpha'_t$ is the value from a corresponding random walk, 
$\upbeta_{\upalpha}$ is a fixed season effect unique to that demographic parameter,
and $b_t$ is a binary index for season (summer = 0 and winter = 1). 
The fixed season effects allowed us to account for systematic differences between the 
sampling intervals that remained consistent throughout the entire time series.

The random walk for each parameter was modeled as
%
\begin{equation} \label{eq:walk}
\upalpha'_t \sim \text{Normal}
\left(
      \upalpha'_{t-1},~\upsigma_{g[\upalpha]}
\right) \text{T}(0, \infty)
\end{equation}
%
with standard deviation $\upsigma_{g[\upalpha]$ and 
truncated from the left ensure that values remained positive. 
The function $g$ maps $\upalpha$ to a given type of demographic process 
(stage-transition, basin-transition, or recruitment),
such that a single random walk standard deviation was used 
for each demographic process type.
While formulated as random walks, 
the realized sequences of inferred demographic rates 
were not truly random walks because they were constrained by fitting the model to data.
Therefore, the "random walks" are best understood as a convenient method 
for allowing the demographic rates to vary through time with implicit "smoothing"
arising from the autocorrelated nature of the walks.

We fit the model in a Bayesian framework.
The prior distributions for random walk standard deviations and initial values were 
modeled using gamma distributions with shape parameter 1.5 and rate parameters set 
to provide a reasonable scale for the corresponding parameter (see appendix). 
This prior has zero density at 0 but is also concave down near zero, 
and therefore allows posterior values arbitrarily close to zero without 
artificially concentrating density at zero. 
Furthermore, its has a long tail and is weakly informative relative to its overall scale. 

% The initial values of the random walks had a tendency to concentrate 
% near the mean of the prior, 
% which is unsurprising given that purpose of the random walks was to allow the 
% demographic rates to move towards reasonable values as informed by the data, 
% which in turn provided limited information for initial value itself.
% To compensate for this, we penalized the initial values for deviating from the
% realized means of the random walks for each step in the MCMC sampling using 
% a normal distribution scaled to have a maximum density of 1 
% (i.e. have no contribution to the posterior) 
% when the initial value equaled the random walk mean,
% and then declined as the initial value deviated from the mean
% at a rate controlled by the corresponding 
% random walk standard deviation $\upsigma_{g[\upalpha]$.
% This way the estimates of initial values were informed by the full time series.

The likelihood of the observed data given the projected abundances was calculated as
%
\begin{equation} \label{eq:likelihood}
\mathcal{L} = 
\displaystyle\prod_{r}
\displaystyle\prod_{t}
\text{Normal}
\left(
\overset{r}{y_t}~|~\overset{r}{x_t},~\upsigma_y
\right)
\end{equation}
%
with standard deviation $\upsigma_y$.
For population densities that are necessarily non-negative, 
it is common to model the likelihood using a distribution that is similarly constrained,
such as a log-normal distribution. 
However, in our case using a log-normal distribution resulted in estimates of 
$\overset{r}{x_t}$ that "ignored" large fluctuations in the observed data 
that likely reflected real changes in the population but were compressed on a log scale. 
Therefore, we opted for a normal (Gaussian) likelihood. 
Because the model was parameterized in a way that ensured
$\overset{r}{x_t}$ was non-negative,
the posterior distribution of $\overset{r}{x_t}$ was also guaranteed to be non-negative
even though normally distributed variables can take negative values. 
However, to model the initial abundances using a normal distribution,
it was necessary to truncate from the left at zero:
%
\begin{equation} \label{eq:x0}
\overset{r}{x_{t=1}} \sim
\text{Normal}
\left(
\overset{r}{y_{t=1}},~\upsigma_y
\right) \text{T}(0, \infty)
\text{.}
\end{equation}
%

We fit the model using Stan 2.19 run from R 3.5.1
with the ‘rstan’ package.
Convergence was assessed by examining the effective sample size 
per iteration of the Markov chain, 
the number of divergent transitions, 
and the potential scale reduction factor (\^{R}),
which quantifies the relative variance within and between chains. 
For all parameters, each of these diagnostics fell within 
acceptable ranges as defined by Stan 2.19.

%========================================================================================

\subsection*{Annual dynamics}

While we parameterized the model in terms of continuous rates
to accommodate the seasonal nature of the data,
we focused our demographic analysis on the annual projection matrix, 
defined as 
%
\begin{equation} \label{eq:A}
\mathbf{A}_y = \left(\mathbf{P}_{\tau[y]+1} + \mathbf{R}_{\tau[y]+1}\right) 
                \left(\mathbf{P}_{\tau[y]} + \mathbf{R}_{\tau[y]}\right)
\end{equation}
%
for year $y$ and sequential time steps within that year $\tau[y]$ and $\tau[y]+1$,
with the year defined to start with the June/July census.
$\mathbf{A}_y$ projects the dynamics from
June/July of one year to June/July of the next year.
Analyzing the annual projection matrix allowed us to circumvent interpretational
issues that arose from unequal sampling intervals within years
and aligned with the annual nature of spawning for M\'{y}vatn stickleback.
To characterize the contribution of different demographic processes 
to the population dynamics,
we calculated the annual contribution of each stage $\times$ basin combination 
to each other stage $\times$ basin combination in the next year as
%
\begin{equation} \label{eq:A}
\mathbf{C}_y = \mathbf{A}_y~{\circ}~\mathbf{X}_{t[y]}
\end{equation}
%
where $\mathbf{C}_{y}$ is a $4\times 4$ matrix of annual contributions,
$\mathbf{X}_{t[y]}$ is a $4\times 4$ matrix 
with idententical rows ${\mathbf{x}_{t[y]}}^\text{T}$,
and ${\circ}$ denotes the Hadamard (element-wise) product. 
This is essentially the same operation as a standard demographic projection
(e.g., equation \ref{eq:RPX}) but without summing the contributions to a given state
from different sources states. 


We characterized the overall dynamics of the population in terms of the annual 
population growth rate $\uplambda_y$, calculated as
%
\begin{equation} \label{eq:lam-n}
\uplambda_y = \frac{N_y}{N_{y-1}}
\end{equation}
%
where $N_y$ is the summed abundance across basins and life stages in year $y$.
Temporal variation in $\uplambda_y$ reflects both variation in the demographic rates
and transient fluctuations due to non-equilibrium state distribution. 
Therefore, it is also informative to calculate the asymptotic population growth rate that 
would obtian if the demographic rates in a given year were fixed indefinitely, 
which is equal to the leading eigenvalue of $\mathbf{A}_y$. 
The asymptotic growth rate $\uplambda^{'}_y$ implies population dynamics
%
\begin{equation} \label{eq:n-prime}
N^{'}_y = \uplambda^{'}_y N^{'}_{y-1}
\end{equation}
%
where $N^{'}_y$ is the analogue to $N_y$ that would obtain in the absence of 
trasience due to non-equilibrium state distributions.
We characterized the importance of transience for the abundance in a given year
as the proportional deviation of the realized population size from asymptotic analogue:
%
\begin{equation} \label{eq:dev}
\kappa = \frac{N_y - N^{'}_{y}}{N^{'}_y}
\end{equation}
%

We related temporal variation in the realized population growth rate 
to changes in the demographic rates 
through both "prospective" and "retrospective" analyses \citep{caswell2001matrix}.
The prosepctive analysis assessed how the population 
would respond to hypothetical perturbations at each time step,
which we quantified as
the partial derivaties (i.e. sensitivities) of $\uplambda_y$ 
with respect to perturbations in elements of $\mathbf{A}_y$ 
according to \cite{caswell2007sensitivity}. 
While proportional sensitivies (i.e. elasticities) 
are commonly used in demographic analyses,
we used sensitivies to retain information about variation in $\uplambda_y$
that is normalized out in calculation of elasticities. 
When projected over a single time step, 
the sensitivity of $\uplambda_y$ with respect to a given element of $\mathbf{A}_y$
equals the relative abundance 
of the contributing population state (Appendix).
In other words, the population growth rate is most sensitive to perturbations
in the contibutions of abundant population states.
However, when perturbations to elements of $\mathbf{A}_y$ are propogated through time,
the sensitivities also include fluctuations
in relative abundances that themselves arise from elements of $\mathbf{A}_y$,
as occurs implicitly for conventional sensitvity analysis of the asymptotic growth rate.

The retrospective analysis decomposed annual variation 
in the realized population growth rate 
into contributions from different demographic processes 
according to \cite{koons2016life} and \cite{koons2017understanding}.
For a given time step, 
the contribution of a change in a demographic rate to the change in $\uplambda_y$
has first-order approximation
%
\begin{equation} \label{eq:l-cont}
\text{contribution}^{\upDelta \uplambda_y}_{\uptheta_k} \approx
(\uptheta_{k,y+1} - \uptheta_{k,y})
\left.\frac{\uppartial\uplambda}{\uppartial\uptheta_k}
\right\vert_{\overline{\boldsymbol\uptheta}}
\end{equation}
%
where $\uptheta_{k,y}$ is element $k$ of a vector $\boldsymbol\uptheta_y$ 
containing elements of the projection matirx $\mathbf{A}_y$ and 
the state distribution vector $\hat{\mathbf{x}}_{\tau[y]}$, 
with total abundance of the state distribution 
standardized to sum to one for each time step.
The partial derivatives in equation \ref{eq:l-cont}
are evaluated for a vector $\overline{\boldsymbol\uptheta}$
containing average values for each element across years.
Similarly, the overall contribution of variance in $\uptheta_k$ 
to variance in $\uplambda_y$ is approximated as 
%
\begin{equation} \label{eq:l-var}
\text{contribution}^{\text{var}(\boldsymbol\uplambda)}_{\uptheta_k} \approx
\sum_{h}\text{cov}(\boldsymbol\uptheta_k, \boldsymbol\uptheta_h)
\left.\frac{\uppartial \uplambda}{\uppartial \uptheta_k}
\frac{\uppartial\uplambda}{\uppartial\uptheta_h}
\right\vert_{\overline{\boldsymbol\uptheta}}
\end{equation}
%
where $\boldsymbol\uplambda$, and $\boldsymbol\uptheta_k$ are vectors
of population growth and demographic rates across years.
While we included contributions of the state distribution 
to changes and variance in $\uplambda$ in our calculations,
these contributions were small relative to the direct contributions 
of the demogoraphic rates.
Therefore, to simplify the presentation we focused on contributions of the demogoraphic rates; 
full results are included in the appendix.

%========================================================================================

\subsection*{For appendix}

The population growth rate is related to elements of the annual project matrix
according to the following relationship:
%
\begin{equation} \label{eq:lam-A}
\uplambda_y = \frac{\sum_{j}\sum_{i}a_{ij}x_j}{\sum_{k}x_k}
\end{equation}
%
where $i$ and $j$ denote columns of the annual projection matrix associated with elements
$a_{ij}$,
and $x_j$ is the element of the population distribution vector associted with column $j$.
While $a_{ij}$ and $x_j$ vary annually, the year index is dropped for clarity.

If projected over a single interval and neglecting any dependence of the state
distribution on previous demographic rates, 
the sensitivity of $\uplambda_y$ with respect to element $a_{ij}$ is 
%
\begin{equation} \label{eq:lam-A}
\frac{\uppartial\uplambda_y}{\uppartial a_{ij}} = \frac{x_j}{\sum_{k}x_k}\end{equation}
%




% ---------------------------------------------------------------------------------------
% ---------------------------------------------------------------------------------------
% Results
% ---------------------------------------------------------------------------------------
% ---------------------------------------------------------------------------------------

% ---------------------------------------------------------------------------------------
\section*{Results}
% ---------------------------------------------------------------------------------------



% ---------------------------------------------------------------------------------------
\subsection*{Model fit and seasonal dynamics} 
% ---------------------------------------------------------------------------------------

The M\'{y}vatn stickleback population fluctuated substantially over the past
three decades, 
and the model largely succeeded in characterizing these fluctuations 
(Figure \ref{fig:fit}).
This stands in contrast to the reduced model with temporally-fixed demographic rates,
which largely failed to capture the population fluctuations (\emph{Supplemental materials}).
Therefore, the population fluctuations cannot be purely explained 
in terms of transient oscillations due to non-equilibrium state structure, 
but rather arise at least in part from directly changes in the demographic rates. 
The dynamics of the adults in both basins were strongly synchronized,
as judged by the large covariance relative to the mean variance of the two time series.
The juvenile dynamics were less sychronized with each other,
and they were ``spikier'' than the adult fluctuations.
In general, juveniles were substantially more abundant in the north basin than in the south,
with the exception of occasional peaks. 
The high abundance of south basin juveniles in June of 1992 appears quite extreme
relative to the rest of the time series for both basins and life stages.
However, this pattern was observed across multiple traps and locations within the south basin,
suggesting that it reflected genuinely elevated abundance.

Per capita recruitment varied substantially across years and was generally of comparable
magnitude between the two basins (Figure \ref{fig:rec}),
although it was at most modestly synchronized between basins.
Similar to juvenile abundance, per capita recruitment in the south 
was characterized by relatively sudden and short-lived peaks, 
in comparison with the north basin.
Total recruitment corresponded very closely with juvenile abundance (Figure \ref{fig:fit}),
reflecting the model's inference 
that few juveniles return to the juvenile class at the next time step (see below).
As noted above, 
any cross-basin recruitment would be attributed 
to within-basin recruitment by the model. 
However, gravid females and males with breeding coloration are routinely sampled in the 
south basin, indicating that much of the recruitment within the south basin is 
likely attributable to spawning in the south basin.

Adult survival probabilities were strongly synchronized between basins 
and also appeared similar to the north basin juveniles (Figure \ref{fig:surv}),
with all three had two peaks in survival corresponding with 
similar peaks in observed abundance (Figure \ref{fig:fit}).
This covariation in survival probabilities was less pronounced in the second half 
of the time series, although still present for the north basin juveniles and adults.
Overall, this suggests that survival probability was affected by processes relevant across 
the life cycle and present at the scale of the entire lake.
In contrast to the other population states, 
survival probability of south basin juveniles was very low,
although still high enough to constitute a meaningful contribution over the one or two-month
``summer'' projection interval.
While the difference between the survival probability of south basin juveniles and 
the other states is quite pronounced, 
the basis for this inference is apparent 
in the population estimates (Figure \ref{fig:fit}). 
Large, sudden increases in the abundance of south basin juveniles appear to contribute
to elevated abundance of neither juveniles nor adults at the next time step,
which directly implies low survival.
The model inferred a high development rate from the juvenile to the adult stage,
with a probability of 92\% [88\%, 95\%] over a six-month period. 
Note that the model implicitly treats all individuals within a given state 
as demographically identical,
so this development probability can be interpreted as the average 
for individuals classified as juveniles, 
including individuals that are just below the size threshold for being classified as adults.
This high development probability reflects the apparent correlation between north basin
juveniles and adults with a lag of approximately one time step (Figure \ref{fig:fit}).
As noted above, we fixed the development probability through time and for both basins
to facilitate estimation of the other demographic parameters.

The model formulates dispersal reciprocally and temporal variation in the 
dispersal probabilites is inferred from the net exchenge of individuals between
the two sub-populations.
Consequently, dispersal probabilites for the two basins necessarily covary negatively 
(Figure \ref{fig:disp}).
Therefore, the dispersal probability is best understood as a measure of the extent 
to which the sub-populations appeared to be dynamically mixed
expressed on a per capita basis.
By this measure, the north and south basins appeared 
to be substantially coupled (Figure \ref{fig:disp}), 
which is consistent with the close correspondence 
between the adult dynamics across basins.
To clarify the contribution of dispersal to the dynamics, 
we calculated the net movement of individuals between basins,
with positive values indicating movement from south to north (``northward'')
and negative values indicating the converse (``southward'').
In each time step, 
net movement was generally low and slightly southward,
with a few of peaks in net movement corresponding to 
both peaks in abundance
and changes in the difference between north and south dispersal probabilities.
The cumulative movement over the entire time series was clearly southward,
resulting in a substantial net subsidization of the south basin over that period.



% ---------------------------------------------------------------------------------------
\subsection*{Annual dynamics and sensitivity analysis} 
% ---------------------------------------------------------------------------------------

The transient population growth rate ($\lambda$) fluctuated substantially between years
(Figure \ref{fig:lam}),
with a geometric mean of 0.98 [0.97, 0.99] implying a slight average decline
of $2\%~\text{year}^{-1}$ ($\lambda = 1$ would indicate no change).
Variance per se reduced the transient growth rate, 
as the arithmetic mean (1.19 [1.14, 1.24]) was much higher than the geometric mean
and would have implied substantial growth ($19\%~\text{year}^{-1}$)
if realized in the population.
The asymptotic growth rate was similar in this regard,
although the geometric mean (1.03 [1.00, 1.06]) was slightly higher 
than for the transient growth rate.
This is because the asymptotic growth rate lacked the extreme fluctuations 
of the transient growth rate,
leading to a lesser reduction relative to the arithmetic mean (1.20 [1.14, 1.27]).
These results show that variance arising from both changes in demographic
rates and transience per se reduced the the average population growth rate,
although the effect of the former predominated.

The wavelet decomposition detected a strong periodicity of around 6 years
in the asymptotic growth rate over the first two decades, 
which is clearly visible in the corresponding time series (Figure \ref{fig:lam}).
The wavelet decomposition of the transient growth rate was generally similar,
although the 6-year periodicity was weaker.
Furthermore, there was a highly localized periodicity of around 3 years 
at the beginning of the time series 
that coincided with the rapid increase and subsequent decline of south basin juveniles.
Together, these results indicate that the population displayed cyclic dynamics 
driven by changes in demographic rates.
However, these cycles were partially obscured by transient fluctuations arising 
from the non-equilibrium state distributions.
The absence of conspicuous periodicity in the final decade
for both the asymptotic and transient growth rates 
suggests a shift in the dynamics, 
although it is difficult to draw strong conclusions 
based on relatively short time series.

The asymptotic growth rate declined with total abundance
(slope = -0.13 [-0.17, -0.11]; Figure \ref{fig:dens}).
Futhermore, 
the intercept of the linear model implied that 
the asymptotic growth rate approached 1.77 [1.59, 2.02] 
as population size approached zero,
which was much higher than the ``observed'' geometric mean of 1.03.
The slope and intercept implied an equilibrium abundance (when $\lambda=1$)
of 4.25 [3.97, 4.52] relative abundance units,
about which there were substantial fluctuations in overall abundance 
(variation along the x-axis in Figure \ref{fig:dens}).
These results suggest that negative density dependence was relevant for the populaton
dynamics without tightly constaining the population to a fixed equilibrium,
which is consistent with the periodicity of the asymptotic growth rate.
While we did not attempt to formally assess whether the strength of density dependence
changed through time,
no such shift was visually apparant in Figure \ref{fig:dens}.
Despite the strong negative relationship between the growth rate and population size, 
there was substantial residual variation with modest
temporal autocorrelation (0.20 [0.08, 0.31]) associated with this linear trend.
This indicates that there were processes, either density dependent or independent,
relevant for the dynamics 
that were not fully captured in the linear model of the population growth rate.

The elasticities in Figure \ref{fig:elas} quantified the proportional change 
in the transient growth rate 
that would result from a proportional perturbation to a given demographic rate.
An elasticity of 1 indicates that a 1\% change in a demographic rate would result in
a 1\% in the population growth rate.
In general, the elasticities were positive,
which is expected as increases in processes such as survival or recruitment
should contribute positively to the growth rate.
However, the elasticity with respect to dispersal from north to south was negative,
which means that increases in southward dispersal 
tended to reduce the population growth rate.
This reflected unfavorable demographic conditions in the south basin,
chiefly attributable to low juvenile survival,
but also due to somewhat lower adult survival and inconsistent recruitment.
Counter-intuitively, the elasticity with respect to south basin recruitment was
slightly negative in some years. 
This was a transient phenomenon,
as demonstrated by the corresponding elasticities 
for the asymptotic growth rate that were positive
(X; the asymptotic results were otherwise largely the same as the transient ones).
The elasiticities with respect to 
demographic rates originating in the north basin were all of greater
average magnitude than those originating in the south,
indicating that that total population growth rate was most sensitive to processes
occurring in the north basin.
Among those north basin rates, juvenile survival and recruitment were associated with
the largest elasticities,
suggesting that the dynamics were most sensitive to processes early in the life cycle.
While the elasticities varied somewhat across years, 
these differences were modest relative to the differences among the demographic rate types.
Note that the temporal variation in the elasticity of development arose from
variation in the other demographic rates, 
as the development probability itself was fixed through time.

Together, the southward orientation of net dispersal and the negative elasticity 
of the population growth rate with respect to southward dispersal implied source-sink dynamics. 
To explore this further, 
we projected the population dynamics eliminating dispersal between basins, 
assuming that of the all individuals that would otherwise disperse remained within
their basin of origin.
This resulted in the rapid extinction of the south basin sub-population
and approximately exponential growth 
of the north basin sub-population (Figure \ref{fig:sink}).
Under natural conditions, 
altering the number of individuals in each basin 
would likely alter the other demographic rates.
For example, increased rention of individuals in the north basin in the absence of dispersal
could increase intraspecific competition and thereby lower recruitment or survival.
Therefore, the scenario described above is not intended to be a realistic projection, 
but rather an illustration of the direct effect of dispersal on the observed dynamics.
The key points are 
(a) that the south basin goes extinct and 
(b) that the north basin grows beyond what it would with southward dispersal,
given this hypothetical scenario.
Therefore, the model does in fact display source-sink dynamics,
with persistence in the south basin contingent on subsidies from the north.














% ---------------------------------------------------------------------------------------
% ---------------------------------------------------------------------------------------
% Discussion
% ---------------------------------------------------------------------------------------
% ---------------------------------------------------------------------------------------




\section*{Discussion}

1. Population crash driven by declines in both recruitment and adult survival 
in the north basin (both per capita and total contributions).
\linebreak 
\linebreak 
2. Demographic rates are fairly synchronized between basins, 
suggesting a role of lake-wide processes
\linebreak
\linebreak
3. Variation in population growth rate is mostly due to direct effect of changes 
in demographic rates, 
but transience per se is relevant for absolute abundance in a given year
\linebreak
\linebreak
4. Population growth rate is most sensitive to perturbations in juvenile development
and adult survival within the north basin,
while recruitment within the north basin is the dominant contributor to variance.
\linebreak
\linebreak
5. Movement between basins is important, 
both in that it can be large relative to internal contributions 
and it contributes to the variance in population growth rate 
(~35\% when combining north -> south and south -> north).
\linebreak
\linebreak
6. Both "prospective" and "retrospective" analyses are relevant. 
The latter is more directly explanatory, 
but the former provides important context.
Furthermore, 
the prospective sensitivity analysis is likely more informative for the evolutionary 
dimensions, 
since selection gradients (and the like) depend 
on the sensitivity of the population growth rate
to changes in traits that influence the demographic rates.









% ---------------------------------------------------------------------------------------
% ---------------------------------------------------------------------------------------
% Acknowledgments
% ---------------------------------------------------------------------------------------
% ---------------------------------------------------------------------------------------
% You may wish to remove the Acknowledgments section while your paper
% is under review (unless you wish to waive your anonymity under
% double-blind review) if the Acknowledgments reveal your identity.
% If you remove this section, you will need to add it back in to your
% final files after acceptance.

% \section*{Acknowledgments}
%
% OEC would like to thank the world. GHC is much indebted to the solar system. AQE was supported by a generous grant from the Milky Way (MW/01010/987654).


% ---------------------------------------------------------------------------------------
% ---------------------------------------------------------------------------------------
% Appendices
% ---------------------------------------------------------------------------------------
% ---------------------------------------------------------------------------------------

% \newpage{}
%
% \input{app_A}


% ---------------------------------------------------------------------------------------
% ---------------------------------------------------------------------------------------
% Literature Cited
% ---------------------------------------------------------------------------------------
% ---------------------------------------------------------------------------------------



\bibliographystyle{ecology.bst}
\clearpage

\bibliography{refs.bib}

\clearpage

\begin{figure}
\centering
\includegraphics{../model/figures/figs/fig_fit.pdf}
\caption{\label{fig:fit}
Model fit
}
\end{figure}

\begin{figure}
\centering
\includegraphics{../model/figures/figs/fig_cont.pdf}
\caption{\label{fig:cont}
Annual demographic projection matrix
}
\end{figure}

\begin{figure}
\centering
\includegraphics{../model/figures/figs/fig_lam.pdf}
\caption{\label{fig:pop}
Population
}
\end{figure}

\begin{figure}
\centering
\includegraphics{../model/figures/figs/fig_sens.pdf}
\caption{\label{fig:sens}
Sensitivity analysis
}
\end{figure}

\begin{figure}
\centering
\includegraphics{../model/figures/figs/fig_cov.pdf}
\caption{\label{fig:cov}
Variance partitioning
}
\end{figure}

% \begin{table}
% \caption{\label{tab:season}
% Season effects.
% The estimates quantify the proportional relationship of winter rates (August to June) 
% relative to summer rates (June to August). 
% Note that the winter period is longer than the summer period, 
% which means that the winter effects receive more weight on an annual basis. 
% Point estimates are posterior medians. 
% The 68\% uncertainty intervals are based on quantiles 
% and are analagous to the coverage of standard errors.
% }
% \begin{tabular}{p{4cm}p{4cm}p{4cm}p}
% \toprule
% rate          &   stage    &    basin                       &  estimate (68\% interval) \\
% \midrule
% mortality     &   small    &    south                       &    0.87 (0.54, 1.41) \\
% mortality     &   small    &    north                       &    0.82 (0.51, 1.31) \\
% mortality     &   large    &    south                       &    1.17 (0.71, 1.80) \\
% mortality     &   large    &    north                       &    0.64 (0.40, 1.00) \\
% development   &   small    &    south                       &    1.41 (0.84, 2.35) \\
% development   &   small    &    north                       &    1.69 (1.02, 2.73) \\
% movement      &   small    &    south $\rightarrow$ north   &    1.07 (0.64, 1.77) \\
% movement      &   small    &    north $\rightarrow$ south   &    1.00 (0.60, 1.70) \\
% movement      &   large    &    south $\rightarrow$ north   &    1.04 (0.66, 1.60) \\
% movement      &   large    &    north $\rightarrow$ south   &    0.91 (0.60, 1.35) \\
% recruitment   &   -----    &    south                       &    0.44 (0.31, 0.64) \\
% recruitment   &   -----    &    north                       &    0.63 (0.46, 0.90) \\
% \bottomrule
% \end{tabular}
% \end{table}
% 
% 
% \clearpage
% 
% 
% \begin{figure}
% \centering
% \includegraphics{../model/output/full/fig_fit.pdf}
% \caption{\label{fig:fit}
% Fit of demographic model to abundance estimates.
% Points and thin lines are the data,
% thick lines are the model estimates,
% and the shaded regions are 68\% uncertainty intervals
% analagous to the coverage of standard errors.
% The abundance estimates are from an N-mixture model
% applied to trap-level data,
% and account for variation in detection rates
% by life stage, basin, and through time.
% }
% \end{figure}
% 
% \clearpage
% 
% \begin{figure}
% \centering
% \includegraphics{../model/output/full/fig_mort_dev.pdf}
% \caption{\label{fig:mort_dev}
% Temporal variation in mortality and development rates.
% The rates are per capita and standardized to an interval of one month
% and exclude season effects.
% A rate equal to one implies that that the expected time to transition
% (i.e. death or development) for a given individual is one month.
% }
% \end{figure}
% 
% \clearpage
% 
% \begin{figure}
% \centering
% \includegraphics{../model/output/full/fig_move.pdf}
% \caption{\label{fig:move}
% Temporal variation in movement rates between basins.
% The rates are per capita and standardized to an interval of one month
% and exclude fixed season effects.
% A rate equal to one implies that that the expected time to transition
% (i.e. movement from one basin to the other) for a given individual is one month.
% }
% \end{figure}
% 
% \clearpage
% 
% \begin{figure}
% \centering
% \includegraphics{../model/output/full/fig_rec.pdf}
% \caption{\label{fig:rec}
% Temporal variation per capita recruitment.
% The rates are per capita and standardized to an interval of one month
% and exclude fixed season effects.
% Note that recruitment is not evenly distributed through the year;
% we present monthly rates to match the model parameterization 
% (see \ref{fig:net} for annual reproduction).
% }
% \end{figure}
% 
% \clearpage
% 
% \begin{figure}
% \centering
% \includegraphics{../model/output/full/fig_cont.pdf}
% \caption{\label{fig:cont}
% Annual contributions of each life stage $\times$ basin combination.
% The projection for each year begins in June/July and projects 
% to the next June/July 
% (depending on when the population was censused). 
% Note that colors denote the recipient basin, 
% while line types indicate whether the contribution is within or between basins.
% }
% \end{figure}
% 
% \clearpage
% 
% \begin{figure}
% \centering
% \includegraphics{../model/output/full/fig_lam.pdf}
% \caption{\label{fig:lam}
% Temporal variation in fitness $\lambda$ for each basin.
% Fitness corresponds to the asymptotic per capita population growth rate
% calculated from the annual projection matrix for each basin excluding movement.
% A fitness of one (thin horizontal line) indicates zero change 
% in the asymptotic population size for the corresponding basin.
% }
% \end{figure}
% 
% \clearpage
% 
% \begin{figure}
% \centering
% \includegraphics{../model/output/full/fig_r0.pdf}
% \caption{\label{fig:r0}
% Expected lifetime reproductive output $R_0$ of adults for each basin.
% $R_0$ was calculated from the annual projection matrix for each basin excluding movement
% and corresponds with the fitnesses shown in Figure \ref{fig:lam}.
% We excluded $R_0$ for juveniles, as it is fixed at 1 through time as a consequence
% of the their non-reproductive status.
% }
% \end{figure}
% 
% \clearpage
% 
% \begin{figure}
% \centering
% \includegraphics{../model/output/full/fig_dist.pdf}
% \caption{\label{fig:dist}
% Stable age distribution for each basin.
% The distribution was calculated from the annual projection matrix
% for each basin excluding movement
% and corresponds with the fitnesses shown in Figure \ref{fig:lam}.
% The relative contributions within a basin sum to one,
% which means that the relative contributions for the different life stages are mirror images 
% of each other.
% }
% \end{figure}
% 
% \clearpage
% 
% \begin{figure}
% \centering
% \includegraphics{../model/output/full/fig_elas.pdf}
% \caption{\label{fig:elas}
% Elasticity of fitness with respect to different life stages.
% The elasticities were calculated from the annual projection matrix 
% for each basin excluding movement
% and corresponds with the fitnesses shown in Figure \ref{fig:lam}.
% Elasticites quantify the sensitivity of fitness to small perturbations to 
% elements of the annual projection matrix scaled so that demographic rates with 
% different scales (e.g. mortality and fertility) can be compared to each other.
% For a given year and basin, the elasticities across all demographic rates sum to one,
% which means that the elasticities for the different life stages are mirror images 
% of each other.
% }
% \end{figure}

\clearpage

\clearpage

\end{document}


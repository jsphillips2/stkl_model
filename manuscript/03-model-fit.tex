
\section*{Demographic model} \label{model}

%========================================================================================

We used a multiregional stage-structured model with time-varying demographic rates
to characterize the dynamics of the stickleback population. 
The model projected the population dynamics due to 
within-basin recruitment, survival, and development, 
as well as dispersal between basins. 
Recruitment, survival, and dispersal were allowed to vary through time,
enabling the model to characterize a range of dynamics, 
including those implicitly due to endogenous (e.g., density dependence) 
and exogenous (e.g, environmental variation) processes. 
We estimated the demographic rates by  
fitting the model to the time series of abundance estimates. 

For a given time interval from $t-1$ to $t$, 
the structured population dynamics were projected as
%
\begin{equation} \label{eq:XPX}
    \mathbf{x}_t = \mathbf{P}_{t-1}~\mathbf{x}_{t-1}
\end{equation}
%
where $\mathbf{P}_{\tau}$ is 4 $\times$ 4 a matrix of demographic rates at time $\tau$, 
and $\mathbf{x}_{\tau}$ is a vector abundances 
for a given stage (juveniles $j$; adults $a$) 
and basin (south $s$; north $n$):
%
\begin{equation} \label{eq:X}
\mathbf{x}_{\tau} = 
\left[
\begin{array}{cccc}
    {x_{j,s,\tau}} \\
    {x_{a,s,\tau}} \\
    {x_{j,n,\tau}} \\
    {x_{a,n,\tau}}
    \end{array}
\right]
\text{.}
\end{equation}
%
The projection matrix $\mathbf{P}_{\tau}$ can be expressed as
%
\begin{equation} \label{eq:P}
\mathbf{P}_{\tau} = 
\left[
\begin{array}{c|ccc}
    \mathbf{W}_{s,\tau}  & \mathbf{B}_{s\rightarrow n,\tau} \\
    \hline
    \mathbf{B}_{n\rightarrow s,\tau} & \mathbf{W}_{n,\tau}
    \end{array}
\right]
\end{equation}
%
where $\mathbf{W}_{i,\tau}$ is a 2 $\times$ 2 matrix characterizing 
per capita contributions within basin $i$,
and $\mathbf{B}_{i\rightarrow k,\tau}$ is a 2 $\times$ 2 matrix characterizing 
contributions from basin $i$ to basin $k$.
Within-basin contributions were modeled as 
 %
\begin{equation} \label{eq:W}
\mathbf{W}_{i,\tau} = 
\left[
\begin{array}{cccc}
    \phi_{j,i,\tau}~(1-\gamma_{j}) & 
    \rho_{i,\tau} \\
    \phi_{j,i,\tau}~\gamma_{j}~(1-\delta_{a,i,\tau}) & 
    \phi_{a,i,\tau}~(1-\delta_{a,i,\tau})
    \end{array}
\right]
\end{equation}
%
where $\phi_{h,i,\tau}$ is the survival probability of life-stage $h$, 
$\gamma_{j}$ is the proportion of surviving juveniles that develop into adults,
$\delta_{a,i,\tau}$ is the proportion of surviving adults that disperse to the other basin,
and $\rho_{i,\tau}$ is per capita recruitment.
We modeled between-basin contributions as
%
\begin{equation} \label{eq:B}
\mathbf{B}_{i,\tau} = 
\left[
\begin{array}{cccc}
    0 & 
    0 \\
    
    \phi_{j,i,\tau}~\gamma_{j}~\delta_{a,i,\tau} & 
    \phi_{a,i,\tau}~\delta_{a,i,\tau}
    \end{array}
\right].
\end{equation}
%
We fixed $\gamma_{j}$ to a singe value for both basins and through time because 
it was difficult to statistically separate changes in development from changes 
in survival probability.
This is unsurprising, as both survival and development probabilities
deteremined the contribution of juveniles to the adult age class,
and the development probability was sufficiently 
high that few juveniles returned to the juvenile class at the next time step. 
Similarly, we assumed that only individuals that were adults at the end of the projection
interval dispersed, because a model allowing juveniles to disperse failed to detect 
a meaningful signature of juvenile dispersal 
(again due to the high probability of development over the course of a projection interval).
Taken literally, the model implies that individuals born in a given basin
remain within that basin until the next time step.
However, any recruitment across basins should largely manifest 
as additional temporal variability in the within-basin recruitment,
and therefore is implicitly accommodated by the model.

To accommodate the unequal duration of the the summer (2-3 months) and winter (9-10 month)
intervals, we modeled the transition probabilities 
($\phi_{h,i,\tau}$, $\gamma_{j}$, and $\delta_{a,i,\tau}$)
in terms of latent transition rates $\theta^{\alpha}_{\tau}$ 
(where $\alpha$ denotes the corresponding demographic parameter).
For each class of transition process (mortality, development, and dispersal),
we specified a transition matrix from which we could then calculate
the probability of transition over a given interval $\Delta T$ 
(see Appendix \ref{sec:A1}).
We did not explicitly model unequal projection intervals for reproduction,
as spawning was not evenly spread throughout the year but instead was concentrated
in the summer (but spanned the ``summer'' and ``winter'' sampling periods).

We modeled temporal variation in demographic rates
(including for $\alpha$ = $\rho_{i,\tau})$ using random walks:
%
\begin{equation} \label{eq:theta}
    \theta^{\alpha}_t & \sim \text{Normal}
        \left(
            \theta^{\alpha}_{t-1},~\sigma_{g[\alpha]}}
        \right) \text{T}(0, \infty)
\end{equation}
%
with standard deviation $\sigma_{g[\alpha]$ and 
truncated from the left to ensure that values remained positive. 
The function $g$ maps $\alpha$ to a given type of demographic process 
(reproduction, mortality, or dispersal),
such that a single random walk standard deviation was used 
for each type of demographic process.
While formulated as random walks, 
the realized sequences of inferred demographic rates 
were not truly random walks because they were constrained by fitting the model to data.
Therefore, the ``random walks'' are best understood as a convenient method 
for allowing the demographic rates to vary through time with implicit ``smoothing''
arising from the autocorrelated nature of the walks.

We fit the model in a Bayesian framework, 
using the abundance estimates from the trapping data. 
The likelihood of the ``observed'' abundance given the projected abundances 
and standard deviation $\sigma_y$ was calculated as
%
\begin{equation} \label{eq:likelihood}
\mathcal{L} = 
\displaystyle\prod_{h}
\displaystyle\prod_{i}
\displaystyle\prod_{\tau}
\text{Normal}
    \left(
        y_{h,i,\tau}~|~x_{h,i,\tau},~\sigma_y
    \right).
\end{equation}
%
For population densities that are necessarily non-negative, 
it is common to model the likelihood using a distribution that is similarly constrained,
such as a log-normal distribution. 
However, the multiplicative nature of population processes is already entailed 
in the population projection, 
and a log-normal likelihood reduces the relative contribution of large population sizes
that likely reflect meaningful dynamics.
Therefore, we opted for a normal (Gaussian) likelihood. 
Because the model was parameterized in a way that ensured
$x_{h,i,t}$ was non-negative,
the posterior distribution of $x_{h,i,t}$ was also guaranteed to be non-negative. 
We used gamma priors with shape parameter 1.5 and scale parameter 0.75
for the initial population size for each stage $\times$ basin combination, 
initial values for random walks,
and standard deviations for the random walks and likelihood.
A gamma distribution with shape parameter of 1.5 has zero density at zero 
and is concave down as it approaches its mode,
allowing the posterior to be arbitrarily close to zero 
while not being artificially drawn towards it.
This shape parameter, along with scale parameter of 0.75,
implies a mean of 2, 
which defines a reasonable scale for all of the parameters 
following the scaling of the population estimates (see below)

We fit the model using Stan 2.19 run from R 4.0.0
with the ‘rstan’ package.
To facilitate selection of prior parameterizations,
we scaled the observed data by dividing by the mean,
such that the resulting in data that had a mean of 1
and ranged from approximately 0 to 11.
We fit the model with 4 chains, 
15000 iterations (7500 of warm-up and 7500 of sampling),
and set the ``adapt-delta'' parameter to 0.9.
Convergence was assessed by the number of divergent transitions 
and the potential scale reduction factor (\^{R}),
which quantifies the relative variance within and between chains. 
To assess the extent to which the data supported statistically meaningful inferences
of temporal variation in demographic rates,
we compared the fit of the model to a reduced version will all demographic rates
fixed through time 
(while still accounting for the unequal projection interval for transition probabilities).
We assessed goodness-of-fit using two metrics: 
(a) the posterior median of the logarithm 
of the likelihood given by equation \ref{eq:likelihood}
and (b) the ``Leave-one-out Information Criterion'' or LOOIC, 
which is analogous to the Akaike Information Criterion (AIC)
and can be interpreted in a similar manner.
The full model had a much higher log-likelihood (posterior median: -230 vs. -378) 
and much lower LOOIC (657 vs. 809) than the reduced model, 
indicating that allowing temporal variation in demographic rates provided a
substantially improved fit to the data.



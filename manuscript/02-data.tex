
\section*{Data and population estimates} \label{data}

%========================================================================================

From 1991 to 2020, surveys of the stickleback population were conducted twice annually, 
the first in either June or July and second in August or September. 
Samples were collected from 5 stations in the south basin and 3 stations in the north basin.
These stations provided wide coverage of the two basins, 
with the exception of sites near the shoreline 
(sampling of shoreline sites began in 2008 but those data are not included here)
and of the eastern portion of the south basin which historically has had negligible 
densities of sticklebacks.
For each sampling event at each station, 
5 traps were set for two 12-hour sessions, 
one during the day and one during the night.
After trapping, individuals were sorted into two size classes,
with a threshold of 50mm in the June/July sampling and 45mm in the August/September sampling.
These size categories roughly map onto sexual maturity (although there is likely variation)
Within each basin and size class, 
station catches were of comparable magnitude and 
strongly synchronized through time (appendix).
Therefore, our analysis focused on the dynamics of basin-level abundances
of the two size classes through time.

We used a modified N-mixture model to estimate relative basin-level population densities
for the two size classes.
The probability of trapping some number of individuals $y_i$ 
was modeled as
%
\begin{equation}
  y_i \sim \text{Binomial}\left(\eta_{g_{\eta}[i]}, ~\nu_{g_{\nu}[i]}\right)
\end{equation}
%
where $\eta_{g_{\eta}[i]}$ is the detection probability 
and $\nu_{g_{\nu}[i]}$ is the population density for a given 
station-size-date combination for the $i$th observation.
The density is in units of individuals per station, 
with each station characterizing a sampling area that is 
taken to be the same size for all stations.
The functions ${g_{\eta}[i]}$ and ${g_{\nu}[i]}$ map observations to the 
appropriate grouping. 
The discrete population density for each station-size-date combination was modeled as
%
\begin{equation}
  \nu_{g_{\nu}[i]} \sim \text{Poisson}\left(\kappa_{g_{\kappa}[i]}\right)
\end{equation}
%
where $\kappa_{g_{\kappa}[i]}$ is the mean population density across stations 
within a basin-size-date combination. 
Variation in $\kappa_{g_{\kappa}[i]}$ across basin, size, 
and date was characterized as 
%
\begin{equation}
  \kappa_{g_{\kappa}[i]} \sim 
    \text{Exponential}\left(\zeta \right)
\end{equation}
%
with rate parameter $\zeta = 0.001$ (implying a mean of 1000).
The detection probability for the $i$th observation was modeled using a 
``logistic regression''-style approach:
%
\begin{equation}
  \eta_{g_{\eta}[i]} = 
    \text{logit}^{-1}\left(\mathbf{z}_{g_{\eta}[i]}^\text{T}~{\boldsymbol\beta}\right)
\end{equation}
%
\noindent where $\mathbf{z}_{g_{\eta}[i]}^\text{T}$ is a transposed vector 
of predictor values for the $i$th observation
(including '1' in the first column for the overall intercept)
and $\boldsymbol\beta$ is a vector of coefficients. 
We included main effects for station, trapping time (day vs. night), size class,
and the size class $\times$ trapping time interaction.
We used a Gaussian prior with mean of 0 and standard deviation of 2 
for the regression" coefficients $\boldsymbol\beta$.
The model was fit using JAGS via 'runjags' in R4.0.0 with,
using 4 chains and 15000 iterations (5000 adaptation, 5000 burn-in, and 5000 sampling).
Convergence was assessed by the potential scale reduction factor (\^{R}),
which quantifies the relative variance within and between chains. 




\section*{Model fit} \label{fit}

%========================================================================================

\clearpage

\begin{figure}
\centering
\includegraphics{../figures/figs/fig_fit.pdf}
\caption{\label{fig:fit}
Model fit
}
\end{figure}

\clearpage

\clearpage

\begin{figure}
\centering
\includegraphics{../figures/figs/fig_rec.pdf}
\caption{\label{fig:fit}
Model fit
}
\end{figure}

\clearpage

\clearpage

\begin{figure}
\centering
\includegraphics{../figures/figs/fig_surv.pdf}
\caption{\label{fig:fit}
Model fit
}
\end{figure}

\clearpage

%========================================================================================

\section*{Annual dynamics} \label{annual}

While we parameterized the model in terms of continuous rates
to accommodate the seasonal nature of the data,
we focused our demographic analysis on the annual projection matrix, 
defined as 
%
\begin{equation} \label{eq:A}
\mathbf{A}_y = \mathbf{P}_{t[y]+1} \mathbf{P}_{t[y]}
\end{equation}
%
for year $y$ and sequential time steps within that year $t[y]$ and $t[y]+1$,
with the year defined to start with the June/July census.
$\mathbf{A}_y$ projects the dynamics from
June/July of one year to June/July of the next year.
Analyzing the annual projection matrix allowed us to circumvent interpretational
issues that arose from unequal sampling intervals within years
and aligned with the annual nature of spawning for M\'{y}vatn stickleback.
To characterize the contribution of different demographic processes 
to the population dynamics,
we calculated the annual contribution of each stage $\times$ basin combination 
to each other stage $\times$ basin combination in the next year as
%
\begin{equation} \label{eq:A}
\mathbf{C}_y = \mathbf{A}_y~{\circ}~\mathbf{X}_{t[y]}
\end{equation}
%
where $\mathbf{C}_{y}$ is a $4\times 4$ matrix of annual contributions,
$\mathbf{X}_{t[y]}$ is a $4\times 4$ matrix 
with idententical rows ${\mathbf{x}_{t[y]}}^\text{T}$,
and ${\circ}$ denotes the Hadamard (element-wise) product. 
This is essentially the same operation as a standard demographic projection
(e.g., equation \ref{eq:XPX}) but without summing the contributions to a given state
from different sources states. 


We characterized the overall dynamics of the population in terms of the annual 
population growth rate $\lambda_y$, calculated as
%
\begin{equation} \label{eq:lam-n}
\lambda_y = \frac{N_y}{N_{y-1}}
\end{equation}
%
where $N_y$ is the summed abundance across basins and life stages in year $y$.
Temporal variation in $\lambda_y$ reflects both variation in the demographic rates
and transient fluctuations due to non-equilibrium state distribution. 
Therefore, it is also informative to calculate the asymptotic population growth rate that 
would obtian if the demographic rates in a given year were fixed indefinitely, 
which is equal to the leading eigenvalue of $\mathbf{A}_y$. }

